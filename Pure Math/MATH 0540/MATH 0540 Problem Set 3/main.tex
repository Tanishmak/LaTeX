\documentclass{article}
\usepackage[utf8]{inputenc}
\usepackage{setspace}
\usepackage{dsfont}
\usepackage{amsmath,amssymb}
\usepackage{graphicx}
\usepackage{multicol}

\graphicspath{ {./} }

\onehalfspacing

\title{MATH 0540 Problem Set 3}
\author{Collaborated with Esmé, Mariana, and Edward}
\date{September 2022}

\begin{document}

\maketitle

\section{A function $f : \mathds{R} \rightarrow \mathds{R}$ is called periodic if there exists a positive number $p$ such that $f(x) = f(x + p)$ for all $x \in \mathds{R}$. Is the set of periodic functions from $\mathds{R}$ to $\mathds{R}$ a subspace of $\mathds{R}^{\mathds{R}}$? Prove or disprove.}

\paragraph{\large
\\Let $P$ be the set of all periodic functions from $\mathds{R}$ to $\mathds{R}$.}

\paragraph{\large
$P$ is a subspace of $\mathds{R}^{\mathds{R}}$ if and only if:
\\\indent 1. $0 \in P$
\\\indent 2. $u,w \in P$ implies $u + w \in P$
\\\indent 3. $a \in \mathds{F}$, $u \in P$ implies $au \in P$}

\paragraph{\large
\begin{multicols}{2}
Let $f$ be a function:
\\\indent\indent $\mathds{R} \rightarrow \mathds{R}$
\\\indent\indent $x \rightarrow cos(x)$
\\\\\indent $\forall x \in \mathds{R}$, 
\\\indent $cos(x) = cos(x + 2\pi)$
\\\indent Thus, $f$ is periodic,
\\\indent meaning $f \in P$.
\columnbreak
\\Let $g$ be a function:
\\\indent $\mathds{R} \rightarrow \mathds{R}$
\\\indent $x \rightarrow cos(\pi x)$
\\\\$\forall x \in \mathds{R}$, 
\\$cos(\pi x) = cos(\pi x + 2)$
\\Thus, $g$ is periodic,
\\meaning $g \in P$.
\end{multicols}}

\paragraph{\large
Consider the following lemma:}

\paragraph{\large
$\forall j, k \in P$ with periods $p_j, p_k$ respectively, $j + k \in P$ if and only if $\exists$ non-zero $m,n \in \mathds{N}$ such that $mp_j = np_k$.}

\paragraph{\large
If $j + k$ is periodic, $\forall x \in \mathds{R}, \exists p \in \mathds{R}$ such that $(j + k)(x + p) = (j + k)(x)$. In other words, $j(x + p) + k(x + p) = j(x) + k(x)$, $\forall x \in \mathds{R}$. This implies that at some regular interval $p$, the periods of $j$ and $k$ must coincide. Therefore, if $j + k$ is periodic, it implies that $\exists$ non-zero $m,n \in \mathds{N}$ such that $mp_j = np_k$.}

\paragraph{\large
From the other side, if $j$ and $k$ are periodic with periods $p_{j}$ and $p_{k}$, and $\exists$ non-zero $m,n \in \mathds{N}$ such that $mp_j = np_k$, it follows that there is a certain least common multiple between $p_j$ and $p_k$. This implies that for some $p \in \mathds{R}$, $(j + k)(x + p) = (j + k)(x)$. In other words, $j + k$ is periodic.}

\paragraph{\large
Now consider $f$ and $g$ once more:
\\\indent The period of $f$ is $2\pi$.
\\\indent The period of $g$ is $2$.}

\paragraph{\large
$\forall$ non-zero $m,n \in \mathds{N}, m \cdot 2\pi \neq n \cdot 2$.
\\\\Thus, $f + g \notin P$, contradicting condition 2. from above. Therefore, $P$ cannot be a subspace of $\mathds{R}^{\mathds{R}}$.}

\newpage

\section{Let $V$ be the set of functions from $\mathds{R} \rightarrow \mathds{R}$. Let $U = \{f \in V \thickspace | \thickspace f(5) = 1\}$. Is this a linear subspace? Prove or disprove.}

\paragraph{\large
\\$U$ is a subspace of $V$ if and only if:
\\\indent 1. $0 \in U$
\\\indent 2. $u,w \in U$ implies $u + w \in U$
\\\indent 3. $a \in \mathds{F}$, $u \in U$ implies $au \in U$}

\paragraph{\large
Let $f$ and $g$ be functions:
\\\indent $\mathds{R} \rightarrow \mathds{R}$}

\paragraph{\large
$\forall x \in \mathds{R}$, consider $(f + g)(x)$:
\\\indent To have $0 \in U$, there must be some function $g$ 
\\\indent such that $f(x) + g(x) = f(x)$.}

\paragraph{\large
However, at $x = 5$, $f$ and $g$ must output $1$.
\\\indent In other words, $U$ contains no function $g$ such that 
\\\indent $g(5) = 0$, meaning it is impossible for $f(5) + g(5) = f(5)$ to 
\\\indent be true.}

\paragraph{\large
By this logic, the first condition for $U$ to be a subspace fails, implying that $U$ is not a linear subspace.}

\newpage

\section{Let $W$ be a subset of a vector space $V$. Prove or disprove: if $W + W = W$, then $W$ is a subspace of $W$.}

\paragraph{\large
\\Consider the empty set, $\emptyset$.
\\\indent $\emptyset \subset V$.
\\\indent $\emptyset + \emptyset = \emptyset$, since there are no vectors
\\\indent to add in the first place.}

\paragraph{\large
$\emptyset$ is a subspace if and only if $0 \in \emptyset$.}

\paragraph{\large
By definition, the empty set has no vectors, so it cannot contain $0$. Therefore, the statement that $W$ is a subspace of $W$ if $W + W = W$ is disproven.}

\newpage

\section{Suppose $U = \{(x, x, y, y) \in \mathds{F}^{4} \thickspace | \thickspace x, y \in F\}$. Find a subspace $W$ of $\mathds{F}^{4}$ such that $\mathds{F}^{4} = U \oplus W$.}

\paragraph{\large
\\$\mathds{F}^{4} = U \oplus W$ if:
\\\indent 1. $W \cap U = \{0\}$
\\\indent 2. $W + U = \mathds{F}^{4}$}

\paragraph{\large
Let $W = \{(0, a, a, b) \in \mathds{F}^{4} \thickspace | \thickspace a,b \in \mathds{F}\}$}

\paragraph{\large
$W$ is a subspace of $\mathds{F}^{4}$ if and only if:
\\\indent 1. $0 \in W$
\\\indent 2. $u,w \in W$ implies $u + w \in W$
\\\indent 3. $c \in \mathds{F}$, $u \in W$ implies $cu \in W$}

\paragraph{\large
Fixing $a$ and $b$ to $0$ results in $(0, 0, 0, 0)$, thus $0 \in W$.}

\paragraph{\large
$\forall a,a',b,b' \in \mathds{F}$,
\\\indent let $u,w \in W$ such that $u = (0, a, a, b)$ and $w = (0, a', a', b')$
\\\indent $ u + w = (0, a + a', a + a', b + b')$.
\\\indent By the definition of a field, $\mathds{F}$ is closed over addition, 
\\\indent meaning $a + a' \in \mathds{F}$, and $b + b' \in \mathds{F}$.
\\\indent This means $u + w \in W$.
\\\indent Therefore, $u,w \in W$ implies $u + w \in W$.}

\paragraph{\large
$\forall a,b,c \in \mathds{F}$,
\\\indent let $u \in W$ such that $u = (0, a, a, b)$
\\\indent $ cu = (0, a \cdot c, a \cdot c, b \cdot c)$.
\\\indent By the definition of a field, $\mathds{F}$ is closed over 
\\\indent multiplication, meaning $a \cdot c \in \mathds{F}$, and $b \cdot c \in \mathds{F}$.
\\\indent This means $cu \in W$.
\\\indent Therefore, $c \in \mathds{F}$ and $u \in W$ implies $cu \in W$.}

\paragraph{\large
Since all three conditions are met, $W$ is a subspace.}

\paragraph{\large
Consider $W \cap U$:
\\\indent $(x, x, y, y) = (0, a, a, b)$
\\\indent implies $0 = x = a = y$,
\\\indent meaning $W \cap U = \{0\}$}

\paragraph{\large
Now consider $W + U$:
\\\indent $W + U = \{(x, x + a, y + a, y + b) \thickspace | \thickspace a,b,x,y \in \mathds{F}\}$
\\\indent $\mathds{F}^{4} = \{(k, l, m, n) \thickspace | \thickspace k,l,m,n \in \mathds{F}\}$
\\\indent It follows that:
\\\indent $x = k$
\\\indent $l = x + b = k + b$
\\\indent $b = l - k$
\\\indent $m = y + b = y + l - k$
\\\indent $y = m - l + k$
\\\indent $n = y + a = m - l + k + a$
\\\indent $a = n - m + l - k$,
\\\indent meaning $W + U = \mathds{F}^{4}$.}

\paragraph{\large
Since both conditions are met, $\mathds{F}^{4} = U \oplus W$.}

\newpage

\section{Let $U1$ and $U2$ be subsets of a vector space $V$. Prove that $span(U1 \cap U2) \subset span(U1) \cap span(U2)$. Prove that if $U1$ and $U2$ are subspaces, then $span(U1 \cap
U2) = span(U1) \cap span(U2)$.}

\paragraph{\large
\\Let $V \in span(U1 \cap U2)$.
\\\\ This means $V \in \{a_{1}v_{1} + ... + a_{n}v_{n} \thickspace | \thickspace a \in \mathds{F}, v \in U1 \cap U2\}$.
\\\indent Since $v_{1}, ... ,v_{n} \in U1 \cap U2$,
\\\indent it follows that $v_{1}, ... ,v_{n} \in U1$ and $v_{1}, ... ,v_{n} \in U2$.
\\\indent Therefore, $V \in \{a_{1}v_{1} + ... + a_{n}v_{n} \thickspace | \thickspace a \in \mathds{F}, v \in U1\}$ and 
\\\indent $V \in \{a_{1}v_{1} + ... + a_{n}v_{n} \thickspace | \thickspace a \in \mathds{F}, v \in U2\}$.}

\paragraph{\large
Now consider $span(U1)$ and $span(U2)$ individually:
\\\indent $span(U1) = \{a_{1}v_{1} + ... + a_{n}v_{n} \thickspace | \thickspace a \in \mathds{F}, v \in U1\}$
\\\indent $span(U2) = \{a_{1}v_{1} + ... + a_{n}v_{n} \thickspace | \thickspace a \in \mathds{F}, v \in U2\}$}

\paragraph{\large
This means $V \in span(U1)$ and $V \in span(U2)$, which can be rewritten as $V \in span(U1) \cap span(U2)$. 
\\\\Finally, since $V \in span(U1 \cap U2)$, 
\\\indent it follows that $span(U1 \cap U2) \subset span(U1) \cap span(U2)$.}

\paragraph{\large
From the other side, since a subspace is closed under addition 
and multiplication, it follows that:
\\\indent $U1 = span(U1)$ and $U2 = span(U2)$.
\\\indent Thus, $U1 \cap U2 = span(U1) \cap span(U2)$,
\\\indent which means $span(U1) \cap span(U2) \subset U1 \cap U2$.}

\paragraph{\large
Finally,
\\\indent $U1 \cap U2 \subset span(U1 \cap U2)$.
\\\indent So, $span(U1) \cap span(U2) \subset U1 \cap U2 \subset span(U1 \cap U2)$,
\\\indent which simplifies to $span(U1) \cap span(U2) \subset span(U1 \cap U2)$.}

\paragraph{\large
Since $span(U1) \cap span(U2) \subset span(U1 \cap U2)$ and
\\ $span(U1 \cap U2) \subset span(U1) \cap span(U2)$, it must follow that
\\ $span(U1 \cap U2) = span(U1) \cap span(U2)$.}

\end{document}