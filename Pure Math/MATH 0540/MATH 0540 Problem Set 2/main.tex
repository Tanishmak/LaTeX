\documentclass{article}
\usepackage[utf8]{inputenc}
\usepackage{setspace}
\usepackage{dsfont}
\usepackage{amsmath,amssymb}
\usepackage{graphicx}

\graphicspath{ {./} }

\onehalfspacing

\title{MATH 0540 Problem Set 2}
\author{Collaborated with Esmé, Mariana, Michael, Kazuya, and Eric}
\date{September 2022}

\begin{document}

\maketitle

\section{Let \(\mathds{F}\) be a field. Prove that \(a \cdot b = 0\) if and only if \(a = 0\) or \(b = 0\).}

\paragraph{\large
\\Let \(a, a', b, b' \in \mathds{F}\) such that \(a', b'\) are the multiplicative inverses of \(a, b\) respectively.
\\\\If we assume \(a \cdot b = 0\), it follows that:
\\\indent \(a \cdot b \cdot b' = 0 \cdot b'\)
\\\indent \(a \cdot 1 = 0 \cdot b'\) (b' is multiplicative inverse of b)
\\\indent \(a = 0 \cdot b'\) (1 is multiplicative identity)
\\\indent \(a = 0\) (0 times a scalar is 0)
\\ Or,
\\\indent \(a' \cdot a \cdot b = a' \cdot 0\)
\\\indent \(1 \cdot b = a' \cdot 0\) (a' is multiplicative inverse of a)
\\\indent \(b = a' \cdot 0\) (1 is multiplicative identity)
\\\indent \(b = 0\) (0 times a scalar is 0)}

\paragraph{\large
Conversely, if we assume \(a = 0\), it follows that:
\\\indent \(a \cdot b = 0 \cdot b\)
\\\indent \(a \cdot b = 0\) (0 times a scalar is 0)
\\ Or, if we assume \(b = 0\),
\\\indent \(a \cdot b = a \cdot 0\)
\\\indent \(a \cdot b = 0\) (0 times a scalar is 0)}

\paragraph{\large
Thus, if \(a \cdot b = 0\) then \(a = 0\) or \(b = 0\), and if \(a = 0\) or \(b = 0\) then \(a \cdot b = 0\), meaning \(a \cdot b = 0\) if and only if \(a = 0\) or \(b = 0\).}

\newpage

\section{Show that there is a unique field with 4 elements (prove 3 axioms).}

\paragraph{\large
\\Beginning with the definition of multiplication for this field:}

\paragraph{\large
Given that
\\\indent 1. 0 times a scalar is 0.
\\\indent 2. 1 is the unique multiplicative identity.
\\\\We can deduce the following values in the multiplication table:}

\paragraph{\large
\begin{center}
\begin{tabular}{ c | c c c c }
    & 0 & 1 & x & y \\
    \hline
    0 & 0 & 0 & 0 & 0 \\
    1 & 0 & 1 & x & y \\
    x & 0 & x &  &  \\
    y & 0 & y &  &  \\
\end{tabular}
\end{center}}

\paragraph{\large
To determine the remaining four values, we must evaluate the following cases for \(x \cdot y\):
\\\indent 1. \(x \cdot y = 0\) 
\\\indent\indent (forces x or y to equal 0)
\\\indent 2. \(x \cdot y = x\)
\\\indent\indent (forces y to be the multiplicative identity of x)
\\\indent 3. \(x \cdot y = y\)
\\\indent\indent (forces x to be the multiplicative identity of y)
\\\indent 4. \(x \cdot y = 1\)
\\\indent\indent (by process of elimination, must be correct)
\\Thus, \(x \cdot y\) must equal 1, allowing two more elements to be added to the multiplication table:}

\paragraph{\large
\begin{center}
\begin{tabular}{ c | c c c c }
    & 0 & 1 & x & y \\
    \hline
    0 & 0 & 0 & 0 & 0 \\
    1 & 0 & 1 & x & y \\
    x & 0 & x &  & 1 \\
    y & 0 & y & 1 &  \\
\end{tabular}
\end{center}}

\paragraph{\large
For multiplication to be closed on this field, each row and column must list each element exactly once. Therefore, the final two values can be deduced, thereby completing the table:}

\paragraph{\large
\begin{center}
\begin{tabular}{ c | c c c c }
    & 0 & 1 & x & y \\
    \hline
    0 & 0 & 0 & 0 & 0 \\
    1 & 0 & 1 & x & y \\
    x & 0 & x & y & 1 \\
    y & 0 & y & 1 & x \\
\end{tabular}
\end{center}}

\paragraph{\large
Now, moving to the definition of addition for this field:
\\\\Given that:
\\\indent 1. 0 is the unique additive identity
\\\\ We can deduce the following values for the addition table:}

\paragraph{\large
\begin{center}
\begin{tabular}{ c | c c c c }
    & 0 & 1 & x & y \\
    \hline
    0 & 0 & 1 & x & y \\
    1 & 1 &  &  &  \\
    x & x &  &  &  \\
    y & y &  &  &  \\
\end{tabular}
\end{center}}

\paragraph{\large
To determine the remaining values, we must evaluate the following cases for \(x + 1\):
\\\indent 1. \(x + 1 = x\)
\\\indent\indent (forces a non-unique additive identity)
\\\indent 2. \(x + 1 = 1\)
\\\indent\indent (forces a non-unique additive identity)
\\\indent 3. \(x + 1 = 0\)
\\\indent 4. \(x + 1 = y\)
\\Thus, \(x + 1\) is either equal to \(0\) or \(y\).}

\paragraph{\large
Assuming commutativity is held and that \(x + 1 = 0\), we can fill in the following values in the addition table:}

\paragraph{\large
\begin{center}
\begin{tabular}{ c | c c c c }
    & 0 & 1 & x & y \\
    \hline
    0 & 0 & 1 & x & y \\
    1 & 1 &  & 0 &  \\
    x & x & 0 &  &  \\
    y & y &  &  &  \\
\end{tabular}
\end{center}}

\paragraph{\large
We also know that \(y + 1 \neq 1\) and \(y + 1 \neq y\) because these would create non-unique additive identities. Additionally, for addition to be complete over the field, \(y + 1 \neq 0\) because each row and column must list every element exactly once.
\\\\Therefore, by process of elimination, we know \(y + 1 = x\), allowing the following values to be filled with the assumption of commutativity:}

\paragraph{\large
\begin{center}
\begin{tabular}{ c | c c c c }
    & 0 & 1 & x & y \\
    \hline
    0 & 0 & 1 & x & y \\
    1 & 1 &  & 0 & x \\
    x & x & 0 &  &  \\
    y & y & x &  &  \\
\end{tabular}
\end{center}}

\paragraph{\large
At this point, we can make an assumption using the distributive property to verify whether \(x + 1 = 0\). If \(x + 1 = 0\), it follows that:
\\\indent \(y (x + 1) = y \cdot x + y \cdot 1\) (distributive property)
\\\indent \(y \cdot 0 = y \cdot x + y \cdot 1\) (get sum from addition table)
\\\indent \(0 = y \cdot x + y \cdot 1\) (0 times any scalar is 0)
\\\indent \(0 = 1 + y\) (get products from multiplication table)
\\\indent \(0 = x\) (get sum from addition table)
\\This is a contradiction, therefore \(x + 1 \neq 0\). 
\\\\By process of elimination, we can deduce that \(x + 1 = y\), allowing us to reach the following table assuming commutativity:
}

\paragraph{\large
\begin{center}
\begin{tabular}{ c | c c c c }
    & 0 & 1 & x & y \\
    \hline
    0 & 0 & 1 & x & y \\
    1 & 1 &  & y &  \\
    x & x & y &  &  \\
    y & y &  &  &  \\
\end{tabular}
\end{center}}

\paragraph{\large
Since \(x + 1 = y\), it follows that:
\\\indent \(y (x + 1) = y \cdot x + y \cdot 1\) (distributive property)
\\\indent \(y \cdot y = y \cdot x + y \cdot 1\) (get sum from addition table)
\\\indent \(x = y \cdot x + y \cdot 1\) (get product from multiplication table)
\\\indent \(x = 1 + y\) (get sums from addition table)
\\Thus, since we now know that \(y + 1 = x\), two more values can be added to the table while preserving commutativity:}

\paragraph{\large
\begin{center}
\begin{tabular}{ c | c c c c }
    & 0 & 1 & x & y \\
    \hline
    0 & 0 & 1 & x & y \\
    1 & 1 &  & y & x \\
    x & x & y &  &  \\
    y & y & x &  &  \\
\end{tabular}
\end{center}}

\paragraph{\large
The only possible result of \(1 + 1\) that keeps addition closed over this field is 0, meaning the table gains one more value:}

\paragraph{\large
\begin{center}
\begin{tabular}{ c | c c c c }
    & 0 & 1 & x & y \\
    \hline
    0 & 0 & 1 & x & y \\
    1 & 1 & 0 & y & x \\
    x & x & y &  &  \\
    y & y & x &  &  \\
\end{tabular}
\end{center}}

\paragraph{\large
Now, for the value of \(x + x\), it follows that:
\\\indent \(x + x = x(1 + 1)\) (distributive property)
\\\indent \(x + x = x \cdot 0\) (get sum from addition table)
\\\indent \(x + x = 0\) (0 times any scalar is 0)
\\Once more, we can add a value to the addition table:}

\paragraph{\large
\begin{center}
\begin{tabular}{ c | c c c c }
    & 0 & 1 & x & y \\
    \hline
    0 & 0 & 1 & x & y \\
    1 & 1 & 0 & y & x \\
    x & x & y & 0 &  \\
    y & y & x &  &  \\
\end{tabular}
\end{center}}

\paragraph{\large
Finally, it is possible to deduce the rest of the table by assuming commutativity and that addition is closed. Thus, addition over this field can be defined as:}

\paragraph{\large
\begin{center}
\begin{tabular}{ c | c c c c }
    & 0 & 1 & x & y \\
    \hline
    0 & 0 & 1 & x & y \\
    1 & 1 & 0 & y & x \\
    x & x & y & 0 & 1 \\
    y & y & x & 1 & 0 \\
\end{tabular}
\end{center}}

\paragraph{\large
Using the definitions of addition and multiplication over this field, the following axioms will be proven:
\\\indent 1. Commutativity
\\\indent 2. Identities
\\\indent 3. Inverses}

\paragraph{\large
Using the completed addition and multiplication tables, it follows that:
\\\indent \(0 + 0 = 0 = 0 + 0\)
\\\indent \(1 + 0 = 1 = 0 + 1\)
\\\indent \(x + 0 = x = 0 + x\)
\\\indent \(y + 0 = y = 0 + y\)
\\\indent \(1 + 1 = 0 = 1 + 1\)
\\\indent \(x + 1 = y = 1 + x\)
\\\indent \(y + 1 = x = 1 + y\)
\\\indent \(x + x = 0 = x + x\)
\\\indent \(y + x = 1 = x + y\)
\\\indent \(y + y = 0 = y + y\)
\\and:
\\\indent \(0 \cdot 0 = 0 = 0 \cdot 0\)
\\\indent \(1 \cdot 0 = 0 = 0 \cdot 1\)
\\\indent \(x \cdot 0 = 0 = 0 \cdot x\)
\\\indent \(y \cdot 0 = 0 = 0 \cdot y\)
\\\indent \(1 \cdot 1 = 1 = 1 \cdot 1\)
\\\indent \(x \cdot 1 = x = 1 \cdot x\)
\\\indent \(y \cdot 1 = y = 1 \cdot y\)
\\\indent \(x \cdot x = y = x \cdot x\)
\\\indent \(y \cdot x = 1 = x \cdot y\)
\\\indent \(y \cdot y = x = y \cdot y\)
\\Thus, commutativity holds for all addition and multiplication operations over the field. Additionally, 0 is the unique additive identity, and 1 is the unique multiplicative identity for \(0,1,x,y \in \mathds{F}_{4}\). Finally, \(\forall k \in \mathds{F}_{4}, \exists v \in \mathds{F}_{4} \thickspace s.t. \thickspace k + v = 0\), and \(\forall k \in \mathds{F}_{4}, \exists w \in \mathds{F}_{4} \thickspace s.t. \thickspace k \cdot w = 1\).}

\newpage

\section{Let \(X\) be any set. Let \(V\) be the set of all subsets of \(X\).\\\\For \(A, B \subset X\), define \(A + B = (A \cup B) \setminus (A \cap B)\)\\\\We define a scalar multiplication on \(V\), with scalars \(\mathds{F}_{2} = {0, 1}\), by defining \(0 \cdot A = \emptyset, 1 \cdot A = A\). Show that \(V\) is a vector space over \(\mathds{F}_{2}\).}

\paragraph{\large
\\To prove that \(V\) is a vector space over \(\mathds{F}_{2}\), the following axioms will proven:
\\\indent 1. Commutativity
\\\indent 2. Associativity
\\\indent 3. Additive Identity
\\\indent 4. Multiplicative Identity
\\\indent 5. Additive Inverse
\\\indent 6. Distributive Property}

\paragraph{\large
\(\forall A, B \subset X\):
\\\indent \(A + B = (A \cup B) \setminus (A \cap B)\)
\\\indent \(A + B = (B \cup A) \setminus (A \cap B)\) (union is commutative)
\\\indent \(A + B = (B \cup A) \setminus (B \cap A)\) (intersection is commutative)
\\\indent \(A + B = B + A\) (definition of addition),
\\and
\\\indent \(0 \cdot A = \emptyset = A \cdot 0\) (0 times a set is empty set)
\\\indent \(1 \cdot A = A = A \cdot 1\) (1 is unique multiplicative identity)
\\ Therefore, addition and multiplication are commutative.}

\paragraph{\large
\(\forall A, B, C \subset X:\)
\\\includegraphics[scale = 0.06]{IMG_1518.png}
\includegraphics[scale = 0.06]{IMG_1519.png}
\\\indent meaning \((A + B) + C = A + (B + C)\),
\\and
\\\indent \((1 \cdot A) \cdot 1 = 1 \cdot (A \cdot 1)\) (definition of associative property)
\\\indent \(A \cdot 1 = 1 \cdot A\) (1 is the multiplicative identity)
\\\indent \(A = A\) (1 is the multiplicative identity)
\\and
\\\indent \((0 \cdot A) \cdot 0 = 0 \cdot (A \cdot 0)\) (associative property)
\\\indent \(\emptyset \cdot 0 = 0 \cdot \emptyset\) (0 times a set is the empty set)
\\\indent \(\emptyset = \emptyset\) (0 times a set is the empty set)
\\and
\\\indent \((1 \cdot A) \cdot 0 = 1 \cdot (A \cdot 0)\) (associative property)
\\\indent \(A \cdot 0 = 1 \cdot \emptyset\) (definition of scalar multiplication)
\\\indent \(\emptyset = \emptyset\) (definition of scalar multiplication)
\\and
\\\indent \((0 \cdot A) \cdot 1 = 0 \cdot (A \cdot 1)\) (associative property)
\\\indent \(\emptyset \cdot 1 = 0 \cdot A\) (definition of scalar multiplication)
\\\indent \(\emptyset = \emptyset\) (definition of scalar multiplication)
\\Thus, associativity holds for addition and multiplication.}

\paragraph{\large
\(\forall A \subset X:\)
\\\indent \(A + \emptyset = (A \cup \emptyset) \setminus (A \cap \emptyset)\) (definition of addition)
\\\indent \(A + \emptyset = A \setminus (A \cap \emptyset)\) (definition of union)
\\\indent \(A + \emptyset = A \setminus \emptyset\) (definition of intersection)
\\\indent \(A + \emptyset = A\) (definition of set-difference)
\\Therefore, there is a unique additive identity, \(\emptyset\).}

\paragraph{\large
\(\forall A \subset X:\)
\\\indent \(1 \cdot A = A\) (definition of multiplication by 1)
\\Therefore, there is a unique multiplicative identity, 1.}

\paragraph{\large
\(\forall A \subset X:\)
\\\indent \(A + A = (A \cup A) \setminus (A \cap A)\) (definition of addition)
\\\indent \(A + A = A \setminus (A \cap A)\) (definition of union)
\\\indent \(A + A = A \setminus A\) (definition of intersection)
\\\indent \(A + A = \emptyset\) (definition of set-difference)
\\Therefore, every element is its own additive inverse.}

\paragraph{\large
\(\forall A, B \subset X\), if the distributive property holds, it follows that:
\\\indent \(0 (A + B) = 0 \cdot A + 0 \cdot B\) (distributive property)
\\\indent \(\emptyset = \emptyset + \emptyset\) (definition of multiplication by 0)
\\\indent \(\emptyset = \emptyset\) (the empty set is the additive identity)
\\and
\\\indent \(1(A + B) = 1 \cdot A + 1 \cdot B\) (distributive property)
\\\indent \(A + B = A + B\) (definition of multiplication by 1)
\\and
\\\indent \(A(1 + 1) = A \cdot 1 + A \cdot 1\) (distributive property)
\\\indent \(A(1 + 1) = A + A\) (1 is the multiplicative identity)
\\\indent \(A(1 + 1) = \emptyset\) (each set is its own additive inverse)
\\\indent \(A \cdot 0 = \emptyset\) (definition of addition in \(\mathds{F}_{2}\))
\\\indent \(\emptyset = \emptyset\) (0 times a set is the empty set)
\\and 
\\\indent \(A(1 + 0) = A \cdot 1 + A \cdot 0\) (distributive property)
\\\indent \(A(1 + 0) = A + \emptyset\) (definition of scalar multiplication)
\\\indent \(A(1 + 0) = A\) (the empty set is the additive identity)
\\\indent \(A \cdot 1 = A\) (0 is additive identity)
\\\indent \(A = A\) (1 is the multiplicative identity)
\\and
\\\indent \(A(0 + 1) = A \cdot 0 + A \cdot 1\) (distributive property)
\\\indent \(A(0 + 1) = \emptyset + A\) (definition of scalar multiplication)
\\\indent \(A(0 + 1) = A\) (the empty set is the additive identity)
\\\indent \(A \cdot 1 = A\) (0 is additive identity)
\\\indent \(A = A\) (1 is multiplicative identity)
\\and
\\\indent \(A(0 + 0) = A \cdot 0 + A \cdot 0\) (distributive property)
\\\indent \(A(0 + 0) = \emptyset + \emptyset\) (0 times a set is empty set)
\\\indent \(A(0 + 0) = \emptyset\) (empty set is additive identity)
\\\indent \(A \cdot 0 = \emptyset\) (0 is additive identity)
\\\indent \(\emptyset = \emptyset\) (0 times a set is empty set)
\\Therefore, the distributive property holds.}

\newpage

\section{Let \(V\) be the set of all functions from \(\mathds{R} \rightarrow \mathds{R}\). For \(f, g \in V\), define \((f + g)(x) = f(x) + g(x)\), and for any \(a \in \mathds{R}\), define scalar multiplication by \((a \cdot f)(x) = f(ax)\). Prove or disprove that \(V\) is a vector space.}

\paragraph{\large
\\Let \(f\) be a function:
\\\indent \(\mathds{R} \rightarrow \mathds{R}\)
\\\indent \(x \rightarrow 3\)}

\paragraph{\large
If the distributive property holds, it follows that \(\forall a, b \in \mathds{R}\):
\\\indent \(((a + b) \cdot f)(x) = (a \cdot f)(x) + (b \cdot f)(x)\)
\\\indent \(f((a+b)x) = f(ax) + f(bx)\)
\\\indent \(3 = 3 + 3\) (definition of function f)
\\This would force a non-unique additive identity, meaning that the distributive property does not hold. Therefore, \(V\) cannot be a vector space.}

\newpage

\section{Let \(V\) be a vector space. Prove that \(-(-v) = v\)  for every \(v \in V\).}

\paragraph{\large
\\\(\forall v \in V\), let \(v, -v\) be additive inverses of each other, and let \(-(-v)\) be the additive inverse of \(-v\):
\\\indent \(v + -v = 0\) (definition of additive inverses)
\\\indent \(-(-v) + -v = 0\) (definition of additive inverses)
\\\indent \(-(-v) = v\) (transitive property)}

\end{document}