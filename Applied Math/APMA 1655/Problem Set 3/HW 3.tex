% me=0 student solutions (ps file), me=1 - my solutions (sol file),
% me=2 - assignment (hw file)
\def\me{1} \def\num{3} %homework number

\def\due{11 pm, February 24} %due date

\def\course{APMA 1655 Honors Statistical Inference I} 

% **** INSERT YOUR NAME HERE ****
\def\name{Tanish Makadia}


% **** INSERT NAMES OF YOUR COLLABORATORS HERE, OR JUST PUT N/A ****
\def\collabs{Taj Gillin}

%

\documentclass[11pt]{article}


% ==== Packages ====
\usepackage{amsfonts}
\usepackage{latexsym}
\usepackage{fullpage}
\usepackage{amsmath}
\usepackage{amssymb}
\usepackage{hyperref}
\usepackage{euler}
\usepackage{amsthm}


% \setlength{\oddsidemargin}{.0in} \setlength{\evensidemargin}{.0in}
% \setlength{\textwidth}{6.5in} \setlength{\topmargin}{-0.4in}
\setlength{\footskip}{1in} \setlength{\textheight}{8.5in}

\newcommand{\handout}[5]{
\renewcommand{\thepage}{#5, Page \arabic{page}}
  \noindent
  \begin{center}
    \framebox{ \vbox{ \hbox to 5.78in { {\bf \course} \hfill #2 }
        \vspace{4mm} \hbox to 5.78in { {\Large \hfill #5 \hfill} }
        \vspace{2mm} \hbox to 5.78in { {\it #3 \hfill #4} }
       
        \vspace{2mm} \hbox to 5.78in { {\it Collaborators: \collabs
            \hfill} }
           
        \vspace{2mm}
        \begin{itemize}
       
       \item You are strongly encouraged to work in groups, but solutions must be written independently. 

        \end{itemize}
      } }
  \end{center}
  
  \vspace*{4mm}
}


\newcounter{pppp}
\newcommand{\prob}{\arabic{pppp}} %problem number
\newcommand{\increase}{\addtocounter{pppp}{1}} %problem number

% Arguments: Title, Number of Points
\newcommand{\newproblem}[2]{
    \increase
    \section*{Problem \prob~(#1) \hfill {#2}}
}



\def\squarebox#1{\hbox to #1{\hfill\vbox to #1{\vfill}}}
\def\qed{\hspace*{\fill}
  \vbox{\hrule\hbox{\vrule\squarebox{.667em}\vrule}\hrule}}
\newenvironment{solution}{\begin{trivlist}\item[]{\bf Solution:}}
  {\qed \end{trivlist}}
\newenvironment{solsketch}{\begin{trivlist}\item[]{\bf Solution
      Sketch:}} {\qed \end{trivlist}}
\newenvironment{code}{\begin{tabbing}
    12345\=12345\=12345\=12345\=12345\=12345\=12345\=12345\= \kill }
  {\end{tabbing}}

%\newcommand{\eqref}[1]{Equation~(\ref{eq:#1})}

\newcommand{\hint}[1]{({\bf Hint}: {#1})}
% Put more macros here, as needed.
\newcommand{\room}{\medskip\ni}
\newcommand{\brak}[1]{\langle #1 \rangle}
\newcommand{\bit}[1]{\{0,1\}^{#1}}
\newcommand{\zo}{\{0,1\}}
\newcommand{\C}{{\cal C}}
\newcommand{\p}{\mathbb{P}}
\newcommand{\tp}{\tilde{\p}}

\newcommand{\nin}{\not\in}
\newcommand{\set}[1]{\{#1\}}
\renewcommand{\ni}{\noindent}
\renewcommand{\gets}{\leftarrow}
\renewcommand{\to}{\rightarrow}
\newcommand{\assign}{:=}

\newcommand{\AND}{\wedge}
\newcommand{\OR}{\vee}

\newcommand{\For}{\mbox{\bf for }}
\newcommand{\To}{\mbox{\bf to }}
\newcommand{\Do}{\mbox{\bf do }}
\newcommand{\If}{\mbox{\bf if }}
\newcommand{\Then}{\mbox{\bf then }}
\newcommand{\Else}{\mbox{\bf else }}
\newcommand{\While}{\mbox{\bf while }}
\newcommand{\Repeat}{\mbox{\bf repeat }}
\newcommand{\Until}{\mbox{\bf until }}
\newcommand{\Return}{\mbox{\bf return }}
\newcommand{\Halt}{\mbox{\bf halt }}
\newcommand{\Swap}{\mbox{\bf swap }}
\newcommand{\Ex}[2]{\textrm{exchange } #1 \textrm{ with } #2}



\begin{document}

\handout{}{\today}{Name: \name{}}{Due: \due}{Homework \num}


\section{Review}

To help you better answer the questions in HW 3, we review the example of \href{https://en.wikipedia.org/wiki/Bernoulli_distribution}{Bernoulli distributions} as follows:
\begin{itemize}
    \item The experiment of interest is flipping a fair coin;
    \item the sample space corresponding to this experiment is $\Omega=\{\texttt{heads}, \texttt{tails}\}$;
    \item the probability $\mathbb{P}$ is defined by $\mathbb{P}(A)=\frac{\# A}{\#\Omega}$, i.e., $\mathbb{P}(\{\texttt{heads}\})=\mathbb{P}(\{\texttt{tails}\})=\frac{1}{2}$;
    \item the random variable $X$ is defined by
    \begin{align*}
        X(\texttt{heads})=1,\ \ \ X(\texttt{tails})=0.
    \end{align*}
\end{itemize}
The CDF of $X$ is 
\begin{align}\label{eq: CDF of Bernoulli 1/2}
    F_X(x)=\left\{
    \begin{aligned}
        0,\ \ \ & \text{ if }x<0,\\
        \frac{1}{2},\ \ \ & \text{ if }0\le x<1, \\
        1,\ \ \ & \text{ if }x\ge1.
    \end{aligned}
    \right.
\end{align}
\textbf{Proof:}
    \begin{enumerate}
        \item When $x<0$, we have $A_x=\left\{\omega\in\Omega \,:\, X(\omega)\le x\right\}=\emptyset$; then, $F_X(x)=\mathbb{P}(A_x)=\mathbb{P}(\emptyset)=0$.
        \item When $0\le x <1$, we have $A_x=\left\{\omega\in\Omega \,:\, X(\omega)\le x\right\}=\{\texttt{tails}\}$; then, $F_X(x)=\mathbb{P}(A_x)=\mathbb{P}(\{\texttt{tails}\})=\frac{1}{2}$.
        \item When $x\ge1$, we have $A_x=\left\{\omega\in\Omega \,:\, X(\omega)\le x\right\}=\Omega$; then, $F_X(x)=\mathbb{P}(A_x)=\mathbb{P}(\Omega)=1$.
    \end{enumerate}
    The proof is completed. \qed

In addition, the \href{https://en.wikipedia.org/wiki/Random_variable}{Wikipedia page on random variables} is nice material for learning the concept of random variables.

\section{Problem Set}

\begin{enumerate}

\item Let $(\Omega,\mathbb{P})$ be a probability space. Suppose $B$ is an event and $0<\mathbb{P}(B)<1$. \textbf{Please prove the following:}
\begin{enumerate}
\item (1 point) If $A$ and $B$ are independent, then $A$ and $B^c$ are also independent. 

\begin{proof}
    Using (b), we have that \(\p(B^c\,|\,A)=1-\p(B\,|\,A)=1-\p(B)=\p(B^c)\).
\end{proof}

\item (1 point) $\mathbb{P}(A|B) + \mathbb{P}(A^c| B) = 1$.

\begin{proof}
    We will first prove that \(\p(A\cap B) = \p(B)-\p(A^c\cap B)\).
    \begin{align*}
        \p(A\cap B) &= \p(B\cap A) & \text{(commutativity)}\\
        &= 1-\p((B\cap A)^c) & \text{(def of complement)}\\
        &= 1-\p(B^c\cup A^c) & \text{(De Morgan's Law)}\\
        &= 1 - \p(B^c\cup A^c \cap \Omega) & \text{(\(E\cap\Omega = E\))}\\
        &= 1 - \p(B^c\cup A^c \cap (B\cup B^c)) & \text{(def of complement)}\\
        &= 1 - \p(B^c\cup (A^c\cap B)\cup (A^c\cap B^c)) &\text{(distributive law)}\\
        &= 1 - \p(B^c\cup (A^c\cap B^c)\cup (A^c\cap B)) & \text{(commutativity)}\\
        &= 1 - \p(B^c\cup (A^c\cap B)) & \text{(def of \(\cup\))}\\
        &= 1 - \p(B^c) - \p(A^c\cap B) & \text{(additivity)}\\
        &= \p(B) - \p(A^c\cap B)
    \end{align*}

    Now, we can use this relation to show that $\mathbb{P}(A|B) + \mathbb{P}(A^c| B) = 1$.
    \begin{align*}
        \p(A\,|\,B) &= \frac{\p(A\cap B)}{\p(B)} & \text{(conditional probability)}\\
        &= \frac{\p(B)-\p(A^c\cap B)}{\p(B)} & \text{(substitute from above)}\\
        &= \frac{\p(B)}{\p(B)}-\frac{\p(A^c\cap B)}{\p(B)} & \text{(distributive prop.)}\\
        &= 1 - \p(A^c\,|\,B) & \text{(conditional probability)}
    \end{align*}
\end{proof}
\end{enumerate}

\item (2 points) Let $n$ be a positive integer, and $\Omega \overset{\operatorname{def}}{=}\{1,2,\ldots,n\}$. Suppose $\mathbb{P}$ is a function of subsets of $\Omega$ defined as follows
\begin{align*}
\mathbb{P}(A)\overset{\operatorname{def}}{=}\frac{\# A}{\#\Omega}, \ \ \text{ for all }A\subset \Omega.
\end{align*}
You have proved in HW 1 that $(\Omega,\mathbb{P})$ is a probability space. 

We define a random variable $X$ as follows
\begin{align*}
    X(\omega)=\omega,\ \ \ \text{ for all }\omega\in\Omega=\{1,2,\ldots,n\}.
\end{align*}
\textbf{Please derive the CDF of the random variable $X$ defined above. Please present your answer using a formula like the one in Eq.~\eqref{eq: CDF of Bernoulli 1/2}.}

\begin{proof} Consider the following cases for the \(CDF\) of the random variable \(X\):
    \begin{itemize}
        \item When \(x<1\), we have that \(F_X(x)=\p(X\leq x)=\p(\emptyset)=0\).
        \item When \(1\leq x<n\), we have that \(F_X(x)=\p(X\leq x)=\p(\{1,\ldots,\lfloor x\rfloor\})=\frac{\lfloor x\rfloor}{n}\).
        \item When \(x\geq n\), we have that \(F_X(x)=\p(X\leq x)=\p(\Omega)=1\).
    \end{itemize}
    Therefore, \(F_X(x)=
    \begin{cases}
        0 & \text{if } x<1,\\
        \frac{\lfloor x\rfloor}{n} & \text{if } 1\leq x<n,\\
        1 & \text{if } x\geq n.
    \end{cases}\)
\end{proof}

\item (2 points) Let $X$ be a random variable defined on the probability space $(\Omega,\mathbb{P})$. Suppose $X$ satisfies the following
\begin{table}[h]
\centering
\begin{tabular}{c|ccccc} \hline
$x$ & 1 & 2 & 3 & 4 & 5 \\\hline
$\mathbb{P}(\{\omega\in\Omega: X(\omega)=x\})$ & 1/2 & 1/4 & 1/8 & 1/16 & 1/16 \\\hline
\end{tabular}
\end{table}

\textbf{Please derive the CDF of the random variable $X$. Please present your answer using a formula like the one in Eq.~\eqref{eq: CDF of Bernoulli 1/2}.}

\begin{proof}
    Consider the following cases for \(F_X(x)\):
    \begin{itemize}
        \item When \(x<1\), we have that \(F_X(x)=\p(X\leq x)=\p(\emptyset)=0\).
        \item When \(1\leq x<2\), we have that \(F_X(x)=\p(X\leq x)=0+1/2 = 1/2\).
        \item When \(2\leq x<3\), we have that \(F_X(x)=\p(X\leq x)=1/2 + 1/4 = 3/4\).
        \item When \(3\leq x<4\), we have that \(F_X(x)=\p(X\leq x)=3/4+1/8=7/8\).
        \item When \(4\leq x<5\), we have that \(F_X(x)=\p(X\leq x)=7/8+1/16=15/16\).
        \item When \(x\geq 5\), we have that \(F_X(x)=\p(X\leq x)=15/16+1/16=1\).
    \end{itemize}
    Therefore, \(F_X(x)=
    \begin{cases}
        0 & \text{if } x<1,\\
        1/2 & \text{if } 1\leq x<2,\\
        3/4 & \text{if } 2\leq x<3,\\
        7/8 & \text{if } 3\leq x<4,\\
        15/16 & \text{if } 4\leq x<5,\\
        1 & \text{if } x\geq 5.
    \end{cases}\)
\end{proof}

\item (2 points) Let $X$ be a random variable defined on the probability space $(\Omega,\mathbb{P})$. Suppose $X$ satisfies the following
\begin{align*}
    \mathbb{P}(\{\omega\in\Omega: X(\omega)=0\})=1.
\end{align*}
\textbf{Please derive the CDF of the random variable $X$. Please present your answer using a formula like the one in Eq.~\eqref{eq: CDF of Bernoulli 1/2}.}

\begin{proof}
    We have that \(\p(X=0)=1=\p(\Omega)\implies \{\omega\in\Omega\ |\ X(\omega)=0\}=\Omega\). Thus,
    \begin{itemize}
        \item When \(x<0\), we have that \(F_X(x)=\p(X\leq x)=\p(\emptyset)=0\).
        \item When \(x\geq 0\), we have that \(F_X(x)=\p(X\leq x)=\p(\Omega)=1\).
    \end{itemize}
    Therefore, \(F_X(x)=
    \begin{cases}
        0 & \text{if } x<0,\\
        1 & \text{if } x\geq 1.
    \end{cases}\)
\end{proof}

\item (2 points) Let $X$ be a random variable defined on the probability space $(\Omega,\mathbb{P})$. Suppose the CDF of $X$ is the following
\begin{align*}
    F_X(x)=\left\{
    \begin{aligned}
    0,\ \ \ &\text{ if }x<1;\\
    \log x,\ \ \ &\text{ if }1\le x<e;\\
    1,\ \ \ &\text{ if }e\le x.
    \end{aligned}
    \right.
\end{align*}
\textbf{Please compute the values of the following:}
\begin{enumerate}
    \item $\mathbb{P}(\{\omega\in\Omega: X(\omega)<2\})$;
    \begin{proof}
        \begin{align*}
            \p(X<2)&=\p(X\leq 2) - \p(X=2) & \text{(additivity)}\\
            &= \p(X\leq 2) & \text{(\(\p(X=2)=0\))}\\
            &= F_X(2) & \text{(def of \(CDF\))}\\
            &= \log(2)
        \end{align*}
    \end{proof}
    \item $\mathbb{P}(\{\omega\in\Omega: 0<X(\omega)\le3\})$;
    \begin{proof}
        \begin{align*}
            \p(0<X\leq 3) &= \p(X\leq 3) - \p(X\leq 0) & \text{(additivity)}\\
            &= F_X(3) - F_X(0) & \text{(def of \(CDF\))}\\
            &= 1 - 0 & \text{(def of \(F_X(x)\))}\\
            &= 1 & \text{(subtraction)}
        \end{align*}
    \end{proof}
    \item $\mathbb{P}(\{\omega\in\Omega: 2<X(\omega)<2.5\})$.
    \begin{proof}
        \begin{align*}
            \p(2<X<2.5) &= \p(X\leq 2.5) - \p(X=2.5) - \p(X\leq 2) & \text{(additivity)}\\
            &= \p(X\leq 2.5) - \p(X\leq 2) & \text{(\(\p(X=2.5)=0\))}\\
            &= F_X(2.5) - F_X(2) & \text{(def of \(CDF\))}\\
            &= \log(2.5) - \log(2) & \text{(def of \(F_X(x)\))}\\
            &= \log(1.25) & \text{(quotient rule)}
        \end{align*}
    \end{proof}
\end{enumerate}
\textbf{Remark:} For simplicity, many textbooks suppress the $\omega$ and represent  $\mathbb{P}(\{\omega\in\Omega: X(\omega)<2\})$, $\mathbb{P}(\{\omega\in\Omega: 0<X(\omega)\le3\})$, and $\mathbb{P}(\{\omega\in\Omega: 2<X(\omega)<2.5\})$ as $\mathbb{P}(X<2)$, $\mathbb{P}(0<X<3)$, and $\mathbb{P}(2<X<2.5)$, respectively. When you read those textbooks, this remark helps you understand what they mean.

\end{enumerate}


\newpage



\end{document}