\documentclass[12pt,reqno]{article}

%%%%%%%%%%%%%%%%%%%% PACKAGES %%%%%%%%%%%%%%%%%%%%

\usepackage[utf8]{inputenc}
\usepackage[all]{xy}
\usepackage[T1]{fontenc}
\usepackage[usenames, dvipsnames]{color}
\usepackage{setspace}
\usepackage{dsfont}
\usepackage{amssymb}
\usepackage{amsthm,bbm}
\usepackage{amscd}
\usepackage{amsfonts}
\usepackage{stmaryrd}
\usepackage{amsmath}
\usepackage{graphicx}
\usepackage{multicol}
\usepackage{xspace}
\usepackage{extarrows}
\usepackage{color}
\usepackage [english]{babel}
\usepackage [autostyle, english = american]{csquotes}
\usepackage[colorlinks, linktocpage, citecolor = red, linkcolor = blue]{hyperref}
\usepackage{fullpage}
\usepackage{color}
\usepackage{euler}

%%%%%%%%%%%%%%%%%%%% INITIALIZATION %%%%%%%%%%%%%%%%%%%%

\MakeOuterQuote{"}
\graphicspath{ {./} }
\setlength{\parskip}{\baselineskip}
\setlength{\parindent}{0pt}

%%%%%%%%%%%%%%%%%%%% COMMANDS %%%%%%%%%%%%%%%%%%%%

\newcommand{\range}{\mathrm{range\,}}
\newcommand{\nul}{\mathrm{null\,}}
\newcommand{\spn}{\mathrm{span\,}}
\newcommand{\card}{\mathrm{cardinality}}
\newcommand{\R}{\mathbb{R}}
\newcommand{\C}{\mathbb{C}}
\newcommand{\F}{\mathbb{F}}
\newcommand{\Z}{\mathbb{Z}}
\newcommand{\bd}{\mathrm{bd\,}}
\newcommand{\divline}{\hrule\vspace{12pt}\noindent}
\newcommand{\sgn}{\mathrm{sgn}}

%%%%%%%%%%%%%%%%%%%% ENVIRONMENTS %%%%%%%%%%%%%%%%%%%%

\theoremstyle{plain}
\newtheorem{maintheorem}{Theorem}
\renewcommand*{\themaintheorem}{\Alph{maintheorem}}

\newtheorem{theorem}{Theorem}[section] 
\newtheorem{lemma}{Lemma}
\newtheorem{corollary}[theorem]{Corollary}

\theoremstyle{definition}
\newtheorem{problem}{Problem}
\newtheorem{example}[theorem]{Example}
\newtheorem{definition}[theorem]{Definition}
\newtheorem{question}[theorem]{Question}

\newtheorem*{maintheorema}{Theorem \ref{thm:main}}

%%%%%%%%%%%%%%%%%%%% TITLE-PAGE %%%%%%%%%%%%%%%%%%%%

\title{MATH 1530 Problem Set 3}
\author{Tanish Makadia\\\small{(Collaborated with Esmé and Kazuya)}}
\date{February 2023}

%%%%%%%%%%%%%%%%%%%% DOCUMENT %%%%%%%%%%%%%%%%%%%%

\begin{document}
\maketitle

%%%%%%%%%%%%%%%%%%%% PROBLEM 1 %%%%%%%%%%%%%%%%%%%%

\begin{problem} 
    Consider $U(40)$. Find a subgroup which is cyclic of order 4. Find a subgroup which is noncyclic of order 4. 
\end{problem}
\begin{proof}
    \(U(40)\) is defined as follows:
    \begin{align*}
        U(40) = \{1,3,7,9,11,13,17,19,21,23,27,29,31,33,37,39\}
    \end{align*}
    We will now create a cyclic subgroup of order \(4\) using \(7\) as the generator.
    \begin{align*}
        \langle 7\rangle &= \{7,9,23,1\} & \text{(Gallian, 3.4)}
    \end{align*}
    Next, we will show that \(\{1,9,11,19\}\) is a noncyclic subgroup of \(U(40)\).
    
    Cayley Table:
    \begin{tabular}{c | c c c c}
        & 1 & 9 & 11 & 19\\
        \cline{1-5}
        1 & 1 & 9 & 11 & 19 \\
        9 & 9 & 1 & 19 & 11 \\
        11 & 11 & 19 & 1 & 9 \\
        19 & 19 & 11 & 9 & 1
    \end{tabular}

    Since \(\{1,9,11,19\}\) is finite and closed under multiplication mod \(40\), we have proven
    that it is subgroup of \(U(40)\). Additionally, since \(|9|=|11|=|19|=2\), there is no element
    in this group which can generate the entire group, proving that it is noncyclic.
\end{proof}

\newpage

%%%%%%%%%%%%%%%%%%%% PROBLEM 2 %%%%%%%%%%%%%%%%%%%%

\begin{problem}
    If $H$ and $K$ are subgroups of a group $G$, prove that $H \cap K$ is a subgroup of $G$. If $H \not \subset K$ and $K \not \subset H$, prove that $H \cup K$ is never a subgroup of $G$.
\end{problem}

\newpage

%%%%%%%%%%%%%%%%%%%% PROBLEM 3 %%%%%%%%%%%%%%%%%%%%

\begin{problem} 
    Prove that a group $G$ is Abelian if and only if $G = Z(G)$.
\end{problem}

\newpage

%%%%%%%%%%%%%%%%%%%% PROBLEM 4 %%%%%%%%%%%%%%%%%%%%

\begin{problem} 
    Suppose $G$ is a group with exactly 8 elements of order $3$. how many subgroups of order $3$ does $G$ have?
\end{problem}

\begin{proof}
    Let \(H\subset G\) be a subgroup of order 3:
    \begin{align*}
        H = \{e, a, b\}
    \end{align*}
    Since \(e\) is unique, we have that \(ab\neq a\) and \(ab\neq b\). In order for \(H\) to be closed, 
    the only remaining choice is \(ab = e\). Thus, for any subgroup of order \(3\), the two elements besides
    the identity must be each other's inverse.

    Now, we will show that \(|a|=|b|=3\). Consider \(a^2\in H\). Since the identity is unique, \(a^2\neq a\),
    and since \(b\) is the unique inverse of \(a\), the only remaining choice is \(a^2=b\). Therefore,
    \begin{equation*}
        \begin{aligned}[c]
            a^3 &= a \cdot a^2 \\
            &= a \cdot b\\
            &= e
        \end{aligned}
        \qquad\qquad
        \begin{aligned}[c]
            b^3 &= (a^2)^3\\
            &= (a^3)^2\\
            &= e
        \end{aligned}
    \end{equation*}
    We have that there are exactly \(8\) elements of \(G\) of order \(3\). Because a subgroup of order \(3\) 
    requires two distinct elements of order \(3\), we can conclude that the number of distinct subgroups of 
    order \(3\) in \(G\) is \(8 / 2 = 4\).

    

\end{proof}

\newpage

%%%%%%%%%%%%%%%%%%%% PROBLEM 5 %%%%%%%%%%%%%%%%%%%%

\begin{problem} 
    Let $G$ be a finite group with more than one element. Show that $G$ has an element of prime order.
\end{problem}

\end{document}