\documentclass[12pt,reqno]{article}

%%%%%%%%%%%%%%%%%%%% PACKAGES %%%%%%%%%%%%%%%%%%%%

\usepackage[utf8]{inputenc}
\usepackage[all]{xy}
\usepackage[T1]{fontenc}
\usepackage[usenames, dvipsnames]{color}
\usepackage{setspace}
\usepackage{dsfont}
\usepackage{amssymb}
\usepackage{amsthm,bbm}
\usepackage{amscd}
\usepackage{amsfonts}
\usepackage{stmaryrd}
\usepackage{amsmath}
\usepackage{graphicx}
\usepackage{multicol}
\usepackage{xspace}
\usepackage{extarrows}
\usepackage{color}
\usepackage [english]{babel}
\usepackage [autostyle, english = american]{csquotes}
\usepackage[colorlinks, linktocpage, citecolor = red, linkcolor = blue]{hyperref}
\usepackage{fullpage}
\usepackage{color}
\usepackage{euler}

%%%%%%%%%%%%%%%%%%%% INITIALIZATION %%%%%%%%%%%%%%%%%%%%

\MakeOuterQuote{"}
\graphicspath{ {./} }
%\setlength{\parskip}{\baselineskip}
\setlength{\parindent}{0pt}

%%%%%%%%%%%%%%%%%%%% COMMANDS %%%%%%%%%%%%%%%%%%%%

\newcommand{\range}{\mathrm{range\,}}
\newcommand{\nul}{\mathrm{null\,}}
\newcommand{\spn}{\mathrm{span\,}}
\newcommand{\card}{\mathrm{cardinality}}
\newcommand{\R}{\mathbb{R}}
\newcommand{\C}{\mathbb{C}}
\newcommand{\F}{\mathbb{F}}
\newcommand{\Z}{\mathbb{Z}}
\newcommand{\bd}{\mathrm{bd\,}}
\newcommand{\divline}{\hrule\vspace{12pt}\noindent}
\newcommand{\sgn}{\mathrm{sgn}}

%%%%%%%%%%%%%%%%%%%% ENVIRONMENTS %%%%%%%%%%%%%%%%%%%%

\theoremstyle{plain}
\newtheorem{maintheorem}{Theorem}
\renewcommand*{\themaintheorem}{\Alph{maintheorem}}

\newtheorem{theorem}{Theorem}[section] 
\newtheorem{lemma}{Lemma}
\newtheorem{corollary}[theorem]{Corollary}

\theoremstyle{definition}
\newtheorem{problem}{Problem}
\newtheorem{example}[theorem]{Example}
\newtheorem{definition}[theorem]{Definition}
\newtheorem{question}[theorem]{Question}

\newtheorem*{maintheorema}{Theorem \ref{thm:main}}

%%%%%%%%%%%%%%%%%%%% TITLE-PAGE %%%%%%%%%%%%%%%%%%%%

\title{APMA 1650 Problem Set 1}
\author{Tanish Makadia}
\date{February 2023}

%%%%%%%%%%%%%%%%%%%% DOCUMENT %%%%%%%%%%%%%%%%%%%%

\begin{document}
\maketitle

%%%%%%%%%%%%%%%%%%%% PROBLEM 1 %%%%%%%%%%%%%%%%%%%%

\begin{enumerate}
    \item \begin{enumerate}
            \item \(S = \{\text{turn left},\ \text{turn right},\ \text{continue straight}\}\)
            \item \(2/3\)
            \item \((2/3)^2=4/9\) \bigskip
    \end{enumerate}
    \item \begin{enumerate}
        \item \begin{enumerate}
            \item \(P(A\cup B) = P(A) + P(B) - P(A\cap B)=0.4+0.12-0.12=0.4\)
            \item \(P(A\cap B) = P(B) = 0.12\)
            \item \(P((A\cap B)^C)=1-P(A\cap B)=1-0.12=0.88\)
        \end{enumerate}
        \item \begin{enumerate}
            \item \(P(A\cup B) = P(A) + P(B) =0.4+0.12=0.52\)
            \item \(P(A\cap B) = 0\) (disjointness implies \(A\cap B=\emptyset,\ P(\emptyset)=0\))
            \item \(P((A\cap B)^C)=1 - P(A\cap B) = 1\) \bigskip
        \end{enumerate}
    \end{enumerate}
    \item \begin{enumerate}
        \item \(31/366\)
        \item \(12/366=2/61\)
        \item \(1/366\) \bigskip
    \end{enumerate}
    \item \begin{enumerate}
        \item \(S = \{(1,1),(1,2),(1,3),(1,4),(1,5),(1,6),(2,1),(2,2),(2,3),(2,4),(2,5),\\(2,6),(3,1),(3,2),(3,3),(3,4),(3,5),(3,6),4,5,6\}\)
        \item 21 elements
        \item \(A = \{(1,5),(2,4),(2,6),(3,3),(3,5),6\}\)
        \item 6 elements
        \item \(P(A) = 5(\frac{1}{36}) + \frac{1}{6} = \frac{11}{36}\)
        \item \(3/36 = 1/12\) \bigskip
    \end{enumerate}
    \item \begin{enumerate}
        \item \begin{proof}
            Let \(A\subseteq S\). By the definition of a complement, we have that \(A\cup A^C=S\)
            and \(A\cap A^C=\emptyset\). Thus, 
            \begin{align*}
               1 &= P(S) & \text{(\(P(S)=1\))}\\
               &= P(A\cup A^C) & \text{(definition of complement)}\\
               &=P(A)+P(A^C) & \text{(\(A\) and \(A^C\) disjoint)}
            \end{align*}
            Therefore, we can conclude that \(P(A)=1-P(A^C)\)
        \end{proof}
    \end{enumerate}
\end{enumerate}

%%%%%%%%%%%%%%%%%%%% PROBLEM 2 %%%%%%%%%%%%%%%%%%%%

%%%%%%%%%%%%%%%%%%%% PROBLEM 3 %%%%%%%%%%%%%%%%%%%%

%%%%%%%%%%%%%%%%%%%% PROBLEM 4 %%%%%%%%%%%%%%%%%%%%

%%%%%%%%%%%%%%%%%%%% PROBLEM 5 %%%%%%%%%%%%%%%%%%%%

%%%%%%%%%%%%%%%%%%%% PROBLEM 6 %%%%%%%%%%%%%%%%%%%%

\end{document}