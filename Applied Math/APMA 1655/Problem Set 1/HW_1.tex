% me=0 student solutions (ps file), me=1 - my solutions (sol file),
% me=2 - assignment (hw file)
\def\me{1} \def\num{1} %homework number

\def\due{11 pm, February 10} %due date

\def\course{APMA 1655 Honors Statistical Inference I} 

% **** INSERT YOUR NAME HERE ****
\def\name{Tanish Makadia}


% **** INSERT NAMES OF YOUR COLLABORATORS HERE, OR JUST PUT N/A ****
\def\collabs{Garv Gaur and Taj Gillin}

%

\documentclass[11pt]{article}


% ==== Packages ====
\usepackage{amsfonts}
\usepackage{latexsym}
\usepackage{fullpage}
\usepackage{amsmath}
\usepackage{amssymb}
\usepackage{hyperref}
\usepackage{euler}
\usepackage{amsthm}

% \setlength{\oddsidemargin}{.0in} \setlength{\evensidemargin}{.0in}
% \setlength{\textwidth}{6.5in} \setlength{\topmargin}{-0.4in}
\setlength{\footskip}{1in} \setlength{\textheight}{8.5in}

\newcommand{\handout}[5]{
\renewcommand{\thepage}{#5, Page \arabic{page}}
  \noindent
  \begin{center}
    \framebox{ \vbox{ \hbox to 5.78in { {\bf \course} \hfill #2 }
        \vspace{4mm} \hbox to 5.78in { {\Large \hfill #5 \hfill} }
        \vspace{2mm} \hbox to 5.78in { {\it #3 \hfill #4} }
       
        \vspace{2mm} \hbox to 5.78in { {\it Collaborators: \collabs
            \hfill} }
           
        \vspace{2mm}
        \begin{itemize}
       
       \item You are strongly encouraged to work in groups, but solutions must be written independently. 

        \end{itemize}
      } }
  \end{center}
  
  \vspace*{4mm}
}


\newcounter{pppp}
\newcommand{\prob}{\arabic{pppp}} %problem number
\newcommand{\increase}{\addtocounter{pppp}{1}} %problem number

% Arguments: Title, Number of Points
\newcommand{\newproblem}[2]{
    \increase
    \section*{Problem \prob~(#1) \hfill {#2}}
}



\def\squarebox#1{\hbox to #1{\hfill\vbox to #1{\vfill}}}
\def\qed{\hspace*{\fill}
  \vbox{\hrule\hbox{\vrule\squarebox{.667em}\vrule}\hrule}}
\newenvironment{solution}{\begin{trivlist}\item[]{\bf Solution:}}
  {\qed \end{trivlist}}
\newenvironment{solsketch}{\begin{trivlist}\item[]{\bf Solution
      Sketch:}} {\qed \end{trivlist}}
\newenvironment{code}{\begin{tabbing}
    12345\=12345\=12345\=12345\=12345\=12345\=12345\=12345\= \kill }
  {\end{tabbing}}

%\newcommand{\eqref}[1]{Equation~(\ref{eq:#1})}

\newcommand{\hint}[1]{({\bf Hint}: {#1})}
% Put more macros here, as needed.
\newcommand{\room}{\medskip\ni}
\newcommand{\brak}[1]{\langle #1 \rangle}
\newcommand{\bit}[1]{\{0,1\}^{#1}}
\newcommand{\zo}{\{0,1\}}
\newcommand{\C}{{\cal C}}
\newcommand{\p}{\mathbb{P}}
\newcommand{\z}{\mathbb{Z}}

\newcommand{\nin}{\not\in}
\newcommand{\set}[1]{\{#1\}}
\renewcommand{\ni}{\noindent}
\renewcommand{\gets}{\leftarrow}
\renewcommand{\to}{\rightarrow}
\newcommand{\assign}{:=}

\newcommand{\AND}{\wedge}
\newcommand{\OR}{\vee}

\newcommand{\For}{\mbox{\bf for }}
\newcommand{\To}{\mbox{\bf to }}
\newcommand{\Do}{\mbox{\bf do }}
\newcommand{\If}{\mbox{\bf if }}
\newcommand{\Then}{\mbox{\bf then }}
\newcommand{\Else}{\mbox{\bf else }}
\newcommand{\While}{\mbox{\bf while }}
\newcommand{\Repeat}{\mbox{\bf repeat }}
\newcommand{\Until}{\mbox{\bf until }}
\newcommand{\Return}{\mbox{\bf return }}
\newcommand{\Halt}{\mbox{\bf halt }}
\newcommand{\Swap}{\mbox{\bf swap }}
\newcommand{\Ex}[2]{\textrm{exchange } #1 \textrm{ with } #2}



\begin{document}

\handout{}{\today}{Name: \name{}}{Due: \due}{Homework \num}

Please feel free to use the following laws without proving them:

Let $A$, $B$, and $C$ be events. Then, we have
\begin{itemize}
\item (Commutative Law ) $A\cup B=B\cup A$,
\item (Commutative Law ) $A\cap B= B\cap A$,
\item (Associative Law) $(A\cup B)\cup B=A\cup (B\cup C)$,
\item (Associative Law) $(A\cap B)\cap C= A\cap(B\cap C)$,
\item (Distributive law) $(A\cup B)\cap C=(A\cap C)\cup (B\cap C)$,
\item (Distributive law) $(A\cap B)\cup C=(A\cup C)\cap (B\cup C)$.
\item Let $\{A_1, A_2,\ldots,A_n,\ldots\}$ be a sequence of events, then we have
\begin{align*}
& \left(\bigcup_{n=1}^\infty A_n\right)^c = \bigcap_{n=1}^\infty A_n^c, \ \ \  \left(\bigcap_{n=1}^\infty A_n\right)^c = \bigcup_{n=1}^\infty A_n^c.
\end{align*}
\end{itemize}


\newproblem{Set theory}{ }

Suppose we are interested in a sample space $\Omega$. Please review the following definitions
\begin{align*}
& \bigcup_{n=1}^\infty A_n=\left\{\omega\in\Omega \,:\, \mbox{ there exists at least one }n'\mbox{ such that }\omega\in A_{n'}\right\}, \\
& \bigcap_{n=1}^\infty A_n=\left\{\omega\in\Omega \,:\, \omega\in A_n \mbox{ for all }n=1,2,3,\ldots\right\}
\end{align*}
\begin{enumerate}
\item (0.5 points) We define a sequence $\{A_n\}_{n=1}^\infty=\{A_1, A_2,\ldots,A_n,\ldots\}$ of events as the following: 
\begin{align*}
& A_1=\Omega,\\
& A_n=\emptyset,\ \ \ \mbox{ for all }n=2,3,\ldots.
\end{align*}
\textbf{Please prove the following:}
\begin{align}\label{eq: infinite union example}
\Omega=\bigcup_{n=1}^\infty A_n.
\end{align}

\begin{proof}
  Since \(A_1=\Omega\), we immediately have that \(\Omega\subset \bigcup_{n=1}^\infty A_n\). Additionally,
  we have that for all \(A_i\in \{A_n\}^{\infty}_{n=1}\), \(A_i\subset\Omega\). Thus, \(\bigcup_{n=1}^\infty A_n\subset\Omega\)
  completing the double inclusion. Therefore, we have proven that \(\Omega = \bigcup_{n=1}^\infty A_n\) as desired.
\end{proof}

\item Let $E_1$ and $E_2$ be two events with $E_1\cap E_2=\emptyset$. We define a sequence $\{A_n\}_{n=1}^\infty=\{A_1, A_2,\ldots,A_n,\ldots\}$ of events as the following: 
\begin{align}\label{eq: E1 E2 question}
\begin{aligned}
& A_1=E_1,\\
& A_2=E_2,\\
& A_n=\emptyset,\ \ \ \mbox{ for all }n=3,4,\ldots.
\end{aligned}
\end{align}
\textbf{Please prove the following:}
\begin{enumerate}
\item (0.5 points) The sequence $\{A_n\}_{n=1}^\infty=\{A_1, A_2,\ldots,A_n,\ldots\}$ defined in Eq.~\eqref{eq: E1 E2 question} is mutually disjoint.

\begin{proof}
  Let \(A_j,A_k\in \{A_n\}^{\infty}_{n=1}\) such that \(j\neq k\). The possible combinations of \(A_j\) and \(A_k\) 
  can be expressed with the following cases: (we have that \(\cap\) is commutative, so the order of \(A_j\) and \(A_k\) does not matter)
  \begin{itemize}
    \item Case 1 (\(A_j = E_1\) and \(A_k = E_2\)): \(A_j\cap A_k=E_1\cap E_2=\emptyset\)
    \item Case 2 (\(A_j = E_1\) and \(A_k = \emptyset\)): \(A_j\cap A_k=E_1\cap \emptyset=\emptyset\)
    \item Case 3 (\(A_j = E_2\) and \(A_k = \emptyset\)): \(A_j\cap A_k=E_2\cap \emptyset=\emptyset\)
  \end{itemize}
  Therefore, since \(A_j,A_k\in \{A_n\}^{\infty}_{n=1} : j\neq k \implies j\cap k=\emptyset\), we have proven that \(\{A_n\}^{\infty}_{n=1}\)
  is mutually disjoint.
\end{proof}

\item (0.5 points) We have the following identity
\begin{align*}
E_1\cup E_2=\bigcup_{n=1}^\infty A_n,
\end{align*}
where $A_1,A_2,\ldots$ are defined in Eq.~\eqref{eq: E1 E2 question}.

\begin{proof}
  Since \(A_1=E_1\) and \(A_2=E_2\), we have that \(E_1\cup E_2\subset \bigcup_{n=1}^\infty A_n\).
  Additionally, for all \(A_i\in \{A_n\}^{\infty}_{n=1}\), we have that \(A_i\subset E_1\cup E_2\).
  Thus, \(\bigcup_{n=1}^\infty A_n \subset E_1\cup E_2\), completing the double inclusion. Therefore,
  we have proven that \(E_1\cup E_2 = \bigcup_{n=1}^\infty A_n\).
\end{proof}

\end{enumerate}

\item (1 points) Let $\Omega=\mathbb{R}=$ the collection of all real numbers. We define a sequence of events as follows
\begin{align}\label{eq: infinite intersection 0,1 example}
\begin{aligned}
A_n&=\left[0,1+\frac{1}{n}\right), \ \ \ \mbox{ for all }n=1,2,3,\ldots.
\end{aligned}
\end{align}
Please prove the following identity
\begin{align*}
[0,1]=\bigcap_{n=1}^\infty A_n,
\end{align*}
where $A_1, A_2, A_3,\ldots$ are defined in Eq.~\eqref{eq: infinite intersection 0,1 example}.

\noindent\textbf{Remark:} Please read the following explanation for notations:
\begin{align*}
\left[0,1+\frac{1}{n}\right)&=\left\{x\,:\, x\mbox{ is a real number such that }0\le x \mbox{ and }x<1+\frac{1}{n}\right\}\\
&= \mbox{the collection of real numbers that are no less than 0 but smaller than }1+\frac{1}{n};
\end{align*}
\begin{align*}
[0,1]&=\mbox{ the collection of real numbers that are no less than 0 but no higher than 1} \\
&=\left\{x\,:\, x\mbox{ is a real number such that }0\le x \mbox{ and }x\le 1\right\}.
\end{align*}
\end{enumerate}

\begin{proof}
  \(A_{n+1}\subsetneq A_n\)
  since \([0,1+\frac{1}{n+1}]\subsetneq [0,1+\frac{1}{n}]\). Thus,
  \[\bigcap_{n=1}^\infty A_n = \lim_{n\to\infty} A_n = \lim_{n\to\infty} [0, 1+\frac{1}{n}] = [0,1]\].
\end{proof}

\newproblem{Definition of Probability Spaces}{}

(1 point) Suppose $n$ is a fixed positive integer. We define the pair $(\Omega,\mathbb{P})$ as follows
\begin{itemize}
\item $\Omega=\{1,2,\cdots,n\}$.
\item For any $A\subset \Omega$, we define $\mathbb{P}(A)=\frac{\# A}{n}$, where $\# A$ denotes the number of elements in $A$.
\end{itemize}
\textbf{Please prove that the pair $(\Omega,\mathbb{P})$ defined herein is a probability space. }

\begin{proof}
  We will show that \((\Omega,\p)\) is a probability space by proving the following three axioms.
  \begin{itemize}
    \item \underline{\(\p(A\subset\Omega)\geq 0\)}: Let \(A\subset \Omega\). We have that \(\p(A)=\frac{\#A}{n}\geq 0\) since
    \(\#A\geq 0\) and \(n\geq 0\).
    \item \underline{\(\p(\Omega)=1  \)}: \(\p(\Omega)=\frac{n}{n}=1\)
    \item \underline{Countable Additivity}: Let \(A_1,\ldots,A_m\subset\Omega\) be mutually disjoint events.
    \begin{align*}
      \p(A_1\cup\cdots\cup A_m) &= \frac{\#A_1+\cdots+\#A_m}{n} & \text{(definition of \(\p\))}\\
      &= \frac{\#A_1}{n}+\cdots+\frac{\#A_m}{n} & \text{(common denominator)}\\
      &= \p(A_1)+\cdots+\p(A_m) & \text{(definition of \(\p\))}
    \end{align*}
  \end{itemize}
\end{proof}

\newproblem{Properties of $\mathbb{P}$}{ }

Let $(\Omega, \mathbb{P})$ be a probability space. Then, we have the following properties
\begin{enumerate}
\item (0 point) $\mathbb{P}(\emptyset)=0$, i.e., the probability of the impossible event is zero; 
\item (0 point) if two events $E_1$ and $E_2$ satisfy $E_1\cap E_2=\emptyset$, we have $\mathbb{P}(E_1\cup E_2)=\mathbb{P}(E_1)+ \mathbb{P}(E_2)$;
\item (0.5 points) suppose $A,B\subset\Omega$. If $A\subset B$, then $\mathbb{P}(A)\le \mathbb{P}(B)$;\footnote{Hint: If $A\subset B$, we have $B=(B\cap A^c)\cup A$; furthermore, $(B\cap A^c)$ and $A$ are disjoint.}

\begin{proof}
  Let \(A\subset B\). We have that \(B=(B\cap A^c)\cup A\) and \((B\cap A^c)\cap A=\emptyset\).
  Thus, \((B\cap A^c)\) and \(A\) are additive, which implies that \(\p(B)=\p((B\cap A^c)\cup A) = \p(B\cap A^c) + \p(A)\).
  Because \(\p(B\cap A^c)\geq 0\) by the definition of a probability space, it must be the case that
  \(\p(B)\geq \p(A)\implies \p(A)\leq\p(B)\).
\end{proof}

\item (0.5 points) $0\le \mathbb{P}\{A\} \le 1$ for any subsets $A \subset \Omega$; 

\begin{proof}
  Let \(A\subset\Omega\). Immediately, we have that \(\p(A)\leq\p(\Omega)\implies\p(A)\leq 1\). Since \(\emptyset\subset A\), we have that \(\p(\emptyset)\leq\p(A)\implies0\leq\p(A)\).
\end{proof}

\item (0.5 points) $\mathbb{P}(A^c)=1-\mathbb{P}(A)$.

\begin{proof}
  Let \(A\subseteq \Omega\). By the definition of a complement, we have that \(A\cup A^c=\Omega\)
  and \(A\cap A^c=\emptyset\). Thus,
  \begin{align*}
     1 &= P(\Omega) & \text{(\(P(\Omega)=1\))}\\
     &= P(A\cup A^c) & \text{(definition of complement)}\\
     &=P(A)+P(A^c) & \text{(\(A\) and \(A^c\) disjoint)}
  \end{align*}
  Therefore, we can conclude that \(P(A^c)=1-P(A)\)
\end{proof}

\item (1 point) for any $A,B\subset\Omega$, we have $\mathbb{P}\{A\cup B\}=\mathbb{P}\{A\}+\mathbb{P}\{B\}-\mathbb{P}\{A\cap B\}$;
\begin{proof}
  Let \(A,B\in\Omega\). 
  \begin{align*}
    A\cup B &= (A\cup B)\cap \Omega & \text{(definition of \(\cap\))}\\
    &= (A\cup B) \cap (A\cup A^c) & \text{(definition of complement)}\\
    &= A\cup (B\cap A^c) & \text{(distributive law)}
  \end{align*}

  Thus, \(\p(A\cup B)\) can be expressed as the probability of the union of two disjoint events.

  \[\p(A\cup B) = \p(A\cup (B\cap A^c)) = \p(A) + \p(B\cap A^c)\]

  Now, we can rewrite \(\p(B\cap A^c)\) using a relation derived from \(\p(B)\).

  \begin{align*}
    \p(B) &= \p(B\cap \Omega) & \text{(definition of \(\cap\))}\\
    &= \p(B\cap (A\cup A^C)) & \text{(definition of complement)}\\
    &= \p((B\cap A)\cup (B\cap A^c)) & \text{(distributive law)}\\
    &= \p(A\cap B) + \p(B\cap A^c) & \text{(\(p\) is additive))}
  \end{align*}
  Using this relation, we have that \(\p(B\cap A^c)=\p(B)-\p(A\cap B)\). By substituting this into 
  the first relation, we get \(\p(A\cup B)=\p(A) + \p(B) - \p(A\cap B)\) as desired.
\end{proof}

\item (1 point) for any sequence of subsets $\{A_n\}_{n=1}^\infty$, we have $\mathbb{P}\{\bigcup_{n=1}^\infty A_n\}\le\sum_{n=1}^\infty\mathbb{P}\{A_n\}$.\footnote{More precisely, we have the following: 
    \begin{align*}
      \mathbb{P}\left(\bigcup_{k=1}^n A_k\right)=\sum_{k=1}^n(-1)^{k-1}\left\{\sum_{1\le i_1<i_2<\cdots<i_k\le n} \mathbb{P}\left(\bigcap_{l=1}^k A_{i_l}\right)\right\};  
    \end{align*}
    However, the identity above is not usually used in applications. You do not need to prove this precise identity in HW 1.}

\begin{proof}
  Let \(\{B_n\}_{n=1}^\infty\) be a sequence of events such that
  \begin{align*}
    B_n &= A_n\setminus\left(\bigcup_{i=1}^{n-1} B_i\right) = A_n\cap\left(\bigcup_{i=1}^{n-1} B_i\right)^c
  \end{align*}
  We will first show that \(\{B_n\}_{n=1}^\infty\) is mutually disjoint. Let \(j,k \in\z_{>0}\) such that \(j<k\). By the definition of \(B_n\), we have
  \[B_k = A_k\setminus(B_1\cup\cdots\cup B_j\cup\cdots\cup B_{k-1})\implies B_j\cap B_k=\emptyset\]
  Therefore, \(\{B_n\}_{n=1}^\infty\) is mutually disjoint. Next, we will show that
  \(\bigcup_{n=1}^\infty A_n = \bigcup_{n=1}^\infty B_n\). For all \(B_k\in\{B_n\}_{n=1}^\infty\), we have that \(B_k\subset A_k\subset \bigcup_{n=1}^\infty A_n\) since
  \(B_k = A_k\setminus (\cdots)\). This implies that \(\bigcup_{n=1}^\infty B_n\subset \bigcup_{n=1}^\infty A_n\). Additionally, for all \(A_k\in\{A_n\}_{n=1}^\infty\),
  \[A_k = \left(A_k\setminus\left(\bigcup_{n=1}^{k-1} B_n\right)\right)\cup\left(\bigcup_{n=1}^{k-1} B_n\right) = B_k \cup\left(\bigcup_{n=1}^{k-1} B_n\right) = \bigcup_{n=1}^{k} B_n\subset \bigcup_{n=1}^{\infty} B_n\]
  This implies that \(\bigcup_{n=1}^\infty A_n\subset \bigcup_{n=1}^\infty B_n\), completing the double inclusion. Finally, we have that 
  \begin{align*}
    \p(\bigcup_{n=1}^\infty A_n) &= \p(\bigcup_{n=1}^\infty B_n) & \text{(above)}\\
    &= \sum_{n=1}^\infty \p(B_n) & \text{(\(\{B_n\}_{n=1}^\infty\) mutually disjoint)}\\
    &\leq \sum_{n=1}^\infty \p(A_n) & \text{since \(\p(B_n)\leq \p(A_n)\)}
  \end{align*}
\end{proof}
    
\end{enumerate}

Properties 1 and 2 were proved in class, and you do not need to prove them in HW 1. 

\textbf{Please prove Properties 3-7 above.}

\newproblem{Application of the Probability Properties}{ }

Let $(\Omega, \mathbb{P})$ be a probability space.

\begin{enumerate}
\item (1 point) Let $A$ and $B$ are two events. Suppose $B\subset A$. Please prove the following:
\begin{align*}
\mathbb{P}(A^c) \le \mathbb{P}(B^c).
\end{align*}

\begin{proof}
  Let \(B\subset A\). We have that
  \begin{align*}
    \p(B)&\leq\p(A) & \text{(definition of \(\p\))}\\
    1-\p(B^c)&\leq 1-\p(A^c) & \text{(definition of complement)}\\
    -\p(B^c)&\leq -\p(A^c) & \text{(subtraction)}\\
    \p(A^c)&\leq\p(B^c) & \text{(addition)}
  \end{align*}
\end{proof}

\item (1 point) Let $A$ and $B$ are two events. If $\mathbb{P}(A) = 0.7$ and $\mathbb{P}(B)=0.6$,  what is the smallest possible value of $\mathbb{P}(A\cup B)$? What is the largest possible value of $\mathbb{P}(A\cup B)$? 

The smallest possible value of \(\p(A\cup B)\) is when \(B\subset A\implies \p(A\cap B)\) is largest (\emph{i.e.} when \(\p(A\cap B)=\p(B)=0.6\)). In this case,
\[\p(A\cup B)=\p(A)+\p(B)-\p(A\cap B)=0.7 + 0.6 - 0.6 = 0.7\]

The largest possibe value is attainable when \(A\cup B=\Omega\):
\[\p(A\cup B)=\p(\Omega)=1\]

\item (1 point) Let $A$ and $B$ are two events. If $\mathbb{P}(A) = 0.7$ and $\mathbb{P}(B)=0.6$, what is the smallest possible value of $P(A\cap B)$? What is the largest possible value of $P(A\cap B)$? 

The smallest possible value of \(\p(A\cap B)\) is attainable when the overlap between \(A\) and \(B\) is smallest.
This occurs when \(A\cup B=\Omega\). Thus, \(\p(A\cap B)=0.6-0.3=0.3\).

The largest possible value of \(\p(A\cap B)\) occurs when \(B\subset A\). In this case, \(\p(A\cap B)=\p(B)=0.6\).

\end{enumerate}

\end{document}