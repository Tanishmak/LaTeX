\documentclass{article}
\usepackage[utf8]{inputenc}
\usepackage[utf8]{inputenc}
\usepackage{setspace}
\usepackage{dsfont}
\usepackage{amsmath,amssymb}
\usepackage{graphicx}
\usepackage{multicol}
\usepackage [english]{babel}
\usepackage [autostyle, english = american]{csquotes}

\MakeOuterQuote{"}
\graphicspath{ {./} }
\onehalfspacing

\title{MATH 0540 Problem Set 6}
\author{Collaborated with Esmé, Praccho, and Edward}
\date{October 2022}

\begin{document}
\maketitle

\section{Give an example of a linear map $T: \mathbb{R}^4 \rightarrow \mathbb{R}^4$ such that $\mathrm{range}(T) = \mathrm{null}(T)$. Why can't there be a map $T: \mathbb{R}^5 \rightarrow \mathbb{R}^5$ with $\mathrm{range}(T) = \mathrm{null}(T)$?}

\paragraph{\large
\\Define $T: \mathbb{R}^4 \rightarrow \mathbb{R}^4 \in L(V, W)$ such that
\\ $T(w, x, y, z) = (y, z, 0, 0)$.}

\paragraph{\large
Let's examine $\mathrm{range}(T)$:}

\begin{align*}
    \mathrm{range}(T) &= \{T(v)\;|\;v \in \mathbb{R}^4\} & \text{(definition of range)}\\
    &= \{T(w, x, y, z)\;|\;w,x,y,z \in \mathbb{R}\} & \text{(definition of $\mathbb{R}^4$)}\\
    &= \{(y, z, 0, 0) \in \mathbb{R}^4\} & \text{(definition of T)}\\
\end{align*}

\paragraph{\large
Let's examine $\mathrm{null}(T)$:}

\begin{align*}
    \mathrm{null}(T) &= \{v\;|\;T(v) = 0,\;v \in \mathbb{R}^4\} & \text{(definition of null space)}\\
    &= \{(w, x, y, z) \in \mathbb{R}^4\;|\;T(w, x, y, z) = (0,0,0,0)\} & \text{(definition of $\mathbb{R}^4$)}\\
    &= \{(w, x, y, z) \in \mathbb{R}^4\;|\;(y, z, 0, 0) = (0,0,0,0)\} & \text{(definition of T)}\\
    &= \{(w, x, 0, 0) \in \mathbb{R}^4\} & \text{(y,z implied to be 0)}
\end{align*}

\paragraph{\large
Since $w,x,y,z$ are all under the same constraints, we have shown that $\mathrm{range}(T)=\mathrm{null}(T)$.}

\paragraph{\large
Assume $T(\mathbb{R}^5, \mathbb{R}^5) \in L(V, W)$ such that $\mathrm{range}(T) = \mathrm{null}(T)$:}

\begin{align*}
    \mathrm{null}(T) &= \mathrm{range}(T) & \text{(definition of T)}\\
    \dim \mathrm{null}(T) &= \dim \mathrm{range}(T) & \text{(definition of dimension)}\\\\
    \dim \mathbb{R}^5 &= \dim \mathrm{null}(T) + \dim \mathrm{range}(T) & \text{(rank-nullity theorem)}\\
    5 &= \dim \mathrm{null}(T) + \dim \mathrm{range}(T) & \text{(definition of $\mathbb{R}^5$)}\\
    &= \dim \mathrm{null}(T) + \dim \mathrm{null}(T) & \text{(from above)}\\
    &= (1 + 1) \cdot \dim \mathrm{null}(T) & \text{(distributive property)}\\
    &= 2 \cdot \dim \mathrm{null}(T) & \text{(addition)}
\end{align*}

\paragraph{\large
The dimension of a vector space is equal to the length of its basis, which implies that $\dim \mathrm{null}(T)$ must be a whole number (Axler, 2.36). However, under this constraint, there is no possible value of $\dim \mathrm{null}(T)$ such that $5 = 2 \cdot \dim \mathrm{null}(T)$. Therefore, we have a contradiction, meaning there cannot be a map $T: \mathbb{R}^5 \rightarrow \mathbb{R}^5$ with $\mathrm{range}(T) = \mathrm{null}(T)$.}

\newpage

\section{Let $V$ and $W$ be vector spaces over a field $\mathbb{F}$, and suppose $V$ is finite-dimensional with $\dim V > 0$. Let $w \in W$ be any vector. Prove that there exists a linear map $T: V \rightarrow W$ with $\mathrm{range}(T) = \mathrm{span}(w)$.}

\paragraph{\large
\\Consider two vector spaces, $V$ and $W$, such that $V$ is finite-dimensional with $\dim V > 0$.}

\paragraph{\large
For all $T: V \rightarrow W \in L(V, W)$:
\\Let $\{v_1,...\,,v_n\}$ be the set of basis vectors of $V$.}

\paragraph{\large
It follows that $T(v) = T(a_1v_1+...+a_nv_n)$, since every $v \in V$ can be expressed as a linear combination of the basis vectors of $V$.}

\paragraph{\large
For any vector $w \in W$, define $T: V \rightarrow W$:
\\$T(v) = T(a_1v_1+...+a_nv_n) = a_1 \cdot w$.}

\paragraph{\large
Assuming additivity holds:}

\begin{align*}
    T(u) + T(v) &= T(u + v) & \text{(additivity)}\\
    &= T(a_1v_1+...+a_nv_n+b_1v_1+...+b_nv_n) & \text{(definition of basis)}\\
    &= T(a_1v_1+b_1v_1+...+a_nv_n+b_nv_n) & \text{(commutative property)}\\
    &= T((a_1+b_1)v_1+...+(a_n+b_n)v_n) & \text{(distributive property)}\\
    &= (a_1+b_1) \cdot w & \text{(definition of T)}\\
    &= a_1w + b_1w & \text{(distributive property)}\\
    &= T(u) + T(v) & \text{(definition of T)}
\end{align*}

\paragraph{\large
Assuming homogeneity holds:}

\begin{align*}
    \lambda T(v) &= T(\lambda v) & \text{(homogeneity)}\\
    &= T(\lambda (a_1v_1+...+a_nv_n)) & \text{(definition of basis)}\\
    &= T(\lambda a_1v_1+...+\lambda a_nv_n) & \text{(distributive property)}\\
    &= \lambda a_1w & \text{(definition of T)}\\
    &= \lambda T(v) & \text{(definition of T)}
\end{align*}

\paragraph{\large
Therefore, we have that $T$ is a linear map.}

\paragraph{\large
Next we will show:
\\$\mathrm{range}(T) = \mathrm{span}(w)$}

\begin{align*}
    \mathrm{range}(T) &= \{T(v)\;|\;v \in V\} & \text{(definition of range)}\\
    &= \{T(a_1v_1+...+a_nv_n)\;|\;a_i \in \mathbb{F}\} & \text{(definition of basis)}\\
    &= \{a_1 \cdot w\;|\;a_1 \in \mathbb{F}\} & \text{(definition of T)}\\
    &= \{\lambda \cdot w\;|\;\lambda \in \mathbb{F}\} & \text{(change $a_1$ to $\lambda$)}\\
    &= \mathrm{span}(w) & \text{(definition of span)}
\end{align*}

\paragraph{\large
Therefore, we have proven that for all vector spaces $V, W$ such that $V$ is finite-dimensional with $\dim V > 0$, for any vector $w \in W$ there exists a linear map $T: V \rightarrow W$ with $\mathrm{range}(T) = \mathrm{span}(w)$.}

\newpage

\section{Suppose $V$ and $W$ are both finite dimensional. Prove that there exists an injective linear map from $V$ to $W$ if and only if $\dim V \leq \dim W$.}

\paragraph{\large
\\Assume $\dim V \leq \dim W$:
\\
\\Let $v_1,...\,,v_n$ be the basis of $V$.
\\Let $w_1,...\,,w_m$ be the basis of $W$.
\\It follows that $n \leq m$, since the length of a basis equals the dimension of the vector space that it is a basis of.}

\paragraph{\large
Consider a linear transformation $T: V \rightarrow W \in L(V, W)$:
\\
\\ Next we will prove that $T$ can be injective:}

\begin{align*}
    T(r) &= T(s) & \text{(assumption)}\\
    T(r_1v_1+...+r_nv_n) &= T(s_1v_1+...+s_nv_n) & \text{(definition of basis)}\\
    T(r_1v_1)+...+T(r_nv_n) &= T(s_1v_1)+...+T(s_nv_n) & \text{(additivity)}\\
    r_1T(v_1)+...+r_nT(v_n) &= s_1T(v_1)+...+s_nT(v_n) & \text{(homogeneity)}\\
    r_1w_1+...+r_nw_n &= s_1w_1+...+s_nw_n & \text{(Axler, 3.5)}\\
    0 &= (s_1w_1+...+s_nw_n) - (r_1w_1+...+r_nw_n) & \text{(additive inverse)}\\
    0 &= s_1w_1+...+s_nw_n - r_1w_1-...-r_nw_n & \text{(distributive property)}\\
    0 &= s_1w_1-r_1w_1+...+s_nw_n-r_nw_n & \text{(commutativity)}\\
    0 &= (s_1-r_1)w_1+...+(s_n-r_n)w_n & \text{(distributive property)}\\
    0 &= s_i-r_i & \text{(w's are linearly independent)}\\
    r_i &= s_i & \text{(additive inverse)}\\
    r &= s & \text{(linear combinations are equivalent)}
\end{align*}

\paragraph{\large
Therefore, we have that $\dim V \leq \dim W$ implies the existence of an injective $T(V, W) \in L(V, W)$.}

\paragraph{\large
From the other way around, assuming there exists an injective linear map $T: V \rightarrow W \in L(V, W)$:}

\begin{align*}
    \dim V &= \dim \mathrm{null}(T) + \dim \mathrm{range}(T) & \text{(rank-nullity theorem)}\\
    &= \dim \{0\} + \dim \mathrm{range}(T) & \text{(definition of injectivity)}\\
    &= 0 + \dim \mathrm{range}(T) & \text{(0 has dimension 0)}\\
    &= \dim \mathrm{range}(T) & \text{(additive identity)}\\
    &\leq \dim W & \text{($\mathrm{range}(T)$ subspace of W)}
\end{align*}

\paragraph{\large
Therefore, we have that the existence of an injective linear map $T(V, W) \in L(V, W)$ implies $\dim V \leq \dim W$.}

\paragraph{\large
Since $\dim V \leq \dim W$ implies the existence of an injective $T(V, W) \in L(V, W)$, and the existence of an injective $T(V, W) \in L(V, W)$ implies $\dim V \leq \dim W$, we have that there exists an injective linear map from $V$ to $W$ if and only if $\dim V \leq \dim W$.}

\newpage

\section{Prove that there does not exist a linear map from $\mathbb{R}^5$ to $\mathbb{R}^2$ whose null space equals
$$
\{(x_1,x_2,x_3,x_4,x_5)\ |\ x_1 = 3 x_2,\ x_3 = x_4 = x_5\}.
$$}

\paragraph{\large
\\The given definition of the null space can be rewritten as:
\\ $\{(3x_2,x_2,x_3,x_3,x_3)\ |\ x_i \in \mathbb{R}\}$.
\\\\
This representation has two free variables, $x_2, x_3 \in \mathbb{R}$, which suggests a basis of the null space composed of two vectors.
\\\\
Fixing each variable to $0$ and the other to $1$ yields the following set of two vectors:
\\\\
$\{(3, 1, 0, 0, 0), (0, 0, 1, 1, 1)\}$}

\paragraph{\large
If this is a basis of the null space, the following two conditions must be met:
\\\indent 1. $\{(3, 1, 0, 0, 0), (0, 0, 1, 1, 1)\}$ is linearly independent.
\\\indent 2. $\{(3, 1, 0, 0, 0), (0, 0, 1, 1, 1)\}$ spans the null space.}

\paragraph{\large
First we will prove the set is linearly independent:}

\begin{align*}
0 = 
\lambda_1
\begin{pmatrix}
    3 \\ 1 \\ 0 \\ 0 \\ 0
\end{pmatrix}
+ \lambda_2
\begin{pmatrix}
    0 \\ 0 \\ 1 \\ 1 \\ 1
\end{pmatrix}
\end{align*}

\paragraph{\large
After distributing the coefficients across each vector, this linear combination can be modified into the following relations:
\\\indent 1. $0 = 3\lambda_1$
\\\indent 2. $0 = \lambda_1$
\\\indent 3. $0 = \lambda_2$}

\paragraph{\large
From equations 2 and 3, we have that if a linear combination of the set equals 0, the coefficients must also be 0. Thus, we have proven that $\{(3, 1, 0, 0, 0), (0, 0, 1, 1, 1)\}$ is linearly independent.}

\paragraph{\large
Next, we will show that the set spans the null space:}

\begin{align*}
\begin{pmatrix}
    3x_2 \\ x_2 \\ x_3 \\ x_3 \\ x_3
\end{pmatrix}
=
\lambda_1
\begin{pmatrix}
    3 \\ 1 \\ 0 \\ 0 \\ 0
\end{pmatrix}
+ \lambda_2
\begin{pmatrix}
    0 \\ 0 \\ 1 \\ 1 \\ 1
\end{pmatrix}
\end{align*}

\paragraph{\large
Distributing the coefficients yields the following relations:
\\\indent 1. $3x_2 = 3\lambda_1$
\\\indent 2. $x_2 = \lambda_1$\
\\\indent 3. $x_3 = \lambda_2$}

\paragraph{\large
Therefore, equations 2 and 3 prove that any $x_2$ and $x_3$ can be expressed using $\lambda_1$ and $\lambda_2$, implying that $\{(3, 1, 0, 0, 0), (0, 0, 1, 1, 1)\}$ spans the null space.}

\paragraph{\large
Because both conditions are met, we have that $\{(3, 1, 0, 0, 0), (0, 0, 1, 1, 1)\}$ is a basis of the null space. Since the dimension of a vector space is defined as the length of its basis, we can conclude that the dimension of the null space is 2, since its basis has 2 elements.}

\paragraph{\large
Assume there exists a $T: \mathbb{R}^5 \rightarrow \mathbb{R}^2 \in L(V, W)$ \\such that $\mathrm{null}(T) = \{(x_1,x_2,x_3,x_4,x_5)\ |\ x_1 = 3 x_2,\ x_3 = x_4 = x_5\}$:}

\begin{align*}
    \dim \mathbb{R}^5 &= \dim \mathrm{null}(T) + \dim \mathrm{range}(T) & \text{(rank-nullity theorem)}\\
    5 &= \dim \mathrm{null}(T) + \dim \mathrm{range}(T) & \text{($\dim \mathbb{R}^5 = 5$)}\\
    5 &= 2 + \dim \mathrm{range}(T) & \text{($\dim \mathrm{null}(T) = 2$)}\\
    3 &= \dim \mathrm{range}(T) & \text{(additive inverse)}\\
\end{align*}

\paragraph{\large
$T$ maps to $\mathbb{R}^2$. Thus, $\mathrm{range}(T)$ is a subspace of $\mathbb{R}^2$ (Axler, 3.19). This implies $\dim \mathrm{range}(T) \leq \dim \mathbb{R}^2 = 2$ (Axler, 2.38). However, we also have that $\dim \mathrm{range}(T) = 3$, which produces $3 \leq 2$. Therefore, we have a contradiction, proving that there does not exist a linear map from $\mathbb{R}^5$ to $\mathbb{R}^2$ whose null space equals $\{(x_1,x_2,x_3,x_4,x_5)\ |\ x_1 = 3 x_2,\ x_3 = x_4 = x_5\}$.}

\newpage

\section{Suppose $T$ is a linear map from $\mathbb{R}^4$ to $\mathbb{R}^2$ such that 
$$
\mathrm{null}(T) = \{(x_1,x_2,x_3,x_4) \in \mathbb{R}^4 \ |\ x_1 = 5x_2,\ x_3 = 7x_4\}.
$$
Prove that $T$ is surjective.}

\paragraph{\large
\\The given definition of the null space can be rewritten as
\\$\{(5x_2,x_2,7x_4,x_4) \in \mathbb{R}^4\}$.
\\
\\This representation has two free variables, $x_2, x_4 \in \mathbb{R}$, which suggests a basis of the null space composed of two vectors.
\\
\\ Fixing each variable to $0$ and the other to $1$ yields the following set of two vectors:
\\
\\$\{(5, 1, 0, 0), (0, 0, 7, 1)\}$}

\paragraph{\large
If this is a basis of the null space, the following two conditions must be met:
\\\indent 1. $\{(5, 1, 0, 0), (0, 0, 7, 1)\}$ is linearly independent.
\\\indent 2. $\{(5, 1, 0, 0), (0, 0, 7, 1)\}$ spans the null space.}

\paragraph{\large
First we will prove the set is linearly independent:}

\begin{align*}
0 = \lambda_1
\begin{pmatrix}
    5 \\ 1 \\ 0 \\ 0
\end{pmatrix}
+ \lambda_2
\begin{pmatrix}
    0 \\ 0 \\ 7 \\ 1
\end{pmatrix}
\end{align*}

\paragraph{\large
Distributing the coefficients across each vector results in the following relations:
\\\indent 1. $0 = 5\lambda_1$
\\\indent 2. $0 = \lambda_1$
\\\indent 3. $0 = 7\lambda_2$
\\\indent 4. $0 = \lambda_2$}

\paragraph{\large
From equations 2 and 4, we have that if a linear combi-
nation of the set equals 0, the coefficients must also be 0. Thus, we have proven that $\{(5, 1, 0, 0), (0, 0, 7, 1)\}$ is linearly independent.}

\paragraph{\large
Next, we will show that the set spans the null space:}

\begin{align*}
\begin{pmatrix}
    5x_2 \\ x_2 \\ 7x_4 \\ x_4
\end{pmatrix}
= \lambda_1
\begin{pmatrix}
    5 \\ 1 \\ 0 \\ 0
\end{pmatrix}
+ \lambda_2
\begin{pmatrix}
    0 \\ 0 \\ 7 \\ 1
\end{pmatrix}
\end{align*}

\paragraph{\large
Distributing the coefficients yields the following relations:
\\\indent 1. $5x_2 = 5\lambda_1$
\\\indent 2. $x_2 = \lambda_1$
\\\indent 3. $7x_4 = 7\lambda_2$
\\\indent 4. $x_4 = \lambda_2$}

\paragraph{\large
Therefore, equations 2 and 4 prove that any $x_2$ and $x_3$ can be expressed using $\lambda_1$ and $\lambda_2$, implying that $\{(5, 1, 0, 0), (0, 0, 7, 1)\}$ spans the null space.}

\paragraph{\large
Because both conditions are met, we have that $\{(5, 1, 0, 0), (0, 0, 7, 1)\}$ is a basis of the null space. Since the dimension of a vector space is defined as the length of its basis, we can conclude that the dimension of the null space is 2, since its basis has 2 elements.}

\paragraph{\large
Define $T: \mathbb{R}^4 \rightarrow \mathbb{R}^2 \in L(V, W)$ \\such that $\mathrm{null}(T) = \{(x_1,x_2,x_3,x_4) \in \mathbb{R}^4 \ |\ x_1 = 5x_2,\ x_3 = 7x_4\}$:}

\begin{align*}
    \dim \mathbb{R}^4 &= \dim \mathrm{null}(T) + \dim \mathrm{range}(T) & \text{(rank-nullity theorem)}\\
    4 &= \dim \mathrm{null}(T) + \dim \mathrm{range}(T) & \text{(definition of $\mathbb{R}^4$)}\\
    4 &= 2 + \dim \mathrm{range}(T) & \text{($\dim \mathrm{null}(T) = 2$)}\\
    2 &= \dim \mathrm{range}(T) & \text{(additive inverse)}\\
\end{align*}

\paragraph{\large
Next we will prove the following lemma:
\\ If $U$ is a subspace of $V$, and $\dim U = \dim V$, then $U = V$.
\\
\\Let $u_1,...\,,u_n$ be the basis of $U$.
\\Let $v_1,...\,,v_n$ be the basis of $V$.
\\
\\$u_1,...\,,u_n$ is linearly independent because it is a basis. Additionally, we have that $\mathrm{length}(u_1,...\,,u_n) = \dim U = \dim V$. Therefore, since a $u_1,...\,,u_n$ is a linearly independent set of vectors with a length equal to the dimension of $V$, we have that $u_1,...\,,u_n$ spans $V$. Thus, $u_1,...\,,u_n$ is a basis of $V$. Since $U$ and $V$ share a common basis, $U = V$.}

\paragraph{\large
Now consider $\dim \mathrm{range}(T)$ once more:}

\begin{align*}
    \dim \mathrm{range}(T) &= 2 & \text{(from above)}\\
    \dim \mathrm{range}(T) &= \dim \mathbb{R}^2 & \text{(definition of $\mathbb{R}^2$)}\\
    \mathrm{range}(T) &= \mathbb{R}^2 & \text{(lemma)}
\end{align*}

\paragraph{\large
Therefore, since the range of $T$ equals its codomain, we have that $T$ is surjective.}

\end{document}