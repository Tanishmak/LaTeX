\documentclass[12pt,reqno]{article}

%%%%%%%%%%%%%%%%%%%% PACKAGES %%%%%%%%%%%%%%%%%%%%

\usepackage[utf8]{inputenc}
\usepackage[all]{xy}
\usepackage[T1]{fontenc}
\usepackage[usenames, dvipsnames]{color}
\usepackage{setspace}
\usepackage{dsfont}
\usepackage{amssymb}
\usepackage{amsthm,bbm}
\usepackage{amscd}
\usepackage{amsfonts}
\usepackage{stmaryrd}
\usepackage{amsmath}
\usepackage{graphicx}
\usepackage{multicol}
\usepackage{xspace}
\usepackage{extarrows}
\usepackage{color}
\usepackage [english]{babel}
\usepackage [autostyle, english = american]{csquotes}
\usepackage[colorlinks, linktocpage, citecolor = red, linkcolor = blue]{hyperref}
\usepackage{fullpage}
\usepackage{color}
\usepackage{euler}
\usepackage{parskip}

%%%%%%%%%%%%%%%%%%%% INITIALIZATION %%%%%%%%%%%%%%%%%%%%

\MakeOuterQuote{"}
\graphicspath{ {./} }

%%%%%%%%%%%%%%%%%%%% COMMANDS %%%%%%%%%%%%%%%%%%%%

\newcommand{\range}{\mathrm{range\,}}
\newcommand{\nul}{\mathrm{null\,}}
\newcommand{\spn}{\mathrm{span\,}}
\newcommand{\card}{\mathrm{cardinality}}
\newcommand{\R}{\mathbb{R}}
\newcommand{\C}{\mathbb{C}}
\newcommand{\F}{\mathbb{F}}
\newcommand{\Z}{\mathbb{Z}}
\newcommand{\bd}{\mathrm{bd\,}}
\newcommand{\divline}{\hrule\vspace{12pt}\noindent}
\newcommand{\sgn}{\mathrm{sgn}}

%%%%%%%%%%%%%%%%%%%% ENVIRONMENTS %%%%%%%%%%%%%%%%%%%%

\theoremstyle{plain}
\newtheorem{maintheorem}{Theorem}
\renewcommand*{\themaintheorem}{\Alph{maintheorem}}

\newtheorem{theorem}{Theorem}[section] 
\newtheorem{lemma}{Lemma}
\newtheorem{corollary}[theorem]{Corollary}

\theoremstyle{definition}
\newtheorem{problem}{Problem}
\newtheorem{example}[theorem]{Example}
\newtheorem{definition}[theorem]{Definition}
\newtheorem{question}[theorem]{Question}

\newtheorem*{maintheorema}{Theorem \ref{thm:main}}

%%%%%%%%%%%%%%%%%%%% TITLE-PAGE %%%%%%%%%%%%%%%%%%%%

\title{MATH 1530 Problem Set 5}
\author{Tanish Makadia\\\small{(Collaborated with Esmé and Kazuya)}}
\date{March 2023}

%%%%%%%%%%%%%%%%%%%% DOCUMENT %%%%%%%%%%%%%%%%%%%%

\begin{document}
\maketitle

%%%%%%%%%%%%%%%%%%%% PROBLEM 1 %%%%%%%%%%%%%%%%%%%%

\begin{problem} 
    How many elements of order 6 are in $S_7$?
\end{problem}

\begin{proof}
    By (Gallian, 5.1), every permutation of a finite set can be expressed as a product of
    disjoint cycles. Additionally, by (Gallian, 5.3), the order of a permutation in disjoint
    cycle form is the \(lcm\) of lengths of the disjoint cycles. 
    
    Let \(P = \{s\in S_7\ |\ |s|=6\}\). We must find the cardinality of \(P\). Let \(p\in P\). From above, \(p\) must have a disjoint cycle form in which the \(lcm\) of
    the disjoint cycle lengths equals \(6\). Therefore, the disjoint cycle form of \(p\) must fall under one
    of the following cases (note that the order of the disjoint cycles does not matter since they are commutative):
    \begin{itemize}
        \item \textbf{Case 1 (lengths: \(2, 2, 3\)):} \(p = (a_1, a_2)(b_1, b_2)(c_1, c_2, c_3)\). In this case, the number of ways to construct \(p\) using
        elements of \(S_7\) is:
        \[\frac{1}{2}\left(\frac{7!}{5!\cdot 2}\cdot\frac{5!}{3!\cdot 2}\cdot\frac{3!}{3}\right)=210\]
        \item \textbf{Case 2 (lengths: \(3,2,1,1\)):} \(p = (a_1,a_2,a_3)(b_1,b_2)(c_1)(d_1)\). In this case, the number of ways to construct
        \(p\) is:
        \[\frac{7!}{4!\cdot 3}\cdot\frac{4!}{2!\cdot 2}=420\]
        \item \textbf{Case 3 (lengths: \(6,1\)):} \(p=(a_1,a_2,a_3,a_4,a_5,a_6)(b_1)\). In this case, the number of ways to construct
        \(p\) is:
        \[\frac{7!}{1!\cdot 6}=840\]
    \end{itemize}
    Therefore, the number of elements of order \(6\) in \(S_7\) is \(\mathrm{card}(P)=210+420+840=1470\).
\end{proof}

\newpage

%%%%%%%%%%%%%%%%%%%% PROBLEM 2 %%%%%%%%%%%%%%%%%%%%

\begin{problem} 
    Let $D_4$ denote the rigid operations on a square taking the square back to itself (i.e., the symmetries of the square). For example, rotating the square by $\pi$ is a rigid operation taking the square back to itself. This is called the \emph{dihedral group}, and it is a group under composition.
    
    Label the vertices of the square from 1 to 4. Use this to represent the elements of $D_4$ a subgroup of $S_4$ (that is, list the elements of $D_4$ using cycle notation). What is the order of $D_4?$ Is $D_4$ isomorphic to $S_4$?
\end{problem}

\newpage

%%%%%%%%%%%%%%%%%%%% PROBLEM 3 %%%%%%%%%%%%%%%%%%%%

\begin{problem}  
    Prove that a permutation with odd order must be an even permutation. Show that the converse is false.
\end{problem}

\newpage

%%%%%%%%%%%%%%%%%%%% PROBLEM 4 %%%%%%%%%%%%%%%%%%%%

\begin{problem} 
    Let $\mathbb{C}$ be the complex numbers and 
    $$
    M = \left \{
    \begin{bmatrix}
       a  & -b \\
       b  &  a
    \end{bmatrix}
    \middle| \  a,b \in \mathbb{R}
    \right \}.
    $$
    prove that $\mathbb{C}^*$ and $M^*$ (the nonzero elements of $M$), viewed as groups with multiplication, are isomorphic. 
\end{problem}

\newpage

%%%%%%%%%%%%%%%%%%%% PROBLEM 5 %%%%%%%%%%%%%%%%%%%%

\begin{problem} Let $G$ be a group. An isomorphism from $G$ to itself is called an \emph{automorphism} of $G$. Let $\mathrm{Aut}(G)$ denote the set of all automorphisms of $G$. This is a group under the operation of function composition.
    Find two groups $G$ and $H$ such that $G \not \approx H$ but $\mathrm{Aut}(G) \approx \mathrm{Aut}(H)$.
\end{problem}

\end{document}