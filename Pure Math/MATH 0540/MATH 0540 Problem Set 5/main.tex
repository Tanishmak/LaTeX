\documentclass{article}
\usepackage[utf8]{inputenc}
\usepackage[utf8]{inputenc}
\usepackage{setspace}
\usepackage{dsfont}
\usepackage{amsmath,amssymb}
\usepackage{graphicx}
\usepackage{multicol}
\usepackage [english]{babel}
\usepackage [autostyle, english = american]{csquotes}

\MakeOuterQuote{"}
\graphicspath{ {./} }
\onehalfspacing

\title{MATH 0540 Problem Set 5}
\author{Collaborated with Esmé, Edward, and Mariana}
\date{October 2022}

\begin{document}
\maketitle

\section{Suppose $U$ and $W$ are both five-dimensional subspaces of $\mathbb{R}^9$. Prove that $U \cap W \not = 0$.}

\paragraph{\large
\\Assume $U \cap W = 0$:
\\ We have that $dim(U) = 5$ and $dim(W) = 5$, therefore}

\begin{align*}
    dim(U + W) &= dim(U) + dim(W) - dim(U \cap W) & \text{(Axler, 2.43)}\\
    &= 5 + 5 - dim(U \cap W) & \text{(definition of U and W)}\\
    &= 5 + 5 - dim(0) & \text{(assumption)}\\
    &= 5 + 5 - 0 & \text{(definition of zero vector space)}\\
    &= 10 & \text{(addition)}
\end{align*}

\paragraph{\large
However, since $U$ and $W$ are subspaces of $\mathbb{R}^9$, we know that $U + W$ is also a subspace of $\mathbb{R}^9$ (Axler, 1.39). This means $dim(U + W) \leq dim(\mathbb{R}^9)$ (Axler, 2.38). In other words, $dim(U + W) \leq 9$.}

\paragraph{\large
Substituting the value of $dim(U + W)$ yields:
$10 \leq 9$.}

\paragraph{\large
Since $10$ is not less than or equal to $9$, we have a contradiction, therefore $U \cap W \neq 0$.}

\newpage

\section{Suppose $U_1, \ldots, U_m$ are finite-dimensional subspaces of $V$. Prove that $U_1 + \cdots + U_m$ is finite dimensional, and
$$
\dim(U_1 + \cdots + U_m) \leq \dim U_1 + \cdots + \dim U_m
$$}

\paragraph{\large
Consider $U_1,...\,,U_m$:
\\\\
There exists $B_1,...\,,B_m$ such that each $B_i$ is a basis of $U_i$.
\\\\
Each $U_i$ in $U_1+...+U_m$ can be expressed as a linear combination $\{(a_1u_1+...+a_mu_m)\;|\; a_i \in \mathbb{F}, u_i \in U_i\}$.
\\\\
$span(B_i) = U_i$ (definition of a basis)
\\ $span(B_i) = \{(a_1u_1+...+a_mu_m)\;|\; a_i \in \mathbb{F}, u_i \in U_i\}$
\\\\
$U_1+...+U_m$ can now be expressed as $\{(span(B_1)+...+span(B_m)\}$
\\\\
Now consider two arbitrary $B_j, B_k \in B_1,...\,,B_m$:
\\ Let $\{u_1,...\,,u_m\} = B_j$
\\ Let $\{v_1,...\,,v_n\} = B_k$
\\ $span(B_j) = \{(a_1u_1+...+a_mu_m)\;|\;a_i \in \mathbb{F}, u_i \in B_j\}$
\\ $span(B_k) = \{(b_1v_1+...+b_mv_n)\;|\;b_i \in \mathbb{F}, v_i \in B_k\}$
\\\\
Taking the union of both bases yields:
\\ $B_j \cup B_k = \{u_1,...\,,u_m, v_1,...\,,v_n\}$
\\ $span(B_j \cup B_k) = 
\\\{(a_1u_1+...+a_mu_m+b_1v_1+...+b_nv_n)\;|\;a_i,b_i \in \mathbb{F}, u_i \in B_j, v_i \in B_k\}$
\\\\
Thus, $span(B_j \cup B_k) = \{span(B_j) + span(B_k)\}$
\\\\
From above, we have that:
\\ $span(B_j) = U_j$ and $span(B_k) = U_k$
\\Thus, $span(B_j \cup B_k) = U_j + U_k$.
\\\\
Generalizing this, $span(B_1 \cup ... \cup B_m) = U_1+...+U_m$. Therefore, we have that $B_1 \cup ... \cup B_m$ spans $U_1+...+U_m$.}

\paragraph{\large
Since a set of vectors spans $U_1+...+U_m$, we have that it is finite dimensional (Axler, 2.10).}

\paragraph{\large
Now suppose $B$ is a basis of $U_1+...+U_m$:
\\then $|B| \leq |B_1 \cup ... \cup B_m|$ since the length of a list of linearly independent vectors must be less than or equal to the length of a spanning list (Axler, 2.23).
\\\\
It follows that $|B_1 \cup ... \cup B_m| \leq |B_1|+...+|B_m|$ because any identical vectors in any $B_i,...\,,B_{i'}$ would only contribute 1 to $|B_1 \cup ... \cup B_m|$, while $|B_1|+...+|B_m|$ would increase by the total number of duplicates. In other words, because a union is the smallest set containing all elements of two sets, and there are no repeat elements in sets, the length of the union of all bases must be less than or equal to the individual lengths summed together.
\\\\
Finally, since the length of a basis equals the dimension of the vector space it is a basis of (Axler, 2.36), we have that $|B| = dim(U_1+...+U_m)$ and $|B_i| = dim(U_i)$. 
\\\\
Thus, we can rewrite $|B| \leq |B_1 \cup ... \cup B_m| \leq |B_1|+...+|B_m|$ as $|B| \leq |B_1|+...+|B_m|$, which can be expressed as:
\\ $dim(U_1+...+U_m) \leq dim(U_1)+...+dim(U_m)$.
}

\newpage

\section{Suppose $b,c \in \mathbb{R}$. Define $T : \mathbb{R}^3 \rightarrow \mathbb{R}^2$ by
$$
T(x,y,z) = (2x - 4 y + 3z + b, 6x + cxyz).
$$
Show that $T$ is linear if and only if $b = 0$ and $c = 0$.}

\paragraph{\large
\\Assume $T$ is a linear map:
\\ Therefore, $\forall x,y,z,\lambda \in \mathbb{R}$,}

\begin{align*}
    \lambda \cdot T(x, y, z) &= T(\lambda \cdot (x, y, z)) & \text{(homogeneity)} \\
    \lambda T(x, y, z) &= T(\lambda x, \lambda y, \lambda z) & \text{(scalar multiplication)} \\
    \lambda (2x - 4y + 3z + b, 6x + cxyz) &= (2\lambda x - 4\lambda y + 3\lambda z + b, 6\lambda x + c\lambda^3 xyz)) & \text{(definition of T)}\\
    (2\lambda x - 4\lambda y + 3\lambda z + \lambda b, 6\lambda x + c\lambda xyz) &= (2\lambda x - 4\lambda y + 3\lambda z + b, 6\lambda x + c\lambda^3 xyz)) & \text{(scalar multiplication)}\\
    (\lambda b, c\lambda xyz) &= (b, c\lambda^3xyz) & \text{(additive inverse)}
\end{align*}

\paragraph{\large
Therefore, we have that $\lambda b = b$ and $c\lambda xyz = c\lambda^3xyz$. $\lambda,x,y,z$ are free and have no fixed relation to each other. The only way for these relations to hold, then, is if $b = 0$ and $c = 0$.}

\paragraph{\large
Now, from the other side, assuming $b = 0$ and $c = 0$, we must show that $T$ is a linear map by proving the following two conditions $\forall \lambda,x,y,z,l,m,n \in \mathbb{R}$:
\\\indent 1. $T(x, y, z) + T(l, m, n) = T((x, y, z) + (l, m, n))$
\\\indent 2. $\lambda T(x, y, z) = T(\lambda(x, y, z))$}

\paragraph{\large
Since $b=0$ and $c=0$, the definition of $T(x, y, z)$ can be rewritten as $T(x,y,z) = (2x-4y+3z,6x)$.}

\begin{align*}
    T(x,y,z) + T(l,m,n) &= (2x-4y+3z,6x) + (2l-4m+3n,6l) & \text{(definition of T)}\\
    &= (2x+2l-4y-4m+3z+3n, 6x+6l) & \text{(vector addition)}\\
    &= (2(x+l)-4(y+m)+3(z+n),6(x+l)) & \text{(distributive property)}\\
    &= T(x+l,y+m,z+n) & \text{(definition of T)}\\
    &= T((x,y,z)+(l,m,n)) & \text{(vector addition)}
\end{align*}

\begin{align*}
    \lambda T(x,y,z) &= \lambda (2x-4y+3z,6x) & \text{(definition of T)}\\
    &= (2\lambda x-4\lambda y+3\lambda z,6\lambda x) & \text{(scalar multiplication)}\\
    &= T(\lambda x, \lambda y, \lambda z) & \text{(definition of T)}\\
    &= T(\lambda (x,y,z)) & \text{(scalar multiplication)}
\end{align*}

\paragraph{\large
Therefore, we have that $T$ is a linear map.}

\paragraph{\large
Since $b = 0$ and $c = 0$ implies $T$ is linear, and $T$ being linear implies $b=0$ and $c=0$, we have that $T$ is linear if and only if $b = 0$ and $c = 0$.}

\newpage

\section{Find a basis for $L(\mathbb{R}^2,\mathbb{R}^2)$, the vector space of linear maps from $\mathbb{R}^2$ to $\mathbb{R}^2$.}

\paragraph{\large
\\A linear map is defined by its behavior on a basis of the vector space it consumes. In this case, $L$ is a linear map $\mathbb{R}^2 \rightarrow \mathbb{R}^2$.}

\paragraph{\large
Therefore, let's examine the standard basis of $\mathbb{R}^2$, which is $\{(0, 1), (1, 0)\}$.}

\paragraph{\large
Suppose $T \in L(\mathbb{R}^2 \rightarrow \mathbb{R}^2)$:
\\ $T(1, 0) = (a, b)$
\\ $T(0, 1) = (c, d)$}

\paragraph{\large
This leads to a general solution:}

\begin{align*}
T(x, y) &= T(x \cdot (1, 0) + y \cdot (0, 1))
\\&= T(x \cdot (1, 0)) + T(y \cdot (0, 1)) & \text{(additivity)}
\\&= x \cdot T(1, 0) + y \cdot T(0, 1) & \text{(homogeneity)}
\\&= x \cdot (a, b) + y \cdot (c, d) & \text{(definition of T on standard basis)}
\\&= (ax, bx) + (cy, dy) & \text{(scalar multiplication)}
\\&= (ax + cy, bx + dy) & \text{(vector addition)}
\end{align*}

\paragraph{\large
A basis can then be extracted from this general solution by fixing all but one coefficient to $0$ and the remaining coefficient to $1$:}

\begin{align*}
T_a(x, y) &= (x, 0)
\\T_b(x, y) &= (0, x)
\\T_c(x, y) &= (y, 0)
\\T_d(x, y) &= (0, y)
\end{align*}

\paragraph{\large
To prove $\{(x, 0), (0, x), (y, 0), (0, y)\}$ is a basis of $L(\mathbb{R}^2 \rightarrow \mathbb{R}^2)$, we must prove the following:
\\\indent 1. $\{(x, 0), (0, x), (y, 0), (0, y)\}$ is linearly independent
\\\indent 2. $\{(x, 0), (0, x), (y, 0), (0, y)\}$ spans $L(\mathbb{R}^2 \rightarrow \mathbb{R}^2)$}

\paragraph{\large
Assume
}
\begin{align*}
0 = \lambda_1
\begin{pmatrix}
    x \\ 0
\end{pmatrix}
+ \lambda_2
\begin{pmatrix}
    0 \\ x
\end{pmatrix}\
+ \lambda_3
\begin{pmatrix}
    y \\ 0
\end{pmatrix}
+ \lambda_4
\begin{pmatrix}
    0 \\ y
\end{pmatrix}
\end{align*}

\paragraph{\large
It follows that:
\\1. $0 = \lambda_1 \cdot x + \lambda_3 \cdot y$
\\2. $0 = \lambda_2 \cdot x + \lambda_4 \cdot y$}

\paragraph{\large
For equation 1, in order for $x$ and $y$ to remain free, $\lambda_1$ and $\lambda_3$ must be fixed in relation to $x$ and $y$. Since $x$ and $y$ have no fixed relation to each other, the only way $0 = \lambda_1 \cdot x + \lambda_3 \cdot y$ is if $\lambda_1$ and $\lambda_3$ equal $0$.}

\paragraph{\large
For equation 2, in order for $x$ and $y$ to remain free, $\lambda_2$ and $\lambda_4$ must be fixed in relation to $x$ and $y$. Since $x$ and $y$ have no fixed relation to each other, the only way $0 = \lambda_2 \cdot x + \lambda_4 \cdot y$ is if $\lambda_2$ and $\lambda_4$ equal $0$.}

\paragraph{\large
Therefore, we have that $\{(x, 0), (0, x), (y, 0), (0, y)\}$ is linearly independent.}

\paragraph{\large
In order for $\{(x, 0), (0, x), (y, 0), (0, y)\}$ to span $L(\mathbb{R}^2 \rightarrow \mathbb{R}^2)$, we must show that there exists some scalars $k_1,k_2,k_3,k_4 \in \mathbb{R}$ which can produce any element of the general solution $T(x,y) = (ax + cy, bx + dy)$.}

\begin{align*}
\begin{pmatrix}
    ax+cy \\ bx+dy
\end{pmatrix}
= k_1
\begin{pmatrix}
    x \\ 0
\end{pmatrix}
+ k_2
\begin{pmatrix}
    0 \\ x
\end{pmatrix}
+ k_3
\begin{pmatrix}
    y \\ 0
\end{pmatrix}
+ k_4
\begin{pmatrix}
    0 \\ y
\end{pmatrix}
\end{align*}

\paragraph{\large
It follows that:
\\\indent 1. $ax+cy = k_1 x + k_3 y$
\\\indent 2. $bx+dy = k_2 x + k_4 y$}

\paragraph{\large
This leads to the following four relations:
\\\indent 1. $a = k_1$
\\\indent 2. $c = k_3$
\\\indent 3. $b = k_2$
\\\indent 4. $d = k_4$}

\paragraph{\large
Therefore, we have that $\{(x, 0), (0, x), (y, 0), (0, y)\}$ spans 
\\$L(\mathbb{R}^2 \rightarrow \mathbb{R}^2)$.}

\paragraph{\large
Since both conditions are met, we have proven that $\{(x, 0), (0, x), (y, 0), (0, y)\}$ is a basis of $L(\mathbb{R}^2 \rightarrow \mathbb{R}^2)$.}

\newpage

\section{Is the product of linear maps commutative? Meaning, given linear maps $S,T \in L(V,V)$, is it true that $ST = TS$? Prove or disprove.}

\paragraph{\large
\setlength{\parindent}{0pt}
\\Define $S,T \in L(\mathbb{R}^2, \mathbb{R}^2)$ by 
\\$T(x, y) = (x, x)$
\\$S(x, y) = (x-y, y-x)$
}

\paragraph{\large
A map F is linear if:
\\\indent 1. $F(x, y) + F(a, b) = F((x, y) + (a, b))$
\\\indent 2. $\lambda F(x, y) = F(\lambda (x, y))$}

\paragraph{\large
We will first show that $S$ and $T$ are linear maps by proving these two conditions.}

\paragraph{\large
Assuming the first condition is true for $T$:}
\begin{align*}
    T(x, y) + T(a, b) &= T((x, y) + (a, b)) & \text{(assumption)} \\
    T(x, y) + T(a, b) &= T(x+a, y+b) & \text{(vector addition)}\\
    (x, x) + (a, a) &= (x+a, x+a) & \text{(definition of T)}\\
    (x+a, x+a) &= (x+a, x+a) & \text{(vector addition)}
\end{align*}

\paragraph{\large
Assuming the second condition is true for $T$:}

\begin{align*}
    \lambda T(x, y) &= T(\lambda (x, y)) & \text{(assumption)}\\
    \lambda T(x, y) &= T(\lambda x, \lambda y) & \text{(scalar multiplication)}\\
    \lambda (x, x) &= (\lambda x, \lambda x) & \text{(definition of T)}\\
    (\lambda x, \lambda x) &= (\lambda x, \lambda x) & \text{(scalar multiplication)}
\end{align*}

\paragraph{\large
Assuming the first condition is true for $S$:}

\begin{align*}
    S(x, y) + S(a, b) &= S((x, y) + (a, b)) & \text{(assumption)}\\
    S(x, y) + S(a, b) &= S(x+a, y+b) & \text{(vector addition)}\\
    (x-y, y-x) + (a-b, b-a) &= ((x+a)-(y+b), (y+b)-(x+a)) & \text{(definition of S)}\\
    (x-y, y-x) + (a-b, b-a) &= (x+a-y-b, y+b-x-a) & \text{(distributive property)}\\
    (x+a-y-b, y+b-x-a) &= (x+a-y-b, y+b-x-a) & \text{(vector addition)}\\
\end{align*}

\paragraph{\large
Assuming the second condition is true for $S$:}

\begin{align*}
    \lambda S(x, y) &= S(\lambda (x, y)) & \text{(assumption)}\\
    \lambda S(x, y) &= S(\lambda x, \lambda y) & \text{(scalar multiplication)}\\
    \lambda (x-y, y-x) &= (\lambda x-\lambda y, \lambda y-\lambda x) & \text{(definition of S)}\\
    (\lambda (x-y), \lambda (y-x)) &= (\lambda x-\lambda y, \lambda y-\lambda x) & \text{(scalar multiplication)}\\
    (\lambda x-\lambda y, \lambda y-\lambda x) &= (\lambda x-\lambda y, \lambda y-\lambda x) & \text{(distributive property)}
\end{align*}

\paragraph{\large
Therefore, we have that $S$ and $T$ both satisfy additivity and homogeneity, proving they are linear maps.}

\paragraph{\large
Assume $ST = TS$:}

\begin{align*}
    S(T(x, y)) &= T(S(x, y)) & \text{(assumption)}\\
    S(x, x) &= T(x-y,y-x) & \text{(definition of S and T)}\\
    (x-x, x-x) &= (x-y, x-y) & \text{(definition of S and T)}\\
    (0, 0) &= (x-y, x-y) & \text{(additive inverse)}\\
    0 &= (x-y, x-y) & \text{(definition of zero vector)}
\end{align*}

\paragraph{\large
Since $x$ and $y$ are free and can be any value in $\mathbb{R}$, we have a contradiction. Therefore, it is disproven that the product of linear maps is commutative.}

\end{document}
