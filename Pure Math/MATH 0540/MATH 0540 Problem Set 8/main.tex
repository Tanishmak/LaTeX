\documentclass[12pt,reqno]{article}

%%%%%%%%%%%%%%%%%%%% PACKAGES %%%%%%%%%%%%%%%%%%%%

\usepackage[utf8]{inputenc}
\usepackage[all]{xy}
\usepackage[T1]{fontenc}
\usepackage[usenames, dvipsnames]{color}
\usepackage{setspace}
\usepackage{dsfont}
\usepackage{amssymb}
\usepackage{amsthm,bbm}
\usepackage{amscd}
\usepackage{amsfonts}
\usepackage{stmaryrd}
\usepackage{amsmath}
\usepackage{graphicx}
\usepackage{multicol}
\usepackage{xspace}
\usepackage{extarrows}
\usepackage{color}
\usepackage [english]{babel}
\usepackage [autostyle, english = american]{csquotes}
\usepackage{euler}

%%%%%%%%%%%%%%%%%%%% INITIALIZATION %%%%%%%%%%%%%%%%%%%%

\MakeOuterQuote{"}
\graphicspath{ {./} }
\onehalfspacing

%%%%%%%%%%%%%%%%%%%% ENVIRONMENTS %%%%%%%%%%%%%%%%%%%%
\theoremstyle{definition}
\newtheorem{problem}{Problem}

%%%%%%%%%%%%%%%%%%%% TITLE-PAGE %%%%%%%%%%%%%%%%%%%%

\title{MATH 0540 Problem Set 8}
\author{Collaborated with Esmé}
\date{November 2022}

%%%%%%%%%%%%%%%%%%%% DOCUMENT %%%%%%%%%%%%%%%%%%%%

\begin{document}
\maketitle

%%%%%%%%%%%%%%%%%%%% PROBLEM 1 %%%%%%%%%%%%%%%%%%%%

\begin{problem}
    Give an example of a vector space $V$ and a linear map $T:V\rightarrow V$ such that $T$ is not the identity, and $T$ is its own inverse (and prove this).
\end{problem}

\begin{proof}
    Define $T \in L(\mathbb{R}^2)$ such that $T\neq I$:
    \begin{align*}
        T(x, y) = (y, x)
    \end{align*}
    Now, we will prove that $T$ is its own inverse by showing that $T\cdot T = I$.
    \begin{align*}
        T\cdot T &= T(T(x, y)) &\text{(composition of linear maps)}\\
        &= T(y, x) &\text{(definition of $T$)}\\
        &= (x, y) &\text{(definition of $T$)}\\
        &= I &\text{(definition of $I$)}
    \end{align*}
    Therefore, we have proven that the linear map $T: \mathbb{R}^2 \rightarrow \mathbb{R}^2$, which is not the identity, is its own inverse. 
\end{proof}

\newpage

%%%%%%%%%%%%%%%%%%%% PROBLEM 2 %%%%%%%%%%%%%%%%%%%%

\begin{problem}
    Let $T:\mathbb{R}^2 \rightarrow \mathbb{R}^2$ be the linear map defined by
    $$
    T(x,y) = (ax + by, bx + ay)\ \ \ \ \text{    for }a,b \in \mathbb{R}.
    $$
    Give a condition on $a,b$ for $T$ to be invertible, and when $T$ is invertible, find the inverse.
\end{problem}

\begin{proof}
    Assuming $T$ has an inverse, define $T^{-1}:\mathbb{R}^2 \rightarrow \mathbb{R}^2$:
    \begin{align*}
            T^{-1}(x, y) &= (cx + dy, ex + fy) &\text{(general form of $L(\mathbb{R}^2)$)}
    \end{align*}
    In order for $T^{-1}$ to invert $T$, it must consume the output of $T$, which is of the form $(ax + by, bx + ay)$, and produce the input of $T$, which is $(x, y)$. In other words, $T^{-1}\cdot T = I$:
    \begin{align*}
        T^{-1}\cdot T &= I & \text{(definition of inverse)}\\
        T^{-1}(T(x, y)) &= (x, y) & \text{(definition of $I$)}\\
        T^{-1}(ax + by, bx + ay) &= (x, y) & \text{(definition of $T$)}\\
        (c(ax+by)+d(bx+ay),\ e(ax+by)+f(bx+ay)) &= (x, y) & \text{(definition of $T^{-1}$)}\\
        (cax+cby+dbx+day,\ eax+eby+fbx+fay) &= (x, y) & \text{(distributive property)}
    \end{align*}
    \begin{align*}
        x &= cax+cby+dbx+day\quad & y &= eax+eby+fbx+fay & \text{(equate components)}\\
        x &= (ca+db)x+(cb+da)y\quad & y &= (ea+fb)x+(eb+fa)y & \text{(distributive prop.)}
    \end{align*}
    For these relationships to hold, the following must be true:
    \begin{align*}
        &\text{1.}\ ca+db=1
        &\text{3.}\ ea+fb=0\\
        &\text{2.}\ cb+da=0
        &\text{4.}\ eb+fa=1
    \end{align*}
    Using these four equations, we will solve for $c,d,e,f$ to determine $T^{-1}$ in terms of $a,b$. \\\\
    Solve for $d$:
    \begin{align*}
        ca+db&=1 & \text{(equation 1)}\\
        cab+db^2&=b &\text{(mult. by b)}\\
        cba+db^2&=b &\text{(commutativity)}\\
        (-da)a+db^2&=b &\text{($cb=-da$ by eq. 2)}\\
        -da^2+db^2&=b &\text{(multiplication)}\\
        db^2-da^2&=b &\text{(commutativity)}\\
        d(b^2-a^2)&=b &\text{(distributive prop.)}\\
        d&=\frac{b}{b^2-a^2} &\text{(division)}
    \end{align*}
    Solve for $c$:
    \begin{align*}
        ca+db&=1 & \text{(equation 1)}\\
        ca^2+dba&=a & \text{(mult. by a)}\\
        ca^2+dab&=a & \text{(commutativity)}\\
        ca^2+(-cb)b&=a & \text{($da=-cb$ from eq. 2)}\\
        ca^2-cb^2&=a & \text{(multiplication)}\\
        c(a^2-b^2)&=a & \text{(distributive prop.)}\\
        c&=\frac{a}{a^2-b^2} & \text{(division)}
    \end{align*}
    Solve for $e$:
    \begin{align*}
        eb+fa&=1 & \text{(equation 4)}\\
        eb^2+fab&=b & \text{(mult. by b)}\\
        eb^2+fba&=b & \text{(commutativity)}\\
        eb^2+(-ea)a&=b & \text{($fb=-ea$ from eq. 3)}\\
        eb^2-ea^2&=b & \text{(multiplication)}\\
        e(b^2-a^2)&=b & \text{(distributive prop.)}\\
        e&=\frac{b}{b^2-a^2} & \text{(distributive prop.)}
    \end{align*}
    Solve for $f$:
    \begin{align*}
        eb+fa&=1 &\text{(equation 4)}\\
        eba+fa^2&=a &\text{(mult. by a)}\\
        eab+fa^2&=a &\text{(commutativity)}\\
        (-fb)b+fa^2&=a &\text{($ea=-fb$ from eq. 3)}\\
        -fb^2+fa^2&=a &\text{(multiplication)}\\
        fa^2-fb^2&=a &\text{(commutativity)}\\
        f(a^2-b^2)&=a &\text{(distributive prop.)}\\
        f&=\frac{a}{a^2-b^2} &\text{(distributive prop.)}
    \end{align*}
    From the denominators of the solutions for $c,d,e,f$, it follows that:
    \begin{align*}
        b^2-a^2&\neq0 & a^2-b^2&\neq0 & \text{(div. by $0$ undefined)}\\
        (b+a)(b-a)&\neq0 & (a+b)(a-b)&\neq0 & \text{(dif. of two squares)}\\
        a&\neq b & a&\neq-b & \text{(zero product prop.)}
    \end{align*}
    Therefore, we have defined a solution for $T^{-1}: \mathbb{R}^2 \rightarrow \mathbb{R}^2$:
    \begin{align*}
        &T^{-1}(x, y) = (\frac{a}{a^2-b^2}x + \frac{b}{b^2-a^2}y,\ \frac{b}{b^2-a^2}x + \frac{a}{a^2-b^2}y)\\
        &\text{where}\ a\neq b\ \text{and}\ a\neq-b
    \end{align*}
    We already have that $T^{-1}\cdot T=I$ since $T^{-1}$ was constructed by assuming the relation. Now we will prove that $T\cdot T^{-1}=I$ as well in order to show that $T^{-1}$ is the inverse of $T$.
    \begingroup
    \allowdisplaybreaks
    \begin{align*}
        T\cdot T^{-1} &= T(T^{-1}(x, y)) & \text{(composition of linear maps)}\\
         &= T(\frac{a}{a^2-b^2}x + \frac{b}{b^2-a^2}y,\ \frac{b}{b^2-a^2}x + \frac{a}{a^2-b^2}y) & \text{(definition of $T^{-1}$)}\\
         &= (a(\frac{a}{a^2-b^2}x + \frac{b}{b^2-a^2}y) + b(\frac{b}{b^2-a^2}x + \frac{a}{a^2-b^2}y),& \text{(definition of $T$)}\\ &\quad\ b(\frac{a}{a^2-b^2}x + \frac{b}{b^2-a^2}y) + a(\frac{b}{b^2-a^2}x + \frac{a}{a^2-b^2}y))\\
         &= (\frac{a^2x}{a^2-b^2} + \frac{aby}{b^2-a^2} + \frac{b^2x}{b^2-a^2} + \frac{aby}{a^2-b^2},& \text{(distributive prop.)}\\ &\quad\ \frac{abx}{a^2-b^2} + \frac{b^2y}{b^2-a^2} + \frac{abx}{b^2-a^2} + \frac{a^2y}{a^2-b^2})\\
         &= (\frac{a^2x}{a^2-b^2} + \frac{b^2x}{b^2-a^2} + \frac{aby}{b^2-a^2} + \frac{aby}{a^2-b^2},& \text{(commutativity)}\\ &\quad\ \frac{abx}{a^2-b^2} + \frac{abx}{b^2-a^2} + \frac{b^2y}{b^2-a^2} + \frac{a^2y}{a^2-b^2})\\
         &= (\frac{a^2x}{a^2-b^2} - \frac{b^2x}{a^2-b^2} + \frac{aby}{b^2-a^2} - \frac{aby}{b^2-a^2},& \text{(distributive prop.)}\\ &\quad\ \frac{abx}{a^2-b^2} - \frac{abx}{a^2-b^2} + \frac{b^2y}{b^2-a^2} - \frac{a^2y}{b^2-a^2})\\
         &= (\frac{a^2x-b^2x}{a^2-b^2} + 0,\ 0 + \frac{b^2y-a^2y}{b^2-a^2})& \text{(subtraction)}\\
         &= (\frac{a^2x-b^2x}{a^2-b^2},\ \frac{b^2y-a^2y}{b^2-a^2})& \text{(additive identity)}\\
         &= (\frac{x(a^2-b^2)}{a^2-b^2},\ \frac{y(b^2-a^2)}{b^2-a^2})& \text{(distributive prop.)}\\
         &= (x,\ y)& \text{(division)}\\
         &= I &\text{(definition of $I$)}
    \end{align*}
    Therefore, we have that $T^{-1}$ is the inverse of $T$ for all $a,b\in\mathbb{R}$ such that $a\neq b$ and $a\neq -b$.
    \endgroup
\end{proof}

\newpage

%%%%%%%%%%%%%%%%%%%% PROBLEM 3 %%%%%%%%%%%%%%%%%%%%

\begin{problem}
    Let $T$ be an invertible linear map from $U$ to $V$ and let $S$ be an invertible linear map from $V$ to $W$. Prove that $ST: U\rightarrow W$ is invertible and that 
    $$
    (ST)^{-1} = T^{-1} S^{-1}.
    $$
\end{problem}
\begin{proof}
    Since $S$ and $T$ are invertible, we have that:
    \begin{align*} 
        &\dim U = \dim V = \dim W & \text{(Axler, 3.59)}
    \end{align*}
    Additionally, we have that $S$ and $T$ are both injective and surjective by the definition of invertibility. To prove that $ST$ is invertible, we will first show that $ST$ is both injective and surjective.
    \\\\
    If $ST$ is injective, $(ST)(u) = (ST)(v)$ implies $u=v$:
    \begin{align*}
        (ST)(u)&=(ST)(v) &\text{(assumption)}\\
        S(T(u))&=S(T(v)) &\text{(composition of linear maps)}\\
        S^{-1}(S(T(u))&= S^{-1}(S(T(v))) &\text{(apply $S^{-1}$ to both sides)}\\
        T(u)&=T(v) &\text{(definition of inverse)}\\
        u&=v &\text{($T$ is injective)}
    \end{align*}
    Therefore, we have proven that $ST$ is injective. Next, to prove $ST$ is surjective, we will prove that $\mathrm{range}\,ST = W$:
    \begin{align*}
        \mathrm{range}\,ST &= \{(ST)(u)\ |\ u\in U\} &\text{(definition of range)}\\
        &= \{S(T(u))\ |\ u\in U\} &\text{(composition of linear maps)}\\
        &= \{S(v)\ |\ v\in V\} &\text{($T$ is surjective)}\\
        &= \{w\ |\ w\in W\} &\text{($S$ is surjective)}\\
        &= W &\text{(definition of vector space)}\\
    \end{align*}
    Because we have proven that $ST$ is both injective and surjective, we have that $ST$ is invertible.
    \\\\
    Next, we will prove that $(ST)^{-1} = T^{-1} S^{-1}$.
    \begin{align*}
        (ST)^{-1} &= T^{-1} S^{-1} &\text{(assumption)}\\
        (ST)(ST)^{-1} &= (ST)T^{-1} S^{-1} &\text{(multiply by $ST$)}\\
        ST((ST)^{-1}(w)) &= ST(T^{-1}(S^{-1}(w))) &\text{(composition of linear maps)}\\
        ST((ST)^{-1}(w)) &= S(T(T^{-1}(S^{-1}(w)))) &\text{(composition of linear maps)}\\
        w&=w & \text{(definition of inverse)}
    \end{align*}
    Therefore, we have that $(ST)^{-1} = T^{-1} S^{-1}$.
\end{proof}

\newpage

%%%%%%%%%%%%%%%%%%%% PROBLEM 4 %%%%%%%%%%%%%%%%%%%%

\begin{problem}
    Let $T: V \rightarrow W$, and let $v_1, \ldots, v_n$ be a basis for $V$ and $w_1,\ldots, w_m$ be a basis for $W$. Prove that $T$ is invertible if and only if the columns of $M(T)$ (with respect to the bases $v_1,\ldots, v_n$ and $w_1,\ldots, w_m$) are a basis for $W$. 
\end{problem}

\begin{proof}
    $\Rightarrow$ Assume $T$ is invertible. It follows that $T$ is both injective and surjective by the definition of invertibility. Because $T$ is surjective, we also have that $\dim V = \dim W$. Thus, the basis of $V$ can be rewritten as $v_1,\ldots,v_m$, and $M(T)$ will be a square matrix:
    \begin{align*}
        \begin{bmatrix}
            M(T)_{1,1} & \cdots & M(T)_{1, m}\\
            \vdots & & \vdots\\
            M(T)_{m, 1} & \cdots & M(T)_{m, m}
        \end{bmatrix}
    \end{align*}
    Consider each $v_k \in v_1,\ldots,v_m$.
    \begin{align*}
        T(v_k) &= M(T)_{1, k}w_1 + \cdots + M(T)_{m, k}w_m &\text{(Axler, 3.32)}
    \end{align*}
    Together, $M(T)_{1, k},\ldots,M(T)_{m, k}$ represent a column $M(T)_{\cdot,k}$ of $M(T)$:
    \begin{align*}
        \begin{bmatrix}
            M(T)_{1,k}\\
            \vdots\\
            M(T)_{m,k}
        \end{bmatrix}
    \end{align*}
    Let $v_j, v_k \in v_1,\ldots,v_m$. Since a basis is linearly independent, we know that $v_j \notin \mathrm{span}(v_k)$. Assume one column of $M(T)$ is a scalar multiple of another:
    \begin{align*}
        \begin{bmatrix}
            M(T)_{1,j}\\
            \vdots\\
            M(T)_{m,j}
        \end{bmatrix}
        &= \lambda
        \begin{bmatrix}
            M(T)_{1,k}\\
            \vdots\\
            M(T)_{m,k}
        \end{bmatrix}
        & \text{(assumption)}\\
        M(T)_{1,j}w_1 +\cdots+M(T)_{m,j}w_m &= \lambda(M(T)_{1,k}w_1+\cdots+M(T)_{m,k}w_m) & \text{(definition of matrix)}\\
        T(v_j) &= \lambda T(v_k) & \text{(Axler, 3.32)}\\
        T(v_j) &= T(\lambda v_k) & \text{(homogeneity)}\\
        v_j &= \lambda v_k & \text{(equate inputs)}
    \end{align*}
    Thus, we have reached a contradiction. If one column of $M(T)$ is a scalar multiple of another, it implies that the basis of $V$ is linearly dependent, which cannot be true. Therefore, we know that the columns of $M(T)$ must be linearly independent.
    \\\\
    Let $C$ be the set of all columns of $M(T)$. We have:
    \begin{align*}
        &1.\ \mathrm{cardinality}(C) = m = \dim W\\
        &2.\ C\ \text{is linearly independent}
    \end{align*}
    Therefore, we have proven that $C$ is a basis of $W$.
    \\\\
    $\Leftarrow$ Assume the columns of $M(T)$ are a basis for $w$. By the definition of a matrix, we know that the number of columns in $M(T)$ equals $\dim V$. Since there are $m$ columns, $\dim V = m = \dim W$. Therefore, the relation $T(v) = M(T) \cdot v$ can be rewritten as:
    \begin{align*}
        T(v) =
        \begin{bmatrix}
            | & & |\\
            w_1 & \cdots & w_m\\
            | & & |
        \end{bmatrix}
        \cdot
        \begin{bmatrix}
             a_1 \\
            \vdots \\
            a_m
        \end{bmatrix}
        &= a_1w_1 + \cdots + a_mw_m
        & \text{(Brandt, Hours)}
    \end{align*}
    It follows that:
    \begin{align*}
        \mathrm{range}\,T &= \{T(v)\ |\ v\in V\} & \text{(definition of range)}\\
        &= \{a_1w_1+\cdots+a_mw_m\ |\ a_i\in \mathbb{F}\} & \text{(from above)}\\
        &= span(W) & \text{(definition of span)}\\
        &= W & \text{($W$ closed)}
    \end{align*}
    Therefore, we have proven that $T$ is surjective. Using this, we can invoke the rank-nullity theorem to prove $T$ is also injective:
    \begin{align*}
        \dim \mathrm{null}\,T &= \dim V - \dim \mathrm{range}\,T &\text{(rank-nullity theorem)}\\
        &= \dim V - \dim W &\text{($T$ is surjective)}\\
        &= 0 &\text{($\dim V = \dim W$)}
    \end{align*}
    Finally, we have shown that $T$ is also injective. Therefore, since we have that $T$ is both injective and surjective, we have proven that $T$ is invertible.
\end{proof}

\newpage

%%%%%%%%%%%%%%%%%%%% PROBLEM 5 %%%%%%%%%%%%%%%%%%%%

\begin{problem}
    Let $X = \{1,2,\ldots,n\}$. Let $V$ be the set of all subsets of $X$. For $A,B \subset X$, define 
    $$
    A + B = (A \cup B) \backslash (A \cap B).
    $$
    We define a scalar multiplication on $V$, with scalars $\mathbb{F}_2 = \{0,1\}$, by defining
    $$
    0 \cdot A = \emptyset,\ \ \ \ 1 \cdot A = A.
    $$
    On Problem Set 2 you showed that this was a vector space over $\mathbb{F}_2$, and on Problem Set 4 you found a basis for this vector space. 
    
    By a result from lecture, $V$ is isomorphic to $\mathbb{F}_2^k$ for some $k$. What is $k$, and what is an isomorphism $V \rightarrow \mathbb{F}_2^k$?
\end{problem}
\begin{proof}
    We have that the basis of V is $\{1\},\ldots,\{n\}$, and that $\dim V = n$. For $V$ to be isomorphic to $\mathbb{F}^k_2$ for some $k$, $\dim V$ must equal $\dim \mathbb{F}^k_n$. Therefore, $k$ must equal $n$.
    \\\\
    For all $u\in V$, define $T: V \rightarrow F^n_2$:
    \begin{align*}
        T(u)_i&=
        \begin{cases}
            1\ \text{if}\ i\in u\\
            0\ \text{if}\ i\notin u
        \end{cases}
    \end{align*}
    Now, we will prove that $T$ is an isomorphism $V \rightarrow \mathbb{F}^n_2$ by showing that it is both injective and surjective. First we will prove injectivity by showing that $T(u)=T(v)$ implies $u=v$.
    \begin{align*}
        T(u)&=T(v) &\text{(assumption)}\\
        T(u)_i&=T(v)_i\quad\forall i\in 1,\ldots,n &\text{(equate components)}
    \end{align*}
    We will now split this into two cases for all $i\in 1,\ldots,n$ based on whether $T(u)_i$ is $1$ or $0$:
    \begin{align*}
        &1 = T(u)_i = T(v)_i & &0 = T(u)_i = T(v)_i &\text{(cases)}\\
        &i\in u\text{ and }i\in v & &i\notin u\text{ and }i\notin v &\text{(definition of $T$)}
    \end{align*}
    Since each component of $T(u)$ equals the fcorresponding component in $T(v)$, we have that $u$ and $v$ are composed of the same elements. Therefore, $u=v$, proving that $T$ is injective.
    \\\\
    Next, we will show that $T$ is surjective by proving that $\mathrm{range}\,T = \dim\mathbb{F}^n_2$.
    \begin{align*}
        \dim\mathrm{range}\,T &= \dim V - \dim\mathrm{null}\, T &\text{(rank-nullity theorem)}\\
        &= n - \dim\mathrm{null}\,T &\text{($\dim V=n$)}\\
        &= n - 0 &\text{($T$ is injective)}\\
        &= n &\text{(additive identity)}\\
        &= \dim\mathbb{F}^n_2 &\text{($\dim\mathbb{F}^n_2=n$)}\\
        \mathrm{range}\,T&=\mathbb{F}^n_2 &\text{(lemma from Problem Set 6, P5)}
    \end{align*}
    Therefore, we have that $T$ is surjective. Since $T$ is both injective and surjective, we have proven that $T$ is an isomorphism $V \rightarrow \mathbb{F}^n_2$.
\end{proof}
\end{document}