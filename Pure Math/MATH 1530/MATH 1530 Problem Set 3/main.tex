\documentclass[12pt,reqno]{article}

%%%%%%%%%%%%%%%%%%%% PACKAGES %%%%%%%%%%%%%%%%%%%%

\usepackage[utf8]{inputenc}
\usepackage[all]{xy}
\usepackage[T1]{fontenc}
\usepackage[usenames, dvipsnames]{color}
\usepackage{setspace}
\usepackage{dsfont}
\usepackage{amssymb}
\usepackage{amsthm,bbm}
\usepackage{amscd}
\usepackage{amsfonts}
\usepackage{stmaryrd}
\usepackage{amsmath}
\usepackage{graphicx}
\usepackage{multicol}
\usepackage{xspace}
\usepackage{extarrows}
\usepackage{color}
\usepackage [english]{babel}
\usepackage [autostyle, english = american]{csquotes}
\usepackage[colorlinks, linktocpage, citecolor = red, linkcolor = blue]{hyperref}
\usepackage{fullpage}
\usepackage{color}
\usepackage{euler}
\usepackage{parskip}

%%%%%%%%%%%%%%%%%%%% INITIALIZATION %%%%%%%%%%%%%%%%%%%%

\MakeOuterQuote{"}
\graphicspath{ {./} }

%%%%%%%%%%%%%%%%%%%% COMMANDS %%%%%%%%%%%%%%%%%%%%

\newcommand{\range}{\mathrm{range\,}}
\newcommand{\nul}{\mathrm{null\,}}
\newcommand{\spn}{\mathrm{span\,}}
\newcommand{\card}{\mathrm{cardinality}}
\newcommand{\R}{\mathbb{R}}
\newcommand{\C}{\mathbb{C}}
\newcommand{\F}{\mathbb{F}}
\newcommand{\Z}{\mathbb{Z}}
\newcommand{\bd}{\mathrm{bd\,}}
\newcommand{\divline}{\hrule\vspace{12pt}\noindent}
\newcommand{\sgn}{\mathrm{sgn}}

%%%%%%%%%%%%%%%%%%%% ENVIRONMENTS %%%%%%%%%%%%%%%%%%%%

\theoremstyle{plain}
\newtheorem{maintheorem}{Theorem}
\renewcommand*{\themaintheorem}{\Alph{maintheorem}}

\newtheorem{theorem}{Theorem}[section] 
\newtheorem{lemma}{Lemma}
\newtheorem{corollary}[theorem]{Corollary}

\theoremstyle{definition}
\newtheorem{problem}{Problem}
\newtheorem{example}[theorem]{Example}
\newtheorem{definition}[theorem]{Definition}
\newtheorem{question}[theorem]{Question}

\newtheorem*{maintheorema}{Theorem \ref{thm:main}}

%%%%%%%%%%%%%%%%%%%% TITLE-PAGE %%%%%%%%%%%%%%%%%%%%

\title{MATH 1530 Problem Set 3}
\author{Tanish Makadia\\\small{(Collaborated with Esmé and Kazuya)}}
\date{February 2023}

%%%%%%%%%%%%%%%%%%%% DOCUMENT %%%%%%%%%%%%%%%%%%%%

\begin{document}
\maketitle

%%%%%%%%%%%%%%%%%%%% PROBLEM 1 %%%%%%%%%%%%%%%%%%%%

\begin{problem} 
    Consider $U(40)$. Find a subgroup which is cyclic of order 4. Find a subgroup which is noncyclic of order 4. 
\end{problem}

\begin{proof}
    \(U(40)\) is defined as follows:
    \begin{align*}
        U(40) = \{1,3,7,9,11,13,17,19,21,23,27,29,31,33,37,39\}
    \end{align*}

    \begin{itemize}
        \item \textbf{Cyclic subgroup:} We will create a cyclic subgroup of order \(4\) using \(7\) as the generator.
        \begin{align*}
            \langle 7\rangle &= \{7,9,23,1\} & \text{(Gallian, 3.4)}
        \end{align*}
        \item \textbf{Non-cyclic subgroup:} We will show that \(\{1,9,11,19\}\) is a noncyclic subgroup.

        Cayley Table:
        \begin{tabular}{c | c c c c}
            & 1 & 9 & 11 & 19\\
            \cline{1-5}
            1 & 1 & 9 & 11 & 19 \\
            9 & 9 & 1 & 19 & 11 \\
            11 & 11 & 19 & 1 & 9 \\
            19 & 19 & 11 & 9 & 1
        \end{tabular}
    
        Since \(\{1,9,11,19\}\) is finite and closed under multiplication mod \(40\), we have proven
        that it is subgroup of \(U(40)\). Additionally, since \(|9|=|11|=|19|=2\), there is no element
        in this group which can generate the entire group, proving that it is noncyclic.
    
    \end{itemize}
\end{proof}

\newpage

%%%%%%%%%%%%%%%%%%%% PROBLEM 2 %%%%%%%%%%%%%%%%%%%%

\begin{problem}
    If $H$ and $K$ are subgroups of a group $G$, prove that $H \cap K$ is a subgroup of $G$. If $H \not \subset K$ and $K \not \subset H$, prove that $H \cup K$ is never a subgroup of $G$.
\end{problem}

\bigskip
1. \(H\cap K\) is a subgroup of \(G\).

\begin{proof}
    We will use the two-step subgroup test.
    \begin{itemize}
        \item \textbf{Closed over inverses:} \(x\in H\cap K \implies x\in H\) and
        \(x\in K\). Since \(H\) and \(K\) are subgroups, we have the existence of \(x^{-1}\in H\) and 
        \(x^{-1}\in K \implies x^{-1}\in H\cap K\). 

        \item \textbf{Closed under group operation:} \(a,b\in H\cap K\implies a,b\in H\) and \(a,b\in K\).
        Since \(H\) and \(K\) are subgroups, we have that \(ab\in H\) and \(ab\in K\implies ab\in H\cap K\).
    \end{itemize}
\end{proof}

2. If $H \not \subset K$ and $K \not \subset H$, $H \cup K$ is never a subgroup of $G$.
\begin{proof}
    We will show that \(H\cup K\) is not closed under the group operation. 
    Since neither \(H\) nor \(K\) are subsets of each other, we have the existence of some 
    \(a,a^{-1}\in H\setminus K\) and \(b,b^{-1}\in K\setminus H\).

    If \(H\cup K\) is a subgroup, it must be closed. This implies that 
    \(ab\in H\cup K \implies ab\in H\) or \(ab\in K\). Thus, we have two cases:

    \begin{itemize}
        \item Case 1 (\(ab\in H\)): Since \(H\) is closed, \(a^{-1}\cdot ab \in H\implies b\in H\). This is a contradiction.

        \item Case 2 (\(ab\in K\)): Since \(K\) is closed, \(ab\cdot b^{-1}\in K\implies a\in K\). This is a contradiction.
    \end{itemize}

    Because both cases lead to a contradiction, we have proven that \(H\cup K\) is not closed
    and therefore is not a subgroup.
\end{proof}

\newpage

%%%%%%%%%%%%%%%%%%%% PROBLEM 3 %%%%%%%%%%%%%%%%%%%%

\begin{problem} 
    Prove that a group $G$ is Abelian if and only if $G = Z(G)$.
\end{problem}

\begin{proof}
    \(\Rightarrow\) Assume \(G\) is an Abelian group. Of course, \(Z(G)\subset G\) by definition. Since 
    \(G\) is Abelian, \(a\in G\implies ax=xa\) for all \(x\in G\implies a\in Z(G)\). Therefore, \(G \subset Z(G)\), completing the double inclusion.

    \(\Leftarrow\) Assume \(G = Z(G)\). Thus, \(a\in G\implies a\in Z(G)\implies ax=xa\) for all \(x\in G\). 
    Therefore, \(G\) must be Abelian.
\end{proof}

\newpage

%%%%%%%%%%%%%%%%%%%% PROBLEM 4 %%%%%%%%%%%%%%%%%%%%

\begin{problem}
    Suppose $G$ is a group with exactly 8 elements of order $3$. how many subgroups of order $3$ does $G$ have?
\end{problem}

\begin{proof}
    Let \(H\subset G\) be a subgroup of order 3:
    \begin{align*}
        H = \{e, a, b\}
    \end{align*}
    Since \(e\) is unique, we have that \(ab\neq a\) and \(ab\neq b\). In order for \(H\) to be closed, 
    the only remaining choice is \(ab = e\). Thus, for any subgroup of order \(3\), the two elements besides
    the identity must be each other's inverse.

    Now, we will show that \(|a|=|b|=3\). Consider \(a^2\in H\). Since identities and inverses are unique,
    \(a^2\neq a\) and \(a^2\neq e\). The only remaining choice is \(a^2=b\). Therefore,
    \begin{equation*}
        \begin{aligned}[c]
            a^3 &= a \cdot a^2 \\
            &= a \cdot b\\
            &= e
        \end{aligned}
        \qquad\qquad
        \begin{aligned}[c]
            b^3 &= (a^2)^3\\
            &= (a^3)^2\\
            &= e
        \end{aligned}
    \end{equation*}
    We have that there are exactly \(8\) elements of \(G\) of order \(3\). Because a subgroup of order \(3\) 
    requires two distinct elements of order \(3\), we can conclude that the number of distinct subgroups of 
    order \(3\) in \(G\) is \(8 / 2 = 4\).
\end{proof}

\newpage

%%%%%%%%%%%%%%%%%%%% PROBLEM 5 %%%%%%%%%%%%%%%%%%%%

\begin{problem} 
    Let $G$ be a finite group with more than one element. Show that $G$ has an element of prime order.
\end{problem}

\end{document}