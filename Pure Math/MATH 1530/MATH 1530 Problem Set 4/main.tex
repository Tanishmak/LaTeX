\documentclass[12pt,reqno]{article}

%%%%%%%%%%%%%%%%%%%% PACKAGES %%%%%%%%%%%%%%%%%%%%

\usepackage[utf8]{inputenc}
\usepackage[all]{xy}
\usepackage[T1]{fontenc}
\usepackage[usenames, dvipsnames]{color}
\usepackage{setspace}
\usepackage{dsfont}
\usepackage{amssymb}
\usepackage{amsthm,bbm}
\usepackage{amscd}
\usepackage{amsfonts}
\usepackage{stmaryrd}
\usepackage{amsmath}
\usepackage{graphicx}
\usepackage{multicol}
\usepackage{xspace}
\usepackage{extarrows}
\usepackage{color}
\usepackage [english]{babel}
\usepackage [autostyle, english = american]{csquotes}
\usepackage[colorlinks, linktocpage, citecolor = red, linkcolor = blue]{hyperref}
\usepackage{fullpage}
\usepackage{color}
\usepackage{euler}
\usepackage{parskip}

%%%%%%%%%%%%%%%%%%%% INITIALIZATION %%%%%%%%%%%%%%%%%%%%

\MakeOuterQuote{"}
\graphicspath{ {./} }

%%%%%%%%%%%%%%%%%%%% COMMANDS %%%%%%%%%%%%%%%%%%%%

\newcommand{\range}{\mathrm{range\,}}
\newcommand{\nul}{\mathrm{null\,}}
\newcommand{\spn}{\mathrm{span\,}}
\newcommand{\card}{\mathrm{cardinality}}
\newcommand{\R}{\mathbb{R}}
\newcommand{\C}{\mathbb{C}}
\newcommand{\F}{\mathbb{F}}
\newcommand{\Z}{\mathbb{Z}}
\newcommand{\bd}{\mathrm{bd\,}}
\newcommand{\divline}{\hrule\vspace{12pt}\noindent}
\newcommand{\sgn}{\mathrm{sgn}}

%%%%%%%%%%%%%%%%%%%% ENVIRONMENTS %%%%%%%%%%%%%%%%%%%%

\theoremstyle{plain}
\newtheorem{maintheorem}{Theorem}
\renewcommand*{\themaintheorem}{\Alph{maintheorem}}

\newtheorem{theorem}{Theorem}[section] 
\newtheorem{lemma}{Lemma}
\newtheorem{corollary}[theorem]{Corollary}

\theoremstyle{definition}
\newtheorem{problem}{Problem}
\newtheorem{example}[theorem]{Example}
\newtheorem{definition}[theorem]{Definition}
\newtheorem{question}[theorem]{Question}

\newtheorem*{maintheorema}{Theorem \ref{thm:main}}

%%%%%%%%%%%%%%%%%%%% TITLE-PAGE %%%%%%%%%%%%%%%%%%%%

\title{MATH 1530 Problem Set 3}
\author{Tanish Makadia\\\small{(Collaborated with Esmé, Marcos, Edward, and Kazuya)}}
\date{February 2023}

%%%%%%%%%%%%%%%%%%%% DOCUMENT %%%%%%%%%%%%%%%%%%%%

\begin{document}
\maketitle

%%%%%%%%%%%%%%%%%%%% PROBLEM 1 %%%%%%%%%%%%%%%%%%%%

\begin{problem}
    Please complete the mid-semester survey. Write "I have completed the mid-semester survey" and sign your name.
\end{problem}

\newpage

%%%%%%%%%%%%%%%%%%%% PROBLEM 2 %%%%%%%%%%%%%%%%%%%%

\begin{problem}
    Let \(a\) be an element of a group \(G\). Prove that \(\langle a^m\rangle \cap \langle a^n\rangle\)
    is cyclic, where \(n,m\) are integers. What is its generator?
\end{problem}

\begin{proof}
    Let \(a^k\in \langle a^m\rangle \cap \langle a^n\rangle\). We have that \(a^{k}\in\langle a^m\rangle\implies a^k=a^{ms}\) where \(s\in\Z\). 
    We also have that \(a^k\in\langle a^n\rangle\implies a^k=a^{nt}\) where \(t\in\Z\). Together, we have
    \[a^k=a^{ms}=a^{nt}\implies k=ms=nt\]
    In other words, \(k\) must be a common multiple of both \(m\) and \(n\). Since every common multiple of \(m\) and
    \(n\) is itself a multiple of \(lcm(m,n)\), we have that \(\langle a^m\rangle \cap \langle a^n\rangle\) is equal to 
    \(\{(a^{lcm(m,n)})^c\ |\ c\in\Z\}=\langle a^{lcm(m,n)}\rangle\). 
\end{proof}

\newpage

%%%%%%%%%%%%%%%%%%%% PROBLEM 3 %%%%%%%%%%%%%%%%%%%%

\begin{problem}
    Let \(a\) and \(b\) belong to a group. If \(|a|\) and \(|b|\) are relatively prime, prove that \(\langle a\rangle \cap \langle b\rangle = \{e\}\).
\end{problem}

\begin{proof}
    Let \(G\) be a group containing elements \(a,b\). Let \(m=|a|\) and \(n=|b|\). We can now express \(\langle a\rangle\) and \(\langle b\rangle\) as:
    \begin{equation*}
        \begin{aligned}[c]
            \langle a\rangle = \{e,a^1,\ldots,a^{m-1}\}
        \end{aligned}
        \qquad\qquad
        \begin{aligned}[c]
            \langle b\rangle = \{e,b^1,\ldots,a^{n-1}\}
        \end{aligned}
    \end{equation*}
    
    Because the identity element of \(G\) is unique, we have that \(e\in\langle a\rangle\cap\langle b\rangle\).
    
    Next, we will show that for all \(a^k\in\langle a\rangle\) such that \(a^k\neq e\), we have that \(a^k\notin\langle b\rangle\).
    By (Gallian, 4.2 Corollary 1), we know that if \(a^k\in\langle a\rangle\), then \(|a^k|\) divides \(m\). Additionally, since \(a^k\neq e\),
    we know \(|a^k|>1\). If \(a^k\in\langle b\rangle\), \(|a^k|\) must divide \(n\). But since \(|a|\) and \(|b|\) are relatively prime, we have that \(gcd(m,n)=1\). Because
    \(|a^k|\neq 1\), we have shown that \(a^k\notin\langle b\rangle\). The same process can be used to show that for all \(b^k\in\langle b\rangle\) such that \(b^k\neq e\),
    we have that \(b^k\notin\langle a\rangle\).

    Therefore, we have proven that \(\langle a\rangle\cap\langle b\rangle=\{e\}\).
\end{proof}

\newpage

%%%%%%%%%%%%%%%%%%%% PROBLEM 4 %%%%%%%%%%%%%%%%%%%%

\begin{problem}
    Let \(G\) be an Abelian group of order \(77\), and assume that for all \(x\in G\), we have that
    \(x^{77}=e\). Prove that \(G\) is cyclic.
\end{problem}

\begin{proof}
    For all \(x\in G\), we have \(x^{77}=e\) which implies that \(|x|\) divides \(77\). Thus, 
    for all \(x\in G\), we have that \(|x|\in\{1,7,11,77\}\). To prove that \(G\) is cyclic, we must show that 
    \(G\) has an element of order \(77\).

    Suppose that every element of \(G\) besides \(e\in G\) has order \(7\). Because \(G\) is closed,
    it must contain every cyclic subgroup generated by these elements. For all \(x\in G\), we have \(|x|=|\langle x\rangle|\)
    and \(e\in\langle x\rangle\). This implies that each cyclic subgroup generated by an element of order \(7\) contains \(6\) distinct non-identity elements. 
    Thus, the number of distinct elements in \(G\) is \(|G|=77=1 + 6n\) where \(n\in\Z\). This is a contradiction since \(6\nmid 76\).

    Alternatively, suppose every element of \(G\) besides \(e\in G\) has order \(11\). We have that each cyclic
    subgroup generated by an element of order \(11\) contains \(10\) distinct non-identity elements. In this case,
    \(|G|=77=1+10n\) where \(n\in\Z\). This is also a contradiction since \(10\nmid 76\).

    We are therefore left with the following cases:
    \begin{itemize}
        \item \textbf{Case 1 (\(G\) has an element of order \(7\) and order \(11\)):} Let \(a,b\in G\) such that
        \(|a|=7\) and \(|b|=11\). Thus, \(a^{77}=a^7=e\) and \(b^{77}=b^{11}=e\). This implies that
        \((ab)^{77}=a^{77}b^{77}=e^2=e\). Hence, we have that \(|ab|\) divides \(77\), giving us the following four cases:
        
        \begin{itemize}
            \item \textbf{Case 1 (\(|ab|=1\)):} This implies \(ab=e\) which is a contradiction since \(a\) and \(b\) do not have the same order.
            \item \textbf{Case 2 (\(|ab|=7\)):} This implies \(e=(ab)^7=a^7b^7=e\cdot b^7=b^7\). This is a contradiction since \(|b|=11\).
            \item \textbf{Case 3 (\(|ab|=11\)):} This implies \(e=(ab)^{11}=a^{11}b^{11}=a^{11}\cdot e=a^{11}\). This is a contradiction since \(7\nmid 11\).
            \item \textbf{Case 4 (\(|ab|=77\)):} By process of elimination, we have that \(|ab|=77\).
        \end{itemize}


        \item \textbf{Case 2 (\(G\) has an element of order \(77\)):} We have that \(G\) can be generated
        by this element and we are done.
    \end{itemize}
    Since both cases lead to the existence of an element of order \(77\) in \(G\), we have
    proven that \(G\) is cyclic.
\end{proof}

\newpage

%%%%%%%%%%%%%%%%%%%% PROBLEM 5 %%%%%%%%%%%%%%%%%%%%

Let \(G\) be a group, and suppose \(G\) has two distinct elements of order \(2\).

1. Prove that \(G\) is not cyclic.
\begin{proof}
    Suppose \(G\) is cyclic. Since \(G\) has an element of order \(2\), we have that \(2\) divides \(|G|\). By (Gallian, 4.3),
    \(G\) must have exactly one subgroup of order \(2\). However, \(G\) has two distinct elements of
    order \(2\) which implies that \(G\) has two distinct subgroups of order \(2\). This is a contradiction. Therefore,
    \(G\) is not cyclic.
\end{proof}

2. Prove that \(U(2^n)\) is not cyclic for \(n\geq 3\).
\begin{proof}
    We will show that \(U(2^n)\) contains two distinct elements of order \(2\).

    First, we will prove that \(2^n-1,\ 2^{n-1}-1\in U(2^n)\).
    \begin{itemize}
        \item \underline{\(2^n-1\in U(2^n)\):} Since the linear combination \(2^n-(2^n-1)\) equals \(1\), we have that
        \(2^n-1\) and \(2^n\) are relatively prime.
        \item \underline{\(2^{n-1}-1\in U(2^n)\):} Consider the prime factorizations of \(2^n\) and \(2^{n-1}-1\).
        Of course, \(2^n=2\cdots 2\). Let \(2^{n-1}-1=p_1\cdots p_m\) where \(p_i\) is a prime number. Since \(2^{n-1}\) is even, it must be the case
        that \(2^{n-1}-1\) is odd. Therefore, \(p_1,\ldots,p_m\) are odd. Because the product of even numbers is always even, all divisors of \(2^n\)
        besides \(1\) must be even. Additionally, since the product of odd numbers is always odd, all divisors of \(2^{n-1}-1\) must be odd. 
        Therefore, we have that \(gcd(2^n,\ 2^{n-1}-1)=1\) which means \(2^n\) and \(2^{n-1}-1\) are relatively prime.
    \end{itemize}

    Now, we will show that \(|2^n-1|=|2^{n-1}-1|=2\).
    \begin{equation*}
        \begin{aligned}[c]
            (2^n-1)^2 &= (2^{2n}-2(2^n)+1)\bmod{2^n}\\
            &= (2^n(2^n-2)+1)\bmod{2^n}\\
            &= 1 = e
        \end{aligned}
        \qquad\qquad
        \begin{aligned}[c]
            (2^{n-1}-1)^2 &= (2^{2n-2}-2(2^{n-1})+1)\bmod{2^n}\\
            &= (2^n(2^{n-2}-1)+1)\bmod{2^n}\\
            &= 1 = e
        \end{aligned}
    \end{equation*}
    Therefore, we have that \(|2^n-1|=|2^{n-1}-1|=2\). Since two distinct elements of 
    order \(2\) exist in \(U(2^n)\), by \((1)\), we have proven that \(U(2^n)\) is not cyclic for \(n\geq 3\).
\end{proof}

\end{document}