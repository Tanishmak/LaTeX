% me=0 student solutions (ps file), me=1 - my solutions (sol file),
% me=2 - assignment (hw file)
\def\me{1} \def\num{2} %homework number

\def\due{11 pm, February 17} %due date

\def\course{APMA 1655 Honors Statistical Inference I} 

% **** INSERT YOUR NAME HERE ****
\def\name{Tanish Makadia}


% **** INSERT NAMES OF YOUR COLLABORATORS HERE, OR JUST PUT N/A ****
\def\collabs{N/A}

%

\documentclass[11pt]{article}


% ==== Packages ====
\usepackage{amsfonts}
\usepackage{latexsym}
\usepackage{fullpage}
\usepackage{amsmath}
\usepackage{amssymb}
\usepackage{hyperref}
\usepackage{euler}
\usepackage{amsthm}

% \setlength{\oddsidemargin}{.0in} \setlength{\evensidemargin}{.0in}
% \setlength{\textwidth}{6.5in} \setlength{\topmargin}{-0.4in}
\setlength{\footskip}{1in} \setlength{\textheight}{8.5in}

\newcommand{\handout}[5]{
\renewcommand{\thepage}{#5, Page \arabic{page}}
  \noindent
  \begin{center}
    \framebox{ \vbox{ \hbox to 5.78in { {\bf \course} \hfill #2 }
        \vspace{4mm} \hbox to 5.78in { {\Large \hfill #5 \hfill} }
        \vspace{2mm} \hbox to 5.78in { {\it #3 \hfill #4} }
       
        \vspace{2mm} \hbox to 5.78in { {\it Collaborators: \collabs
            \hfill} }
           
        \vspace{2mm}
        \begin{itemize}
       
       \item You are strongly encouraged to work in groups, but solutions must be written independently. 

        \end{itemize}
      } }
  \end{center}
  
  \vspace*{4mm}
}


\newcounter{pppp}
\newcommand{\prob}{\arabic{pppp}} %problem number
\newcommand{\increase}{\addtocounter{pppp}{1}} %problem number

% Arguments: Title, Number of Points
\newcommand{\newproblem}[2]{
    \increase
    \section*{Problem \prob~(#1) \hfill {#2}}
}



\def\squarebox#1{\hbox to #1{\hfill\vbox to #1{\vfill}}}
\def\qed{\hspace*{\fill}
  \vbox{\hrule\hbox{\vrule\squarebox{.667em}\vrule}\hrule}}
\newenvironment{solution}{\begin{trivlist}\item[]{\bf Solution:}}
  {\qed \end{trivlist}}
\newenvironment{solsketch}{\begin{trivlist}\item[]{\bf Solution
      Sketch:}} {\qed \end{trivlist}}
\newenvironment{code}{\begin{tabbing}
    12345\=12345\=12345\=12345\=12345\=12345\=12345\=12345\= \kill }
  {\end{tabbing}}

%\newcommand{\eqref}[1]{Equation~(\ref{eq:#1})}

\newcommand{\hint}[1]{({\bf Hint}: {#1})}
% Put more macros here, as needed.
\newcommand{\room}{\medskip\ni}
\newcommand{\brak}[1]{\langle #1 \rangle}
\newcommand{\bit}[1]{\{0,1\}^{#1}}
\newcommand{\zo}{\{0,1\}}
\newcommand{\C}{{\cal C}}
\newcommand{\p}{\mathbb{P}}
\newcommand{\tp}{\tilde{\p}}

\newcommand{\nin}{\not\in}
\newcommand{\set}[1]{\{#1\}}
\renewcommand{\ni}{\noindent}
\renewcommand{\gets}{\leftarrow}
\renewcommand{\to}{\rightarrow}
\newcommand{\assign}{:=}

\newcommand{\AND}{\wedge}
\newcommand{\OR}{\vee}

\newcommand{\For}{\mbox{\bf for }}
\newcommand{\To}{\mbox{\bf to }}
\newcommand{\Do}{\mbox{\bf do }}
\newcommand{\If}{\mbox{\bf if }}
\newcommand{\Then}{\mbox{\bf then }}
\newcommand{\Else}{\mbox{\bf else }}
\newcommand{\While}{\mbox{\bf while }}
\newcommand{\Repeat}{\mbox{\bf repeat }}
\newcommand{\Until}{\mbox{\bf until }}
\newcommand{\Return}{\mbox{\bf return }}
\newcommand{\Halt}{\mbox{\bf halt }}
\newcommand{\Swap}{\mbox{\bf swap }}
\newcommand{\Ex}[2]{\textrm{exchange } #1 \textrm{ with } #2}



\begin{document}

\handout{}{\today}{Name: \name{}}{Due: \due}{Homework \num}


Please feel free to use all the results in the Appendix of HW 2 without proving them.

\section{Problem Set}

\begin{enumerate}

\item (2 points) Suppose $(\Omega,\mathbb{P})$ is a probability space, and $B$ is a event with $\mathbb{P}(B)>0$. We define a function $\Tilde{\mathbb{P}}$ of subsets of $\Omega$ by the following
\begin{align*}
\Tilde{\mathbb{P}}(A) \overset{\operatorname{def}}{=} \mathbb{P}(A \,\vert\, B),\ \ \ \text{ for all }A\subset \Omega.
\end{align*}
\textbf{Please prove that $\Tilde{\mathbb{P}}$ is a probability, i.e., $(\Omega,\Tilde{\mathbb{P}})$ is a probability space as well.}

\begin{proof}
  We will prove that \((\Omega,\tp)\) is a probability space by proving the following three axioms:
  \begin{itemize}
    \item \underline{\(\tp(A\subset\Omega)\geq 0\)}: \(\tp(A)=\p(A\,|\,B)\geq 0\)
    \item \underline{\(\tp(\Omega)=1\)}: \(\tp(\Omega)=\p(\Omega\,|\,B)=\frac{\p(\Omega\cap B)}{\p(B)}=\frac{\p(B)}{\p(B)}=1\)
    \item \underline{Countable Additivity}: Let \(A_1,\ldots,A_m\subset\Omega\) be mutually disjoint events.
    \begin{align*}
      \tp(A_1\cup\cdots\cup A_m)&=\p((A_1\cup\cdots\cup A_m)\,|\,B)\\
      &=\frac{\p((A_1\cup\cdots\cup A_m)\cap B)}{\p(B)}\\
      &=\frac{\p((A_1\cap B)\cup\cdots\cup(A_m\cap B))}{\p(B)}\\
      &= \frac{\p(A_1\cap B)+\cdots+\p(A_m\cap B)}{\p(B)}\\
      &= \tp(A_1)+\cdots+\tp(A_m)
    \end{align*}
  \end{itemize}
\end{proof}

\item (1 point) Let $(\Omega,\mathbb{P})$ be a probability space and $n$ be a positive integer. $B_1, B_2,\ldots, B_n$ are events and provide a partition of $\Omega$, i.e.,
\begin{itemize}
    \item $\bigcup_{i=1}^n B_i=\Omega$,
    \item $B_1, B_2,\ldots, B_n$ are mutually disjoint.
\end{itemize}
Let $A$ be any event. \textbf{Please prove that $A\cap B_1, A\cap B_2, A\cap B_3,\ldots, A\cap B_n$ are mutually disjoint}, i.e.,
\begin{align*}
    (A\cap B_i)\cap (A\cap B_j)=\emptyset,\ \ \text{ if }i\ne j.
\end{align*}


\item (2 points) A box contains $w$ white balls and $b$ black balls.  A ball is chosen at random. 
\begin{itemize}
\item If the chosen ball is white, we add $d$ white balls to the box, that is, now there are $w+d$ white balls and $b$ black balls.
\item If the chosen ball is black, we add $d$ black balls to the box, that is, now there are $w$ white balls and $b+d$ black balls. 
\end{itemize}
After adding the $d$ balls, another ball is drawn at random from the box.  \textbf{Show that the probability that the second chosen ball is white does not depend on $d$.} Hint: Use the law of total probability (LTP).


\item (1 point) Suppose the underlying probability space is $(\Omega,\mathbb{P})$. Let $G$ and $H$ be events such that $0<\mathbb{P}(G)<1$ and $0<\mathbb{P}(H)<1$. \textbf{Give a formula for $\mathbb{P}(G|H^c)$ in terms of $\mathbb{P}(G)$, $\mathbb{P}(H)$ and $\mathbb{P}(G\cap H)$ only.}


\item (1 point) Suppose we have the following
\begin{align*}
    & \mathbb{P}(\text{``snow today"})=30\%, \\
    & \mathbb{P}(\text{``snow tomorrow"})=60\%, \\
    & \mathbb{P}(\text{``snow today and tomorrow"})=25\%.
\end{align*}
\textbf{Given that it snows today, what is the probability that it will snow tomorrow?}


\item (3 points) Let $(\Omega,\mathbb{P})$ be a probability space. Suppose we have two events $A$ and $B$ such that $\mathbb{P}(A)>0$ and $\mathbb{P}(B)>0$. \textbf{Please prove that the following three equations are equivalent.}
    \begin{enumerate}
        \item $\mathbb{P}(A\,\vert\,B)=\mathbb{P}(A)$,
        \item $\mathbb{P}(A\cap B)=\mathbb{P}(A)\cdot \mathbb{P}(B)$,
        \item $\mathbb{P}(B\,\vert\,A)=\mathbb{P}(B)$.
    \end{enumerate}

\end{enumerate}


\newpage

\section{Appendix}

Please feel free to use all the results in the appendix without proving them.

\subsection{Appendix 1}

Let $A$, $B$, and $C$ be events. Then, we have
\begin{itemize}
\item (Commutative Law ) $A\cup B=B\cup A$,
\item (Commutative Law ) $A\cap B= B\cap A$,
\item (Associative Law) $(A\cup B)\cup B=A\cup (B\cup C)$,
\item (Associative Law) $(A\cap B)\cap C= A\cap(B\cap C)$,
\item (Distributive law) $(A\cup B)\cap C=(A\cap C)\cup (B\cap C)$,
\item (Distributive law) $(A\cap B)\cup C=(A\cup C)\cap (B\cup C)$,
\item $ (A\cup B)^c=A^c\cap B^c$,
\item $ (A\cap B)^c=A^c\cup B^c$.
\end{itemize}

\subsection{Appendix 2}

Let $A_1,A_2,\ldots$ be any sequence of events and $B$ be an event. We have the following
\begin{align*}
& \left(\bigcup_{n=1}^\infty A_n\right)^c = \bigcap_{n=1}^\infty A_n^c, \\
& \left(\bigcap_{n=1}^\infty A_n\right)^c = \bigcup_{n=1}^\infty A_n^c, \\
& B\cap\left(\bigcup_{n=1}^\infty A_n\right) = \bigcup_{n=1}^\infty (B\cap A_n),\\
& B\cup\left(\bigcap_{n=1}^\infty A_n\right) = \bigcap_{n=1}^\infty (B\cup A_n).
\end{align*}

\end{document}