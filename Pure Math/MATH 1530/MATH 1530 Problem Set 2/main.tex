\documentclass[12pt,reqno]{article}

%%%%%%%%%%%%%%%%%%%% PACKAGES %%%%%%%%%%%%%%%%%%%%

\usepackage[utf8]{inputenc}
\usepackage[all]{xy}
\usepackage[T1]{fontenc}
\usepackage[usenames, dvipsnames]{color}
\usepackage{setspace}
\usepackage{dsfont}
\usepackage{amssymb}
\usepackage{amsthm,bbm}
\usepackage{amscd}
\usepackage{amsfonts}
\usepackage{stmaryrd}
\usepackage{amsmath}
\usepackage{graphicx}
\usepackage{multicol}
\usepackage{xspace}
\usepackage{extarrows}
\usepackage{color}
\usepackage [english]{babel}
\usepackage [autostyle, english = american]{csquotes}
\usepackage[colorlinks, linktocpage, citecolor = red, linkcolor = blue]{hyperref}
\usepackage{fullpage}
\usepackage{color}
\usepackage{euler}

%%%%%%%%%%%%%%%%%%%% INITIALIZATION %%%%%%%%%%%%%%%%%%%%

\MakeOuterQuote{"}
\graphicspath{ {./} }
\setlength{\parskip}{\baselineskip}
\setlength{\parindent}{0pt}

%%%%%%%%%%%%%%%%%%%% COMMANDS %%%%%%%%%%%%%%%%%%%%

\newcommand{\range}{\mathrm{range\,}}
\newcommand{\nul}{\mathrm{null\,}}
\newcommand{\spn}{\mathrm{span\,}}
\newcommand{\card}{\mathrm{cardinality}}
\newcommand{\R}{\mathbb{R}}
\newcommand{\C}{\mathbb{C}}
\newcommand{\F}{\mathbb{F}}
\newcommand{\Z}{\mathbb{Z}}
\newcommand{\bd}{\mathrm{bd\,}}
\newcommand{\divline}{\hrule\vspace{12pt}\noindent}
\newcommand{\sgn}{\mathrm{sgn}}

%%%%%%%%%%%%%%%%%%%% ENVIRONMENTS %%%%%%%%%%%%%%%%%%%%

\theoremstyle{plain}
\newtheorem{maintheorem}{Theorem}
\renewcommand*{\themaintheorem}{\Alph{maintheorem}}

\newtheorem{theorem}{Theorem}[section] 
\newtheorem{lemma}{Lemma}
\newtheorem{corollary}[theorem]{Corollary}

\theoremstyle{definition}
\newtheorem{problem}{Problem}
\newtheorem{example}[theorem]{Example}
\newtheorem{definition}[theorem]{Definition}
\newtheorem{question}[theorem]{Question}

\newtheorem*{maintheorema}{Theorem \ref{thm:main}}

%%%%%%%%%%%%%%%%%%%% TITLE-PAGE %%%%%%%%%%%%%%%%%%%%

\title{MATH 1530 Problem Set 2}
\author{Collaborated with Esmé and Kazuya}
\date{February 2023}

%%%%%%%%%%%%%%%%%%%% DOCUMENT %%%%%%%%%%%%%%%%%%%%

\begin{document}
\maketitle

%%%%%%%%%%%%%%%%%%%% PROBLEM 1 %%%%%%%%%%%%%%%%%%%%

\begin{problem} Prove that the set $\{5,15,25,35\}$ is a group under multiplication mod 40.
\end{problem}

\begin{proof}
    Consider the \emph{Cayley Table} of this set:
    
    \begin{tabular}{c | c c c c}
         & 5 & 15 & 25 & 35 \\
         \cline{1-5}
         5 & 25 & 35 & 5 & 15\\
         15 & 35 & 25 & 15 & 5\\ 
         25 & 5 & 15 & 25 & 35\\ 
         35 & 15 & 5 & 35 & 25
    \end{tabular}

    Closure: As can be seen in the table, $a,b\in\{5,15,25,35\}$ implies $ab\bmod{40}\in\{5,15,25,35\}$.

    Identity: $(25\cdot a)\bmod{40} = (a\cdot25)\bmod{40}=a$ for all $a\in\{5, 15, 25, 35\}$. Therefore, $25$ is the identity of this group.

    Inverse: Let $a \in \{5, 15, 25, 35\}$. There exists some $b \in \{5, 15, 25, 35\}$ such that $ab\bmod{40} = ba\bmod{40} = 25$.

    Associativity: We will first prove the following lemma.
    \begin{lemma}
    \label{lem:modulo}
    Multiplication modulo $n\in\Z$ is associative over the integers.
    \begin{proof}
        Let $a,b,c,n\in\Z$.
        \begin{align*}
            (ab\bmod{n})c\bmod{n} &= (ab\bmod{n} \cdot c\bmod{n})\bmod{n} & \text{(definition of $\bmod$)}\\
            &= (ab\cdot c)\bmod{n} & \text{(definition of $\bmod$)}\\
            &= (a\cdot bc)\bmod{n} & \text{(associativity)}\\
            &= (a\bmod{n}\cdot bc\bmod{n})\bmod{n} & \text{(definition of $\bmod$)}\\
            &= a(bc\bmod{n})\bmod{n} & \text{(definition of $\bmod$)}
        \end{align*}
    \end{proof}
    \end{lemma}
    
    Associativity follows immediately from lemma \ref{lem:modulo}.
    \end{proof}
    
\newpage

%%%%%%%%%%%%%%%%%%%% PROBLEM 2 %%%%%%%%%%%%%%%%%%%%

\begin{problem} 
For any integer $n > 2$, prove that there are at least two elements of $U(n)$ that satisfy $x^2 = 1$. 
\end{problem}

\begin{proof}
    For all integers $n>2$, we have that $1,(n-1)\in U(n)$ since the linear combinations $1(1) + n(0)$ and $n(1)+(n-1)(-1)$ both equal $1$.

    Now, we will prove that both satisfy $x^2\bmod{n}=1$.
    \begin{align*}
        (n-1)^2\bmod{n} &= (n^2-2n+1)\bmod{n} & \text{(distributive property)}\\
        &= (n(n-2) + 1)\bmod{n} & \text{(polynomial division)}\\
        &= 1 & \text{(definition of $\bmod$)}
    \end{align*}
    \begin{align*}
        1^2 \bmod{n} &= 1 & \text{(definition of $\bmod$)}
    \end{align*}
\end{proof}

\newpage

%%%%%%%%%%%%%%%%%%%% PROBLEM 3 %%%%%%%%%%%%%%%%%%%%

\begin{problem} 
Prove that the set $\{1,2,\ldots,n-1\}$ is a group under multiplication mod $n$ if and only if $n$ is prime.
\end{problem}

\begin{proof}
Let $S = \{1,2,\ldots,n-1\}$.

$\Rightarrow$ Assume $S$ is a group under multiplication modulo $n$. Suppose $n$ is not prime. Then, there exists some $b,q\in S$ such that $bq=n$. Thus, $bq\bmod{n}=0\notin S$. We have reached a contradiction, since $S$ being a group implies that it is closed. Thus, $n$ must be prime.

$\Leftarrow$ Assume $n$ is prime. We will prove that $S$ is a group under multiplication modulo $n$.

Closure: Let $a,b\in S$. By definition, $0\leq ab\bmod{n}<n$. Additionally, $ab\bmod{n}\neq 0$ because this would imply that $ab=nq$ where $q\in\Z$. By Euclid's Lemma, this implies that $n$ divides $a$ or $b$ which cannot be true. Therefore, since $0 < ab\bmod{n} < n$ we have proven that $ab\bmod{n}\in S$.

Associativity: By lemma \ref{lem:modulo}, we have that multiplication modulo $n$ is associative over the integers.

Identity: $1\in S$ is the identity since for all $a\in S$, $(1\cdot a)\bmod{n}=(a\cdot 1)\bmod{n}=a$.

Inverses: We will first prove the following lemma.
\begin{lemma}
    \label{lem:mod1}
    Let $a,x,n\in\Z_{>0}$. $ax\bmod{n}=1$ has a solution if and only if $a$ and $n$ are relatively prime.
    \begin{proof}
        $\Rightarrow$ Assume $ax\bmod{n}=1$ has a solution. Then, $ax-nq=1$ where $q$ is the largest integer such that $nq\leq ax$. Since we have shown that a linear combination of $a$ and $n$ which equals $1$ exists, we have proven that they are relatively prime.

        $\Leftarrow$ Assume $a$ and $n$ are relatively prime. For some $x,q\in\Z$, we have that the linear combination $ax + n(-q) = 1$. Thus, we have proven that $ax\bmod{n}=1$ has a solution.
    \end{proof}
\end{lemma}
Let $a\in S$. $a$ is relatively prime with $n$ because $n$ itself is prime. By lemma \ref{lem:mod1}, there exists some $x\in\Z_{>0}$ such that $ax\bmod{n}=1$.
\begin{align*}
    1 &= ax\bmod{n} & \text{(lemma \ref{lem:mod1})}\\
    &= (a\bmod{n} \cdot x\bmod{n})\bmod{n} & \text{(definition of $\bmod$)}\\
    &= (a \cdot x\bmod{n})\bmod{n} & \text{($a<n$)}
\end{align*}
Let $b=x\bmod{n}$. We have that $b\in S$ since $0<x\bmod{n}<n$. Therefore, we have proven that for all $a\in S$, there exists some $b\in S$ such that $ab\bmod{n}=ba\bmod{n}=1$.
\end{proof}

\newpage

%%%%%%%%%%%%%%%%%%%% PROBLEM 4 %%%%%%%%%%%%%%%%%%%%

\begin{problem} Suppose $G$ is a group with identity $e$ such that for all $x \in G$, $x^2 = e$. Prove that $G$ is Abelian.
\end{problem}

\begin{proof}
    Let $a,b\in G$. We will prove that the group operation is commutative by showing that $ab=ba$.
    \begin{align*}
        a &= a\\
        a(bb) &= a & \text{($b^2=e$)}\\
        (ab)b &= a & \text{(associativity)}\\
        (ab)(ba) &= a^2 & \text{(right multiply $a$)}\\
        (ab)(ba) &= e & \text{($a^2=e$)}\\
        ab &= ba & \text{(definition of $G$)}
    \end{align*}
\end{proof}

\newpage

%%%%%%%%%%%%%%%%%%%% PROBLEM 5 %%%%%%%%%%%%%%%%%%%%

\begin{problem} Let $G$ be a finite group with identity $e$. Show that the number of elements $x$ of $G$ such that $x^2 \not = e$ is even.
\end{problem}

\begin{proof}
    Let $S\subseteq G$ be the set of elements in $G$ that are not their own inverse. For all $x\in S$, there must be some unique $x'\in S$ such that $x\neq x'$ and $xx'=x'x=e$ (Gallian, 2.3). Let $K\subseteq S = \{x,x'\}$. For any other $y\in S\setminus K$, we have that there exists another $y'\neq y$ that is in $S\setminus K$; together, these two elements form another set $K_2\subseteq S = \{y,y'\}$ that is disjoint from $K$. Thus, this process can be repeated until we can partition $S$ cleanly into sets of exactly two elements. This implies that the elements of $S$ can be divided evenly into pairs, proving that the cardinality of $S$ is even.
\end{proof}

\end{document}