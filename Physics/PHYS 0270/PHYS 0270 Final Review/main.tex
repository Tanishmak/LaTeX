\documentclass[11pt,reqno]{article}

%%%%%%%%%%%%%%%%%%%% PACKAGES %%%%%%%%%%%%%%%%%%%%

\usepackage[utf8]{inputenc}
\usepackage[all]{xy}
\usepackage[T1]{fontenc}
\usepackage[usenames, dvipsnames]{color}
\usepackage{setspace}
\usepackage{dsfont}
\usepackage{amssymb}
\usepackage{amsthm,bbm}
\usepackage{amscd}
\usepackage{amsfonts}
\usepackage{stmaryrd}
\usepackage{amsmath}
\usepackage{graphicx}
\usepackage{multicol}
\usepackage{xspace}
\usepackage{extarrows}
\usepackage{color}
\usepackage [english]{babel}
\usepackage [autostyle, english = american]{csquotes}
\usepackage[colorlinks, linktocpage, citecolor = red, linkcolor = blue]{hyperref}
\usepackage{fullpage}
\usepackage{color}
\usepackage{euler}

%%%%%%%%%%%%%%%%%%%% INITIALIZATION %%%%%%%%%%%%%%%%%%%%

\MakeOuterQuote{"}
\graphicspath{ {./} }

%%%%%%%%%%%%%%%%%%%% COMMANDS %%%%%%%%%%%%%%%%%%%%

\newcommand{\range}{\mathrm{range\,}}
\newcommand{\nul}{\mathrm{null\,}}
\newcommand{\spn}{\mathrm{span\,}}
\newcommand{\card}{\mathrm{cardinality}}
\newcommand{\R}{\mathbb{R}}
\newcommand{\C}{\mathbb{C}}
\newcommand{\F}{\mathbb{F}}
\newcommand{\bd}{\mathrm{bd\,}}
\newcommand{\divline}{\hrule\vspace{12pt}\noindent}
\newcommand{\sgn}{\mathrm{sgn}}

%%%%%%%%%%%%%%%%%%%% ENVIRONMENTS %%%%%%%%%%%%%%%%%%%%

\theoremstyle{definition}
\newtheorem{problem}{Problem}

%%%%%%%%%%%%%%%%%%%% TITLE-PAGE %%%%%%%%%%%%%%%%%%%%

\title{PHYS 0270 Final Review}
\author{Tanish Makadia}
\date{December 2022}

%%%%%%%%%%%%%%%%%%%% DOCUMENT %%%%%%%%%%%%%%%%%%%%

\begin{document}
\maketitle

%%%%%%%%%%%%%%%%%%%% MAIN CONTENT %%%%%%%%%%%%%%%%%%%%

\newpage

\section*{Lecture 1}
Meridian\\\\
Right Ascension\\\\
Sidereal Time\\\\
Local Sidereal Time $LST = HA + RA$\\\\
Rising and Setting\\\\

\section*{Lecture 2}
Sun's Path\\\\
Declination\\\\
Ecliptic\\\\
Calendars\\\\

\section*{Lecture 3}
Moon Phases\\\\
Lunar Cycles\\\\
Eclipses\\\\

\section*{Lecture 4}
Parallax\\\\
Telescopes\\\\
Lenses\\\\
Magnification\\\\
Resulution and Plate Scale\\\\
Atmospheric Opacity\\\\
Waves\\\\
Inferometry\\\\

\section*{Lecture 5}
Flux and Luminosity\\\\
Inverse Square Law\\\\
Magnitudes\\\\
Blackbodies\\\\
Weins and Boltzmann Laws\\\\
Emission and Absorption\\\\
Balmer Sequence\\\\
Energy Levels\\\\

\section*{Lecture 6}
Circular Motion\\\\
Kepler's Laws\\\\
2 Body Problem\\\\
Doppler Effect\\\\
Escape Velocity\\\\
Virial Theorem and Lagrange Points\\\\

\section*{Lecture 7}
Disks and Accretion\\\\
Oligarchical Phase\\\\
Temperature across Disk (and flux/luminosity)\\\\
Composition of Planets\\\\
Atmospheres and Greenhouse Effect\\\\

\section*{Lecture 8}
Magnetic Fields\\\\
Aurorae\\\\
Types of Moons\\\\
Bombardment\\\\
Tides\\\\
Procession\\\\
Composition of Planets\\\\
Roche Limit (no moons) $\rightarrow$ Rings\\\\
Asteroids, KBOs, Comets, and Planet IX\\\\

\section*{Lecture 9}
Kuiper Belt Objects\\\\
Planet IX\\\\
Oort Cloud\\\\
OBAFGKM\\\\
Boltzmann Equation\\\\
Ionization\\\\
HR Diagram\\\\
Size/Mass Relation\\\\
Main Sequence\\\\
Center of Mass\\\\
Period and Velocity\\\\
Orbit Size\\\\
Binary System Orbits\\\\
Eclipsing Binaries\\\\
Mass and Luminosity Relationship\\\\

\section*{Lecture 10}
Gravitational Potential\\\\
Clouds\\\\
Jean's Length\\\\
Fragmentation\\\\
Infall Time\\\\
Heat Transport\\\\
Diff Eqs

\section*{Lecture 11}
Stellar Structure Equations Including Pressure\\\\
Stefan-Boltzmann Law and its relation to temperature\\\\
Convective Transport\\\\
Optical Depth and Equations:
\begin{align*}
    \sigma = \pi r^2\\
    \tau = n\sigma\lambda
\end{align*}
Absorption Coefficients and Mean Free Path Equation:
\begin{align*}
    1 / n\sigma
\end{align*}
X-Ray Corona\\\\
Chromosphere\\\\
Photosphere\\\\
Sunspots\\\\

\section*{Lecture 12}
Weak Force and Deuterium one in $10^{27}$ Collisions
PP-I Chain\\\\
Stellar Lifetimes\\\\
Mass vs. Final Outcome\\\\

\section*{Lecture 13}
Helium Flash\\\\
HR Diagrams\\\\
Electron Degeneracy\\\\
Giant Stars\\\\
Helium Shell\\\\
Heat Pulsations\\\\
Cepheid Variables\\\\
Leavitt's Law and Standard Candles:
\begin{align*}
    m - M = 5\log(d) - 5
\end{align*}
Planetary Nebulae\\\\
White Dwarfs\\\\
Chandrasekhar Mass\\\\
Massive Stars (> 8 Solar Masses)\\\\
P-P vs. CNO\\\\
Layers of Ash - Fusion Diagram\\\\

\section*{Lecture 14}
Iron Death Star Collapse\\\\
Photodisintegration\\\\
Supernovae and Remnants\\\\
Special Relativity\\\\
General Relativity\\\\
Degeneracy Pressure\\\\
Neutron Star Rotation\\\\
Pulsars\\\\

\section*{Lecture 15}
General Relativity\\\\
Interferometers (LIGO)\\\\
Merger Events with Black Holes and Neutron Stars\\\\
Escape Velocity:
\begin{align*}
    v_{esc} = \sqrt{\frac{2GM}{R}}
\end{align*}
Schwarzschild Radius or Event Horizon:
\begin{align*}
    R_s = \frac{2GM}{c^2}
\end{align*}
Tidal forces (difference in force between head and foot):
\begin{align*}
    \frac{2GMm}{r^3}
\end{align*}
Light Distortion by Gravity\\\\
Hawking Radiation\\\\
Binary Systems and Roche Lobes\\\\
Roche Lobes $\rightarrow$ Accretion Disk:
\begin{align*}
    L = mv_{rot}R\\
    a = v^2 / r
\end{align*}
Accretion Disk Luminosities:
\begin{align*}
    L = \frac{dM}{dt}GM(\frac{1}{r_f}-\frac{1}{r_i})
\end{align*}
Accretion Disk Temperature and Virial Theorem\\\\
Novae and Types\\\\
Virial Theorem; Potential/Kinetic Energy for a Uniform Density Sphere of Stars:
\begin{align*}
    U = -\frac{3}{5}\frac{GM^2}{r}\\
    \langle K\rangle = \frac{1}{2} M\langle v^2\rangle\\
    \langle v^2\rangle = \frac{3}{5} \frac{GM}{r}\\
    \langle v_r^2\rangle = \frac{\langle v^2\rangle}{3}\\
    M = 5r\langle v_r^2\rangle / G
\end{align*}
Population I and II Stars\\\\
Interactions of Stars within a Cluster:
\begin{align*}
    t_s = 1 / (n\pi r^2 v)\\
    t = \frac{v^3}{4\pi G^2 m^2 n}\\
    t = NR/v
\end{align*}

\section*{Lecture 16}
Herschel and Kapteyn's Models\\\\
Galactic Dust\\\\
Globular Clusters and Galactic Center\\\\
Extinction $A$:
\begin{align*}
    A = m_{observed} - m_{true} = 2.5 \log(I_{true}/ I_{observed})
\end{align*}
Shift of measured magnitudes:
\begin{align*}
    m = M + 5\log(d/10) + A
\end{align*}
Color dependence based on dust\\\\
Blue Scattered Light\\\\
Composition and Temperature of Dust Grains:
\begin{align*}
    T_{dust} = T \cdot (\frac{R}{2d})^{1/2}
\end{align*}
Distance of HI Dust - Redshift/Blueshift and Infrared\\\\
Giant Bubbles in the Milky Way\\\\
Molecular Clouds:
\begin{align*}
    U = -\frac{3GM^2}{5R}\\
    K = \frac{3MkT}{2m}\\
    \frac{M}{R} = \frac{4\pi R^3\rho}{3R} = \frac{5kT}{2Gm}
\end{align*}
Masers\\\\
Jeans Length and Mass:
\begin{align*}
    R_j = \sqrt{\frac{15kT}{8\pi Gm\rho}}\\
    M_j = 4\pi R_j^3 / 3
\end{align*}
Gas Collapse\\\\
O and B Stars and Photoionization\\\\
HII Regions and Stromgren Radius\\\\
Disks, Jets, and Outflows\\\\
T-Tauri Stars\\\\

\section*{Lecture 17}
Measuring Shift of HI Clouds\\\\
Galactic Arms\\\\
Spiral Arms as Density Waves\\\\
Components of a Galaxy\\\\
Bulges and Halos\\\\
Spiral Galaxy Classification\\\\
Grand Design vs. Flocculent\\\\
Elliptical Galaxies\\\\
Irregular Galaxies\\\\
Rotation Curves $\rightarrow$ Dark Matter:
\begin{align*}
    \rho(r) = \frac{v_0^2}{4\pi Gr^2}
\end{align*}
WIMPs\\\\
Shapley Curtis Debate\\\\

\section*{Lecture 18}
Cepheids to measure distance to galaxies\\\\
The Hubble Law:
\begin{align*}
    v = Hd
\end{align*}
Tully-Fisher Relation\\\\
Distance Ladder\\\\
Cosmic Acceleration\\\\
Local Group\\\\
Virgo Cluster\\\\
Local Supercluster\\\\
Coma Cluster\\\\
Clusters of Galaxies and X-ray emitting hot gas\\\\
Dark matter in clusters\\\\
Gravitational Lensing\\\\
Filamentary Structures\\\\
Redshift Space Distortions\\\\
Universe is Homogenous and Isotropic\\\\

\section*{Lecture 19}
Hubble Deep Field and Galaxies\\\\
Red Shift Galaxies\\\\
Starburst Galaxies\\\\
Galaxy Evolution\\\\
Active Galaxies\\\\
Radio Galaxies\\\\
Seyfert Galaxies\\\\
Quasars and QSO's\\\\
Blazars\\\\
Active Galactic Nuclei\\\\
Eddington Luminosity\\\\
Lyman Alpha Forest\\\\
Gunn-Peterson trough\\\\
Gamma Ray Bursts and Afterglow\\\\
Cosmology and why the night sky is not dark\\\\
Olber's Paradox\\\\

\section*{Lecture 20}
Big Bang Model\\\\
Hubble Law\\\\
Expansion and Scale Factor:
\begin{align*}
    H^2 = \frac{8\pi G\rho}{3} - \frac{kc^2}{R^2}
\end{align*}
Various Densities of Matter\\\\
Matter vs. Radiation:
\begin{align*}
    \rho_c = \frac{3H^2}{8\pi G}
\end{align*}
Curvature\\\\
Quantized Gravity and Inflation Epoch\\\\
Planck time\\\\
Matter from Energy\\\\
Annihilations and $kT > mc^2$\\\\
Freeze Out\\\\
Baryogenesis\\\\
Asymmetry\\\\
Helium Bottleneck\\\\
Recombination\\\\
Heat Death and Big Rip\\\\

\section*{Lecture 21}
Friedman Equations for Hubble's Constant\\\\
Recombination and "Cosmic Photosphere"\\\\
Cosmic Microwave Background\\\\
Fluctuations in CMB\\\\
Cosmological Parameters\\\\
Unsolved Problems in Cosmology\\\\
Flatness Problem\\\\
Horizon Problem\\\\
Structure Formation Problem\\\\
Topological Defect Problem\\\\

\section*{Lecture 22}
Inflation\\\\
Quantum Fluctuations\\\\
Cosmic Acceleration\\\\
Hubble's Constant\\\\

\section*{Lecture 23}
Direct Imaging of Exoplanets\\\\
Astrometric Method\\\\
Doppler Method\\\\
Hot Jupiters\\\\
Highly eccentric planet orbits\\\\
Habitable Zone\\\\
Exoplanet Transits\\\\
Planet Migration and Ice Lines\\\\

\section*{Lecture 24}
Search for Life\\\\
Planets and liquid water\\\\
JWST atmospheric composition\\\\
Tree of Life\\\\

\section*{Lecture 25}
Drake Equation: $N = R^* \cdot fp\cdot ne\cdot fl\cdot fi\cdot fc\cdot L$\\\\
Cosmic Rays and Cerenkov Radiation\\\\
Neutrino Detection\\\\
Lensing\\\\


\end{document}